\documentclass[11pt]{book}
\usepackage{ifpdf}
\frenchspacing
\newfont{\HUGE}{cmr17 at 96pt}
\setlength{\textwidth}{3.0in}
\setlength{\textheight}{5.125in}
\setlength{\oddsidemargin}{0.0in}
\setlength{\evensidemargin}{-0.375in}
\setlength{\topmargin}{-0.375in}
\setlength{\headsep}{0.25in}
\setlength{\headheight}{0.125in}
\setlength{\footskip}{0.25in}
\setlength{\parindent}{0.5\parindent}
\ifpdf
\pdfpageheight 7.5in
\pdfpagewidth 4.625in
\else
%\setlength{\paperheight}{7.5in}
%\setlength{\paperwidth}{4.625in}
\addtolength{\oddsidemargin}{1.9375in}
\addtolength{\evensidemargin}{1.9375in}
\addtolength{\topmargin}{1.75in}
\fi
\newcommand{\pgtm}{Project \mbox{Gutenberg-tm}}
\newcommand{\PGtm}{PROJECT \mbox{GUTENBERG-tm}}
\newcommand{\hstroke}{\rule[0.5ex]{5.0em}{0.2ex}}
\newcommand{\indix}{\hspace*{\parindent}}
\newcommand{\indpar}{\par\noindent\hspace*{\parindent}}
\newcommand{\ingredient}{\indpar}
\newcommand{\instruction}{\indpar}
\newcommand{\OneHalf}{\ensuremath{{}^1\!\!/\!{}_2\mbox{\ }}}
\newcommand{\OneQuarter}{\ensuremath{{}^1\!\!/\!{}_4\mbox{\ }}}
\newcommand{\ThreeQuarters}{\ensuremath{{}^3\!\!/\!{}_4\mbox{\ }}}
\newcommand{\ixfill}{\leaders\hbox to 1.5em{\hss.\hss}\hfill}
\newenvironment{RecipeTitle}{\medskip\begin{center}\large\bf }{\end{center}\smallskip}
\newenvironment{FoodTypeTitle}{\begin{center}\large\bf }{\end{center}}
\begin{document}
\thispagestyle{empty}
\vspace*{10ex}
\noindent {\small \em The Project Gutenberg eBook, A Little Cook Book for a Little Girl,
by Caroline French Benton}\\
\medskip\par\noindent
{\small This eBook is for the use of anyone anywhere at no cost and with almost
no restrictions whatsover.  You may copy it, give it away or re-use it
under the terms of the Project Gutenberg License included with this
eBook or online at {\em www.gutenberg.net}}
\bigskip\par\noindent
{\footnotesize
Title: A Little Cook Book for a Little Girl\\
Author: Caroline French Benton\\
Release Date: August 12, 2005 [eBook \#16514]\\
Language: English\\
Character set encoding: ISO-8859-1}
\bigskip\par\noindent
{\scriptsize *** START OF THE PROJECT GUTENBERG EBOOK\\
$\phantom{\mbox{\scriptsize *** }}$A LITTLE COOK BOOK FOR A LITTLE GIRL ***}\\
\medskip\par\noindent
{\footnotesize\em This eBook was prepared by Stewart A.\ Levin.}
\newpage
\thispagestyle{empty}
\ 
\newpage
\thispagestyle{empty}
\begin{center}
\setlength{\unitlength}{1.0em}%
\begin{picture}(45,16)(0,0)
\put(-0.625,0){\framebox(6.2,7.25){\HUGE A}}
\put(5.875,5.625){\parbox[t]{15em}{\Huge \noindent LITTLE COOK\linebreak
BOOK FOR A\linebreak
LITTLE GIRL\linebreak
}}
\end{picture}
\hspace*{-0.6em}\rule{1.08\textwidth}{0.3ex}\\
{\small \sc by}\\
{\small CAROLINE FRENCH BENTON}\\
{\footnotesize Author of ``Gala Day Luncheons''}\\
\vspace*{18.5 ex}
\rule{\textwidth}{0.3ex}\\
{\small
BOSTON\hfill\raisebox{0.5ex}{$\bullet$}\hfill THE PAGE\linebreak
COMPANY\hfill\raisebox{0.5ex}{$\,\bullet$}\hfill PUBLISHERS
}
\end{center}
\newpage
\thispagestyle{empty}
\vspace*{18ex}
\begin{center}
{\em Copyright, $\mit 1905$}\\
{\sc By Dana Estes \& Company}\\
\hstroke\\
{\em All rights reserved}\\
\vspace*{18ex}
{\sc A little cook book}\\
{\sc for a little girl}\\
\vspace*{12ex}
Made in U.S.A.
\end{center}
\newpage
\frontmatter
\thispagestyle{empty}
\vspace*{20ex}
\begin{center}
                           {\scriptsize FOR}\\
                {\large \bf Katharine, Monica and Betty}\\
\smallskip
                   {\footnotesize THREE LITTLE GIRLS\\
                     WHO LOVE TO DO\\
                   ``LITTLE GIRL COOKING''}
\end{center}
\newpage
\thispagestyle{empty}
\ 
\newpage
\thispagestyle{empty}
\vspace*{30ex}
\indpar Thanks are due to the editor of {\it Good Housekeeping\/} for
permission to reproduce the greater part of this book
from that magazine.
\newpage
\thispagestyle{empty}
\ 
\newpage
\markboth{INTRODUCTION}{INTRODUCTION}
\thispagestyle{plain}
\vspace*{3ex}
\begin{center}
{\Large INTRODUCTION}\\
\hstroke
\end{center}
\indpar
  Once upon a time there was a little girl named Margaret, and she
wanted to cook, so she went into the kitchen and tried and tried, but
she could not understand the cook-books, and she made dreadful messes,
and spoiled her frocks and burned her fingers till she just had to cry.
\indpar
  One day she went to her grandmother and her mother and her Pretty
Aunt and her Other Aunt, who were all sitting sewing, and asked them to
tell here about cooking.
\indpar
  ``What is a roux,'' she said, ``and what's a mousse and what's an
entr\'{e}e?  What are timbales and saut\'{e}s and ingredients, and how do you
mix 'em and how long do you bake 'em?  Won't somebody please tell me all
about it?''
\indpar
  And her Pretty Aunt said, ``See the flour all over that new frock!''
and her mother said, ``Dear child, you are not old enough to cook yet;''
and her grandmother said, ``Just wait a year or two, and I'll teach you
myself;'' and the Other Aunt said, ``Some day you shall go to
cooking-school and learn everything; you know little girls can't cook.''
\indpar
  But Margaret said, ``I don't want to wait till I'm big; I want to
cook now; and I don't want to do cooking-school cooking, but little girl
cooking, all by myself.''
\indpar
  So she kept on trying to learn, $\!$but she burned her fingers and
spoiled her dresses worse than ever, and her messes were so bad they had
to be thrown out, every one of them; and she cried and cried.  And then
one day her grandmother said, ``It's a shame that child should not learn
to cook if she really wants to so much;'' and her mother said ``Yes, it is
a shame, and she shall learn!  Let's get her a small table and some tins
and aprons, and make a little cook-book all her own out of the old ones
we wrote for ourselves long ago,---just the plain, easy things anybody
can make.'' And both her aunts said, ``Do! We will help, and perhaps we
might put in just a few cooking-school things beside.''
\indpar
  It was not long after this that Margaret had a birthday, and she was
taken to the kitchen to get her presents, which she thought the funniest
thing in the world.  There they all were, in the middle of the room:
first her father's present, a little table with a white oilcloth cover
and casters, which would push right under the big table when it was not
being used.  Over a chair her grandmother's present, three nice gingham
aprons, with sleeves and ruffled bibs.  On the little table the presents
of the aunties, shiny new tins and saucepans, and cups to measure with,
and spoons, and a toasting-fork, and ever so many things; and then on
one corner of the table, all by itself, was her mother's present, her
own little cook-book, with her own name on it, and that was best of all.
\indpar
  When Margaret had looked at everything, she set out in a row the big
bowl and the middle-sized bowl and the little wee bowl, and put the
scalloped patty-pans around them, and the real egg-beater in front of
all, just like a picture, and then she read a page in her cook-book, and
began to believe it was all true.  So she danced for joy, and put on a
gingham apron and began to cook that very minute, and before another
birthday she had cooked every single thing in the book.
\indpar
  This is Margaret's cook-book.
\newpage
\thispagestyle{empty}
\ 
\newpage
\thispagestyle{empty}
\vspace*{6ex}
\begin{center}
{\Large CONTENTS}\\
\vspace*{2ex}
\hstroke
\end{center}
\smallskip
{\footnotesize \sc part \hfill page}\linebreak
\hspace*{0.5em}{\footnotesize \sc $\phantom{\mbox{\footnotesize II}}$I.\hspace*{0.5em}The Things Margaret Made For}\hspace*{3em}\linebreak
\hspace*{0.5em}$\phantom{\mbox{\footnotesize \sc $\phantom{\mbox{\footnotesize II}}$I.\hspace*{0.5em}}}$\hspace*{0.5em} {\footnotesize \sc Breakfast}\ixfill{\small \pageref{PART_I}}\linebreak
\hspace*{0.5em}{\footnotesize \sc $\phantom{\mbox{\footnotesize \sc I}}$II.\hspace*{0.5em}The Things She Made For Luncheon}\hspace*{3em}\linebreak
\hspace*{0.5em}$\phantom{\mbox{\footnotesize \sc $\phantom{\mbox{\footnotesize \sc I}}$II.\hspace*{0.5em}}}$\hspace*{0.5em}{\footnotesize \sc or Supper}\ixfill{\small \pageref{PART_II}}\linebreak
\hspace*{0.5em}{\footnotesize \sc III.\hspace*{0.5em}The Things She Made For Dinner}\ixfill{\small \pageref{PART_III}}\linebreak
\newpage
\thispagestyle{empty}
\ 
\mainmatter
\setcounter{page}{11}
\thispagestyle{empty}
\begin{center}
\vspace*{30ex}
{\large \bf PART I.}\\
\ \\
THE THINGS MARGARET MADE FOR BREAKFAST\label{PART_I}\\
\end{center}
\newpage
\thispagestyle{empty}
\ 
\newpage
\markboth{A LITTLE COOK BOOK}{FOR A LITTLE GIRL}
\thispagestyle{plain}
\bigskip
\begin{center}
{\Large A LITTLE COOK BOOK}\\
{\Large FOR A LITTLE GIRL}\\
\bigskip
\hstroke
\end{center}
\begin{FoodTypeTitle}
CEREALS\label{CEREALS}
\end{FoodTypeTitle}
\ingredient  \mit 1 quart of boiling water.
\ingredient  \mit 4 tablespoonfuls of cereal.
\ingredient  \mit 1 teaspoonful of salt.
\indpar
  When you are to use a cereal made of oats or wheat, always begin to cook it
the night before, even if it says on the package that it is not necessary.  Put
a quart of boiling water in the outside of the double boiler, and another quart
in the inside, and in this last mix the salt and cereal.  Put the boiler on the
back of the kitchen range, where it will be hardly cook at all, and let it
stand all night.  If the fire is to go out, put it on so that it will cook for
two hours first.  In the morning, if the water in the outside of the boiler
is cold, fill it up hot, and boil hard for an hour without stirring the
cereal.  Then turn it out in a hot dish, and send it to the table with a
pitcher of cream.
\indpar
  The rather soft, smooth cereals, such as farina and cream of rice, are to be
measured in just the same way, but they need not be cooked overnight; only
put on in a double boil\-er in the morning for an hour.  Margaret's\linebreak mother was
very particular to have all cereals cooked a long time, because they are
difficult to digest if they are only partly cooked, even though they look
and taste as though they were done.
\begin{RecipeTitle}
Corn-meal Mush\label{cornmeal_mush}
\end{RecipeTitle}
\ingredient  1 quart of boiling water.
\ingredient  1 teaspoon of salt.
\ingredient  4 tablespoons of corn-meal.
\instruction
  Be sure the water is boiling very hard when you are ready; then put in the
salt, and pour slowly from your hand the corn-meal, stirring all the time
till there is not one lump.  Boil this half an hour, and serve with cream.
Some like a handful of nice plump raisins stirred in, too.  It is better to
use yellow corn-meal in winter and white in summer.
\begin{RecipeTitle}
Fried Corn-meal Mush\label{fried_cornmeal_mush}
\end{RecipeTitle}
\instruction
  Make the corn-meal mush the day before you need it, and when it has cooked
half an hour put it in a bread-tin and smooth it over; stand away overnight
to harden.  In the morning turn it out and slice it in pieces half an inch
thick.  Put two tablespoons of lard or nice drippings in the frying-pan,
and make it very hot.  Dip each piece of mush into a pan of flour, and
shake off all except a coating of this.  Put the pieces, a few at a time,
into the hot fat, and cook till they are brown; have ready a heavy brown
paper on a flat dish in the oven, and as you take out the mush lay it on
this, so that the paper will absorb the grease.  When all are cooked put
the pieces on a hot platter, and have a pitcher of maple syrup ready to
send to the table with them.
\indpar
  Another way to cook corn-meal mush is to have a kettle of hot fat ready,
and after flouring the pieces drop them into the fat and cook like
doughnuts.  The pieces have to be rather smaller to cook in this way
than in the other.
\begin{RecipeTitle}
Boiled Rice\label{boiled_rice}
\end{RecipeTitle}
\ingredient  1 cup of rice.
\ingredient  2 cups of boiling water.
\ingredient  1 teaspoonful of salt.
\instruction
  Pick the rice over, taking out all the bits of brown husk; fill the
outside of the double boiler with hot water, and put in the rice, salt,
and water, and cook forty minutes, but do not stir it.  Then take off the
cover from the boiler, and very gently, without stirring, turn over the
rice with a fork; put the dish in the oven without the cover, and let it
stand and dry for ten minutes.  Then turn it from the boiler into a hot
dish, and cover.  Have cream to eat on it.  If any rice is left over from
breakfast, use it the next morning as---
\begin{RecipeTitle}
Fried Rice\label{fried_rice} 
\end{RecipeTitle}
\instruction
  Press it into a pan, just as you did the mush, and let it stand
overnight; the next morning slice it, dip it in flour, and fry, either
in the pan or in the deep fat in the kettle, just as you did the mush.
\begin{RecipeTitle}
Farina Croquettes\label{farina_croquettes}
\end{RecipeTitle}
\instruction
  When farina has been left from breakfast, take it while still warm
and beat into a pint of it the beaten yolks of two eggs.  Let it then
get cold, and at luncheon-time make it into round balls; dip each one
first into the beaten yolk of an egg mixed with a tablespoonful of cold
water, and then into smooth, sifted bread-crumbs; have ready a kettle
of very hot fat, and drop in three at a time, or, if you have a wire
basket, put three in this and sink into the fat till they are brown.
Serve in a pyramid, on a napkin, and pass scraped maple sugar with them.
\indpar
  Margaret's mother used to have no cereal at breakfast sometimes, and
have these croquettes as a last course instead, and every one liked them
very much.
\begin{RecipeTitle}
Rice Croquettes\label{rice_croquettes}
\end{RecipeTitle}
\ingredient  1 cup of milk.
\ingredient  Yolk of one egg.
\ingredient  \OneQuarter cup of rice.
\ingredient  1 large tablespoonful of powdered sugar.
\ingredient  Small half-teaspoonful of salt.
\ingredient  \OneHalf cup of raisins and currants, mixed.
\ingredient  \OneHalf teaspoonful of vanilla.
\instruction
  Wash the rice and put in a double boiler with the milk, salt and
sugar and cook till very thick; beat the yolks of the eggs and stir into
the rice, and beat till smooth.  Sprinkle the washed raisins and currants
with flour, and roll them in it and mix these in, and last the vanilla.
Turn out on a platter, and let all get very cold.  Then make into
pyramids, dip in the yolk of an egg mixed with a tablespoonful of water,
and then into sifted bread-crumbs, and fry in a deep kettle of boiling
fat, using a wire basket.  As you take these from the fat, put them on
paper in the oven with the door open.  When all are done, put them on a
hot platter and sift powdered sugar over them, and put a bit of red
jelly on top of each.  This is a nice dessert for luncheon.  All white
cereals may be made into croquettes; if they are for breakfast, do not
sweeten them, but for luncheon use the rule just given, with or without
raisins and currants.
\begin{RecipeTitle}
Hominy\label{hominy}
\end{RecipeTitle}
\instruction
  Cook this just as you did the rice, drying it in the oven; serve one
morning plain, as cereal, with cream, and then next morning fried, with
maple syrup, after the rest of the meal.  Fried hominy is always nice
to put around a dish of fried chicken or roast game, and it looks
especially well if, instead of being sliced, it is cut out into fancy
shapes with a cooky-cutter.
\indpar
  After Margaret had learned to cook all kinds of cereals, she went on
to the next thing in her cook-book.
\bigskip
\begin{FoodTypeTitle}
EGGS\label{EGGS}
\end{FoodTypeTitle}
\vspace*{-3.0ex}
\begin{RecipeTitle}
Soft Boiled\label{soft_boiled_eggs}
\end{RecipeTitle}
\instruction
  Put six eggs in a baking-dish and cover them with boiling water; put
a cover on and let them stand where they will keep hot, but not cook,
for ten minutes, or, if the family likes them well done, twelve minutes.
They will be perfectly cooked,\ \  but not tough,\ \  soft and creamy all the
way through.
\instruction
  Another way to cook them is this:
\instruction
  Put the eggs in a kettle of cold water on the stove, and the moment
the water boils take them up, and they will be just done.  An easy way
to take them up all at once is to put them in a wire basket, and sink
this under the water.  A good way to serve boiled eggs is to crumple
up a fresh napkin in a deep dish, which has been made very hot, and lay
the eggs in the folds of the napkin; this prevents their breaking, and
keeps them warm.\pagebreak[2]
\begin{RecipeTitle}
Poached Eggs\label{poached_eggs}
\end{RecipeTitle}
\instruction
  Take a pan which is not more than three inches deep, and put in as
many muffin-rings as you wish to cook eggs.  Pour in boiling water till
the rings are half covered, and scatter half a teaspoonful of salt in
the water.  Let it boil up once, and then draw the pan to the edge of
the stove, where the water will not boil again.  Take a cup, break one
egg in it, and gently slide this into a ring, and so on till all are
full.  While they are cooking, take some toast and cut it into round
pieces with the biscuit cutter; wet these a very little with boiling
water, and butter them.  When the eggs have cooked twelve minutes,
take a cake-turner and slip it under one egg with its ring, and lift
the two together on to a piece of toast, and then take off the ring;
and so on with all the eggs.  Shake a very little salt and pepper
over the dish, and put parsley around the edge.  Sometimes a little
chopped parsley is nice to put over the eggs, too.
\begin{RecipeTitle}
Poached Eggs with Potted Ham\label{poached_eggs_with_potted_ham}
\end{RecipeTitle}
\instruction
  Make the rounds of toast and poach the eggs as before.  Make a white
sauce in this way: melt a tablespoonful of butter, and when it bubbles
put in a tablespoonful of flour; shake well, and add a cup of hot milk
and a small half-teaspoonful of salt; cook till smooth.  Moisten each
round of toast with a very little boiling water, and spread with some
of the potted ham which comes in little tin cans; lay a poached egg on
each round, and put a teaspoonful of white sauce on each egg.
\instruction
  If you have no potted ham in the house, but have plain boiled ham,
put this through the meat-chopper till you have half a cupful, put in
a heaping teaspoonful of the sauce, a saltspoonful of dry mustard,
and a pinch of red pepper, and it will do just as well.
\begin{RecipeTitle}
Scrambled Eggs\label{scrambled_eggs}
\end{RecipeTitle}
\ingredient  4 eggs.
\ingredient  2 tablespoonfuls of milk.
\ingredient  \OneHalf teaspoonful of salt.
\instruction
  Put the eggs in a bowl and stir till they are well mixed; add the
milk and salt.  Make the frying-pan very hot, and put a tablespoonful
of butter in it; when it melts, shake it well from side to side, till
all the bottom of the pan is covered.  Put in the eggs and stir them,
scraping them off the bottom of the pan until they begin to get a
little firm; then draw the pan to the edge of the stove, and scrape up
from the bottom all the time till the whole looks alike, creamy and
firm, but not hard.  Put them in a hot, covered dish.
\begin{RecipeTitle}
Scrambled Eggs with Parsley\label{scrambled_eggs_with_parsley}
\end{RecipeTitle}
\instruction
  Chop enough parsley to make a teaspoonful, and mince half as much
onion.  Put the onion in the butter when you heat the pan, and cook
the eggs in it; when you are nearly ready to take the eggs off the fire,
put in the parsley.
\indpar\ \indpar
  After Margaret had learned to make these perfectly, she began to mix
other things with the eggs.
\begin{RecipeTitle}
Scrambled Eggs with Tomato\label{scrambled_eggs_with_tomato}
\end{RecipeTitle}
\instruction
  When Margaret found a cupful of tomato in the refrigerator, she would
take that, add a half-teaspoonful of salt, two shakes of pepper, and a
teaspoonful of chopped parsley, and simmer it all on the fire for
five minutes; then she would cook half a teaspoonful of minced onion
in the butter in the hot frying-pan as before, and turn in the eggs,
and when they were beginning to grow firm, put in the tomato.  In
summer-time she often cut up two fresh tomatoes and stewed them down
to a cupful, instead of using the canned.
\begin{RecipeTitle}
Scrambled Eggs with Chicken\label{scrambled_eggs_with_chicken}
\end{RecipeTitle}
\instruction
  Chop fine a cup of cold chicken, or any light-colored meat, and heat
it with a tablespoonful of water, a half-teaspoonful of salt, two shakes
of pepper, and a teaspoonful of chopped parsley.  Cook a half-teaspoonful
of minced onion in the butter you put in the hot frying-pan, and turn
in the eggs, and when they set mix in the chicken.
\instruction
  Sometimes Margaret used both the tomato filling and the chicken in the
eggs, when she wanted to make a large dish.
\begin{RecipeTitle}
Creamed Eggs\label{creamed_eggs}
\end{RecipeTitle}
\instruction
  Cook six eggs twenty minutes, and while they are on the fire make a
cup of white sauce, as before: one tablespoonful of butter, melted,
one of flour, one cup of hot milk, a little salt; cook till smooth.
Peel the eggs and cut the whites into pieces as large as the tip of your
finger, and put the yolks through the potato-ricer.  Mix the eggs white
with the sauce, and put in a hot dish, with the yellow yolks over the
top.  Or, put the whites on pieces of toast, which you have dipped in
part of the white sauce, and put the yolks on top, and serve on a small
platter.
\instruction
  Another nice way to cream eggs is this:  Cook them till hard, and cut
them all up into bits.  Make the white sauce, and into it stir the
beaten yolk of one egg, just after taking it from the fire.  Mix the
eggs with this, and put in a hot dish or on toast.  You can sprinkle
grated cheese over this sometimes, for a change.
\begin{RecipeTitle}
Creamed Eggs in Baking-Dishes\label{creamed_eggs_in_baking_dishes}
\end{RecipeTitle}
\instruction
  Cut six hard-boiled eggs up into bits, mix with a cup of white sauce,
and put in small baking-dishes which you have buttered.  Cover over
with fine, sifted bread-crumbs, and dot with bits of butter, about
four to each dish, and brown in the oven.  Stick a bit of parsley in
the top of each, and put each dish on a plate, to serve.
\begin{RecipeTitle}
Birds' Nests\label{birds_nests}
\end{RecipeTitle}
\instruction
  Sometimes when she wanted something very pretty for breakfast,
Margaret used this rule:
\instruction
  Open six eggs, putting the whites together in one large bowl, and
the yolks in six cups on the kitchen table.  Beat the whites till they
are stiff, putting in half a teaspoonful of salt just at the last.
Divide the whites, putting them into six patty-pans, or small
baking-dishes.  Make a little hole or nest in the middle of each,
and slip one yolk carefully from the cup into the place.  Sprinkle a
little salt and pepper over them, and put a bit of butter on top,
and put the dishes into a pan and set in the oven till the egg-whites
are a little brown.
\begin{RecipeTitle}
Omelette\label{omelette}
\end{RecipeTitle}
\instruction
  Making an omelette seems rather a difficult thing for a little girl,
but Margaret made hers in a very easy way.  Her rule said:
\instruction
  Break four eggs separately.  Beat the whites till they are stiff,
and then wash and wipe dry the egg-beater, and beat the yolks till they
foam, and then put in half a teaspoonful of salt.  Pour the yolks over
the whites, and mix gently with a large spoon.  Have a cake-griddle
hot, with a piece of butter melted on it and spread over the whole
surface; pour the eggs on and let them cook for a moment.  The take
a cake-turner and slip under an edge, and look to see if the middle is
getting brown, because the color comes there first.  When it is a nice
even color, slip the turner well under, and turn the omelette half
over, covering one part with the other, and then slip the whole off
on a hot platter.  Bridget had to show Margaret how to manage this
the first time, but after that she could do it alone.
\begin{RecipeTitle}
Spanish Omelette\label{spanish_omelette}
\end{RecipeTitle}
\ingredient  1 cup of cooked tomato.
\ingredient  1 green pepper.
\ingredient  1 slice of onion.
\ingredient  1 teaspoonful of chopped parsley.
\ingredient  1 teaspoonful salt.
\ingredient  3 shakes of pepper.
\instruction
  Cut the green pepper in half and take out all the seeds; mix with
the tomato, and cook all together with the seasoning for five minutes.
Make an omelette by the last rule while the tomato is cooking, and
when it is done, just before you fold it over, put in the tomato.
\begin{RecipeTitle}
Omelette with Mushrooms\label{omelette_with_mushrooms}
\end{RecipeTitle}
\instruction
  Take a can of mushrooms and slice half of them into thin pieces.
Make a cup of very rich white sauce, using cream instead of milk,
and cook the mushrooms in it for one minute.  Make the omelette as
before, and spread with the sauce when you turn it over.
\begin{RecipeTitle}
Omelette with Mushrooms and Olives\label{omelette_with_mushrooms_and_olives}
\end{RecipeTitle}
\instruction
  This was a very delicious dish, and Margaret only made it for
company.  She prepared the mushrooms just as in the rule above,
and added twelve olives, cut into small pieces, and spread the
omelette with the whole when she turned it.
\begin{RecipeTitle}
Eggs Baked in Little Dishes\label{eggs_baked_in_little_dishes}
\end{RecipeTitle}
\instruction
  Margaret's mother had some pretty little dishes with handles,
brown on the outside and white inside.  These Margaret buttered,
and put one egg in each, sprinkling with salt, pepper, and butter,
with a little parsley.  She put the dishes in the oven till the eggs
were firm, and served them in the small dishes, one on each plate.
\begin{RecipeTitle}
Eggs with Cheese\label{eggs_with_cheese}
\end{RecipeTitle}
\ingredient  6 eggs.
\ingredient  2 heaping tablespoonfuls Parmesan cheese.
\ingredient  \OneHalf teaspoonful salt.
\ingredient  Pinch of red pepper.
\instruction
  Beat the eggs without separating till light and foamy, and then
add the cheese, salt, and pepper.  Put a tablespoonful of butter in
the frying-pan, and when it is hot put in the eggs, and stir till
smooth and firm.  Serve on small pieces of buttered toast.
\instruction
  Parmesan cheese is very nice to use in cooking; it comes in bottles,
all ready grated to use.
\begin{RecipeTitle}
Eggs with Bacon\label{eggs_with_bacon}
\end{RecipeTitle}
\instruction
  Take some bacon and put in a hot frying-pan, and cook till it crisps.
Then lift it out on a hot dish and put in the oven.  Break six eggs
in separate cups, and slide them carefully into the fat left in the
pan, and let them cook till they are rather firm and the bottom is
brown.  Then take a cake-turner and take them out carefully, and put
in the middle of the dish, and arrange the bacon all around, with
parsley on the edge.
\begin{RecipeTitle}
Ham and Eggs, Moulded\label{moulded_ham_and_eggs}
\end{RecipeTitle}
\instruction
  Take small, deep tins, such as are used for timbales, and butter
them.  Make one cup of white sauce; take a cup of cold boiled ham
which has been put through the meat-chopper, and mix with a
tablespoonful of white sauce and one egg, slightly beaten.  Press
this like a lining into the tins, and then gently drop a raw egg
in the centre of each.  Stand them in a pan of boiling water in the
oven till the eggs are firm,---about ten minutes,---and turn
out on a round platter.  Put around them the rest of the white sauce.
You can stand the little moulds on circles of toast if you wish.
This rule was given Margaret by her Pretty Aunt, who got it at
cooking-school; it sounded harder than it really was, and after
trying it once Margaret often used it.
\bigskip
\begin{center}
\hstroke
\end{center}\pagebreak[2]
\begin{FoodTypeTitle}
FISH\label{FISH}
\end{FoodTypeTitle}
\indpar
  One day some small, cunning little fish came home from market,
and Margaret felt sure they must be meant for her to cook.  They
were called smelts, and, on looking, she found a rule for cooking
them, just as she had expected.
\begin{RecipeTitle}
Fried Smelts\label{fried_smelts}
\end{RecipeTitle}
\instruction
  Put a deep kettle on the fire, with two cups of lard in it, to
get it very hot.  Wipe each smelt inside and out with a clean wet
cloth, and then with a dry one.  Have a saucer of flour mixed
with a teaspoonful of salt, and another saucer of milk.  Put the
tail of each smelt through its gills---that is, the opening near
its mouth.  Then roll the smelts first in milk and then in flour,
and shake off any lumps.  Throw a bit of bread into the fat in the
kettle, and see if it turns brown quickly; it does if the fat is
hot enough, but if not you must wait.  Put four smelts in the wire
basket, and stand it in the fat, so that the fish are entirely
covered, for only half a minute, or till you can count thirty.
As you take them out of the kettle, lay them on heavy brown paper
on a pan in the oven, to drain and keep hot, and leave the door
open till all are done.  Lay a folded napkin on a long, narrow
platter, and arrange the fishes in two rows, with slices of lemon
and parsley on the sides.
\begin{RecipeTitle}
Fish-balls\label{fish_balls}
\end{RecipeTitle}
\instruction
  One morning there was quite a good deal of cold mashed potato
in the ice-box, so Margaret decided to have fish-balls for
breakfast.  Her rule said:  Take a box of prepared codfish and
put it in a colander and pour a quart of boiling water through
it, stirring it as you do so.  Let it drain while you heat two
cups of mashed potato in a double boiler, with half a cup of hot
milk, beating and stirring till it is smooth.  Squeeze the water
from the codfish and mix with the potato.  Beat one egg without
separating it, and put this in, too, with a very little pepper,
and beat it all well.  Turn it out on a floured board, and make
into small balls, rolling each one in flour as it is done, and
brushing off most of the flour afterward.  Have ready a kettle of
hot lard, just as for smelts, and drop in three or four of the
balls at one time, and cook till light brown.  Lift them out on
a paper in the oven, and let them keep hot while you cook the
rest.  Serve with parsley on a hot platter.
\begin{RecipeTitle}
Creamed Codfish\label{creamed_codfish}
\end{RecipeTitle}
\instruction
  Pour boiling water over a package of prepared codfish in the
colander and drain it.  Heat a frying-pan, and, while you are
waiting, beat the yolk of an egg.  Squeeze the water from the
fish.  Put one tablespoonful of butter in a hot pan, and when it
bubbles put in two tablespoonfuls of flour, and stir and rub till
all is smooth.  Pour in slowly a pint of hot milk, and mix well,
rubbing in the flour and butter till there is not a single lump.
Then stir in the fish with a little pepper, and when it boils
put in the egg.  Stir it all up once, and it is done.  Put in a
hot covered dish, or on slices of buttered toast.
\begin{RecipeTitle}
Salt Mackerel\label{salt_mackerel}
\end{RecipeTitle}
\instruction
  This\, was\, a\, dish Margaret's grandmother liked so much that they
had\, it every little while, even though it was old-fashioned.
\instruction
  Put the mackerel into a large pan of cold water with the skin
up, and soak it all one afternoon and night, changing the water
four times.  In the morning put it in a pan on the fire with enough
water to cover it, and drop in a slice of onion, minced fine, a
teaspoonful of vinegar, and a sprig of parsley.  Simmer it twenty
minutes,---that is, let it just bubble slowly,---and while it
is cooking make a cup of white sauce as before: one tablespoonful
of butter, melted, one tablespoonful of flour, one cup of hot milk,
a little salt.  Cook till smooth.  Take up the fish and pour off
all the water; place it on a hot platter and pour the sauce over it.
\bigskip
\begin{center}
\hstroke
\end{center}
\begin{FoodTypeTitle}
MEATS\label{MEATS}
\end{FoodTypeTitle}
\indpar
  When it came to cooking meat for breakfast, Margaret thought she
had better take first what looked easiest, so she chose---
\begin{RecipeTitle}
Corned Beef Hash\label{corned_beef_hash}
\end{RecipeTitle}
\ingredient  1 pint of chopped corned beef.
\ingredient  1 pint of cold boiled potatoes.
\ingredient  1 cup of clear soup, or one cup of cold water.
\ingredient  1 tablespoonful of butter.
\ingredient  1 teaspoonful of finely minced onion.
\ingredient  \OneHalf teaspoonful of salt.
\ingredient  3 shakes of pepper.
\instruction
  Mix all together.  Have a hot frying-pan, and in it put a
tablespoonful of butter or nice fat, and when it bubbles shake it
all around the pan.  Put in the hash and cook it till dry, stirring
it often and scraping it from the bottom of the pan.  When none of
the soup or water runs out when you lift a spoonful, and when it
seems steaming hot, you can send it to the table in a hot dish,
with parsley around it.  Or you can let it cook without stirring
till there is a nice brown crust on the bottom, when you can
double it over as you would an omelette.  Or you can make a pyramid
of the hash in the middle of a round platter, and put poached eggs
in a circle around it.
\instruction
  Many people like one small cold boiled beet cut up fine in corned
beef hash, and sometimes for a change you can put this in before
you put it in the frying-pan.
\begin{RecipeTitle}
Broiled Bacon\label{broiled_bacon} 
\end{RecipeTitle}
\instruction
  Margaret's mother believed there was only one very nice way to
cook bacon.  It was like this:  Slice the bacon very, very thin,
and cut off the rind.  Put the slices close together in a wire
broiler, and lay this over a shallow pan in a very hot oven for
about three minutes.  If it is brown on top, then you can turn the
broiler over, but if not, wait a moment longer.  When both sides
are toasted, lay it on a hot platter and put sprigs of parsley
around.  This is much nicer than bacon cooked in the frying-pan
or over coals, for it is neither greasy nor smoky, but pink
and light brown, and crisp and delicious, and good for sick people
and little children and everybody.
\begin{RecipeTitle}
Broiled Chops\label{broiled_chops}
\end{RecipeTitle}
\instruction
  Wipe off the chops with a clean wet cloth and trim off the edges;
if very fat cut rather close to the meat.  Rub the wire broiler
with some of the fat, so that the chops will not stick.  Lay in
the chops and put over a clear, red fire without flame, and toast
one side first and then the other; do this till they are brown.
Lay on a hot platter, and dust both sides with salt and a tiny bit
of pepper.  Put bits of lemon and parsley around, and send to the
table hot.
\begin{RecipeTitle}
Panned Chops\label{panned_chops}
\end{RecipeTitle}
\instruction  If the fire is not clear so that you cannot broil the chops, you
must pan them.  Take a frying-pan and make it very hot indeed; then
lay in the chops, which you have wiped and trimmed, and cook one
side very quickly, and then the other, and after that let them cook
more slowly.  When they are done,---you can tell by picking open
a little place in one with a fork and looking on the inside,---
put them on a platter as before, with pepper and salt.  If they
are at all greasy,  put on brown paper in the oven first, to drain,
leaving the door of the oven open.  Be careful not to let them
get cold.
\begin{RecipeTitle}
Liver and Bacon\label{liver_and_bacon}
\end{RecipeTitle}
\instruction
  Buy half a pound of calf's liver and half a pound of bacon.  Cut
the liver in thin slices and pour boiling water over it, and then
wipe each slice dry.  Slice the bacon very thin and cut off the
rind; put this in a hot frying-pan and cook very quickly, turning
it once or twice.  Just as soon as it is brown take it out and lay
it on brown paper in the oven in a pan.  Take a saucer of flour and
mix in it a teaspoonful of salt and a very little pepper; dip the
slices of liver in this, one at a time, and shake them free of lumps.
Lay them in the hot fat of the bacon in the pan and fry till brown.
Have a hot platter ready, and lay the slices of liver in a nice
row on it, and then put one slice of bacon on each slice of liver.
Put parsley all around, and sometimes use slices of lemon, too,
for a change.
\begin{RecipeTitle}
Liver and Bacon on Skewers\label{liver_and_bacon_on_skewers}
\end{RecipeTitle}
\instruction
  Get from the butcher half a dozen small wooden skewers, and
prepare the liver and bacon as you did for frying, scalding,
dipping the liver in flour, and taking the rind off the bacon.
Make three slices of toast, cut into strips, and put in the oven
to keep hot.  Cut up both liver and bacon into pieces the size
of a fifty-cent piece and put them on the skewers, first one of
the liver and then one of the bacon, and so on, about six of each.
Put these in the hot frying-pan and turn them over till they are
brown.  Then lay one skewer on each strip of toast, and put lemon
and parsley around.  You can also put large oysters on the skewers
with pieces of bacon, and cook in the same way.
\begin{RecipeTitle}
Broiled Steak\label{broiled_steak}
\end{RecipeTitle}
\instruction
  See that the fire is clear and red, without flames.  Trim off
most of the fat from the steak, and rub the wires of the broiler
with it and heat it over the coals.  Then put in the meat and
turn over and over as it cooks, and be careful not to let it take
fire.  When brown, put it on a hot platter, dust over with salt
and a very little pepper, and dot it with tiny lumps of butter.
Put parsley around.  Steak ought to be pink inside; not brown
and not red.  Put a fork in as you did with the chops, and twist
in a little, and you can see when it gets the right color.
\begin{RecipeTitle}
Steak with Bananas\label{steak_with_bananas}
\end{RecipeTitle}
\instruction
  Peel one banana and slice in round pieces, and while the steak
is cooking fry them in a little hot butter till they are brown.
After the meat is on the platter, lay these pieces over it,
arranging them prettily, and put the parsley around as before.
Bananas are very nice with steak.\pagebreak[4]
\begin{RecipeTitle}
Frizzled Dried Beef\label{frizzled_dried_beef}
\end{RecipeTitle}
\instruction
  Take half a pound of dried beef, shaved very thin.  Chop it fine
and pull out the strings.  Put a large tablespoonful of butter in
the frying-pan, and when it bubbles put in the meat.  Stir till it
begins to get brown, and then sprinkle in one tablespoonful of flour
and stir again, and then put in one cup of hot milk.  Shake in a
little pepper, but no salt.  As soon as it boils up once, it is
done, and you can put it in a hot covered dish.  If you like a
change, stir in sometimes two beaten eggs in the milk instead of
using it plain.
\begin{RecipeTitle}
Veal Cutlet\label{veal_cutlet}
\end{RecipeTitle}
\instruction
  Wipe off the meat with a clean wet cloth, and then with one that
is dry.  Dust it over with salt, pepper, and flour.  Put a tablespoonful
of nice dripping in a hot frying-pan, and let it heat till it smokes
a little.  Lay in the meat and cook till brown, turning it over twice
as it cooks.  Look in the inside and see if it is brown, for cutlet
must not be eaten red or pink inside.  Put in a hot oven and cover it
up while you make the gravy, by putting one tablespoonful of flour
into the hot fat in the pan, stirring it till it is brown.  Then
put in a cup of boiling water, half a teaspoonful of salt, and a
very little pepper; put this through the wire sieve, pressing it
with a spoon, and turn over the meat.  Put parsley around the cutlet,
and send hot to the table.
\bigskip
\begin{center}
\hstroke
\end{center}
\medskip
\indpar
  Margaret's father said he could not possibly manage without
potatoes for breakfast, so sometimes Margaret let Bridget cook
the cereal and meat, while she made something nice out of the
cold potatoes she found in the cupboard.
\begin{RecipeTitle}
Creamed Potatoes\label{creamed_potatoes}
\end{RecipeTitle}
\instruction
  Cut cold boiled potatoes into pieces as large as the end of your
finger; put them into a pan on the back of the stove with enough milk
to cover them, and let them stand till they have drunk up all the
milk; perhaps they will slowly cook a little as they do this, but
that will do no harm.  In another saucepan or in the frying-pan
put a tablespoonful of butter, and when it bubbles put in a
tablespoonful of flour, and stir till they melt together; then
put in two cups of hot milk, and stir till it is all smooth.  Put
in one teaspoonful of salt, and last the potatoes, but stir them
only once while they cook, for fear of breaking them.  Add one
teaspoonful of chopped parsley, and put them in a hot covered dish.
You can make another sort of potatoes when you have finished
creaming them in this way, by putting a layer of them in a deep
buttered baking-dish, with a layer of white sauce over the top,
and break-crumbs and bits of butter for a crust.  Brown well in a
hot oven.  When you do this, remember to make the sauce with three
cups of milk and two tablespoonfuls of flour and two of butter,
and then you will have enough for everything.
\begin{RecipeTitle}
Hashed Browned Potatoes\label{hashed_browned_potatoes}
\end{RecipeTitle}
\instruction
  Chop four cold potatoes fine, and add one teaspoonful of salt
and a very little pepper.  Put a tablespoonful of butter in the
frying-pan, and turn it so it runs all over;  when it bubbles
put in the potatoes, and smooth them evenly over the pan.  Cook
till they are brown and crusty on the bottom; then put in a
teaspoonful of chopped parsley, and fold over like an omelette.
\begin{RecipeTitle}
Saratoga Potatoes\label{saratoga_potatoes}
\end{RecipeTitle}
\instruction
  Wash and pare four potatoes, and rub them on the potato-slicer
till they are in thin pieces; put them in ice-water for fifteen
minutes.  Heat two cups of lard very hot, till when you drop in
a bit of bread it browns at once.  Wipe the potatoes dry and drop
in a handful.  Have a skimmer ready, and as soon as they brown
take them out and lay on brown paper in the oven, and put in
another handful.
\begin{RecipeTitle}
Potato Cakes\label{potato_cakes}
\end{RecipeTitle}
\instruction
  Take two cups of mashed potato, and mix well with the beaten
yolk of one egg, and make into small flat cakes;  dip each into
flour.  Heat two tablespoonfuls of nice dripping, and when
it is hot lay in the cakes and brown, turning each with the
cake-turner as it gets crusty on the bottom.\pagebreak[4]
\begin{RecipeTitle}
Fried Sweet Potatoes\label{fried_sweet_potatoes}
\end{RecipeTitle}
\instruction
  Take\, six\, cold\, boiled\, sweet-potatoes, slice them and lay in hot
dripping in the frying-pan till brown.  These are especially nice
with veal cutlets.
\bigskip
\begin{center}
\hstroke
\end{center}
\begin{RecipeTitle}
Toast\label{toast}
\end{RecipeTitle}
\instruction
  Toast is very difficult for grown people to make, because they
have made it wrong all their lives, but it is easy for little
girls to learn to make, because they can make it right from the
first.
\indpar
  Cut bread that is at least two days old into slices a quarter
of an inch thick.  If you are going to make only a slice or two,
take the toasting-fork, but if you want a plateful, take the wire
broiler.  Be sure the fire is red, without any flames.  Move the
slices of bread back and forth across the coals, but do not let
them brown; do both sides this way, and then brown first one and
then the other afterward.  Trim off the edges, butter a little
quickly, and send to the table hot.  Baker's bread makes the
best toast.
\begin{RecipeTitle}
Milk Toast\label{milk_toast}
\end{RecipeTitle}
\instruction
  Put one pint of milk on in a double boiler and let it heat.
Melt one tablespoonful of butter, and when it bubbles stir in
one small tablespoonful of corn-starch, and when these are
rubbed smooth, put in one-third of the milk.  Cook and stir
till even, without lumps, and then put in the rest of the milk
and stir well; add half a teaspoonful of salt, and put on the
back of the stove.  Make six slices of toast; put one slice in
the dish and put a spoonful of the white sauce over it, then
put in another and another spoonful, and so on till all are in,
and pour the sauce that is left over all.  If you want this
extra nice, do not take quite so much butter, and use a pint of
cream instead of the milk.
\begin{RecipeTitle}
Baking-powder Biscuit\label{baking_powder_biscuit}
\end{RecipeTitle}
\instruction
  Margaret's Other Aunt said little girls could never, never
make biscuit, but this little girl really did, by this rule:
\ingredient  1 pint sifted flour.
\ingredient  \OneHalf teaspoonful of salt.
\ingredient  4 teaspoonfuls of baking-powder.
\ingredient  \ThreeQuarters cup of milk.
\ingredient  1 tablespoonful of butter.
\instruction
  Put the salt and baking-powder in the flour and sift well,
and then rub the butter in with a spoon.  Little by little put
in the milk, mixing all the time, and then lift out the dough
on a floured board and roll it out lightly, just once, till
it is one inch thick.  Flour your hands and mould the little
balls as quickly as you can, and put them close together in a
shallow pan that has had a little flour shaken over the bottom,
and bake in a hot oven about twenty minutes, or till the
biscuits are brown.  If you handle the dough much, the biscuits
will be tough, so you must work fast.
\begin{RecipeTitle}
Grandmother's Corn Bread\label{grandmothers_cornbread}
\end{RecipeTitle}
\ingredient  1\OneHalf cups of milk.
\ingredient  1 cup sifted yellow corn-meal.
\ingredient  1 tablespoonful melted butter.
\ingredient  1 teaspoonful sugar.
\ingredient  1 teaspoonful baking-powder.
\ingredient  2 eggs.
\ingredient  \OneHalf teaspoonful of salt.
\instruction
  Scald the milk---that is, let it boil up just once---and
pour it over the corn-meal.  Let this cool while you
are separating and beating the eggs;  let these wait while
you mix the corn-meal,\, the butter,\, salt,\, baking-powder,\, and
sugar, and then the yolks; add the whites last, very lightly.
Bake in a buttered biscuit-tin in a hot oven for about half
an hour.
\smallskip\indpar
  Because grandmother's corn bread was a little old-fashioned,
Margaret's Other Aunt put in another recipe, which made a corn
bread quite like cake, and most delicious.
\begin{RecipeTitle}
Perfect Corn Bread\label{perfect_cornbread}
\end{RecipeTitle}
\ingredient  1 large cup of yellow corn-meal.
\ingredient  1 small cup of flour.
\ingredient  \OneHalf cup of sugar.
\ingredient  2 eggs.
\ingredient  2 teaspoonfuls of baking-powder.
\ingredient  3 tablespoonfuls of butter.
\ingredient  1 teaspoonful of salt.
\ingredient  Flour to a thin batter.
\instruction
  Mix the sugar and butter and rub to a cream; add the yolks
of the eggs, well beaten, and then half a cup of milk; then put
in the baking-powder mixed in the flour and the salt, and then
part of the corn-meal, and a little more milk; next fold in
the beaten whites of the eggs, and if it still is not like
``a thin batter,'' put in a little more milk.  Then bake in a
buttered biscuit-tin till brown, cut in squares and serve hot.
This is particularly good eaten with hot maple syrup.
\begin{RecipeTitle}
Popovers\label{popovers}
\end{RecipeTitle}
\instruction
  Put the muffin-tins or iron gem-pans in the oven to get
very hot, while you mix these popovers.
\ingredient  2 eggs.
\ingredient  2 cups of milk.
\ingredient  2 cups of flour.
\ingredient  1 small teaspoonful of salt.
\instruction
  Beat the eggs very lightly without separating them.  Pour the
milk in and beat again.  Sift the salt and flour together, pour
over the eggs and milk into it, and beat quickly with a spoon
till it is foamy.  Strain through a wire sieve, and take the hot
pans out of the oven and fill each one-half full; bake just
twenty-five minutes.
\begin{RecipeTitle}
Cooking-school Muffins\label{cooking_school_muffins}
\end{RecipeTitle}
\ingredient  2 cups sifted flour.
\ingredient  2 teaspoonfuls baking-powder.
\ingredient  \OneHalf teaspoonful of salt.
\ingredient  1 cup of milk.
\ingredient  2 eggs.
\ingredient  1 large teaspoonful of melted butter.
\instruction
  Mix the flour, salt, and baking-powder, and sift.  Beat the
yolks of the eggs, put in the butter with them and the milk,
then the flour, and last the stiff whites of the eggs.  Have
the muffin-tins hot, pour in the batter, and bake fifteen or
twenty minutes.  These must be eaten at once or they will fall.
\medskip
\indpar
  There was one little recipe in Margaret's book which she
thought must be meant for the smallest girl who ever tried
to cook, it was so easy.  But the little muffins were good
enough for grown people to like.  This was it:
\begin{RecipeTitle}
Barneys\label{barneys}
\end{RecipeTitle}
\ingredient  4 cups of whole wheat flour.
\ingredient  3 teaspoonfuls of baking-powder.
\ingredient  1 teaspoonful of salt.
\ingredient  Enough water to make it seem like cake batter.
\instruction
  Drop with a spoon into hot buttered muffin-pans, and bake
in a hot oven about fifteen minutes.
\instruction
  Bridget\, had to show\, Margaret\, what was meant by a ``cake batter,''
but after she had seen once just how thick that was, she could
always tell in a minute when she had put in water enough.
\begin{RecipeTitle}
Griddle-cakes\label{griddle_cakes}
\end{RecipeTitle}
\ingredient  2 eggs.
\ingredient  1 cup of milk.
\ingredient  1\OneHalf cups flour.
\ingredient  2 teaspoonfuls of baking-powder.
\ingredient  \OneHalf teaspoonful of salt.
\instruction  Put the eggs in a bowl without separating them, and beat
them with a spoon till light.  Put in the milk, then the flour
mixed with the salt, and last the baking-powder all alone.  Bake
on a hot, buttered griddle.  This seems a queer rule, but it makes
delicious cakes, especially if eaten with sugar and thick cream.\pagebreak[4]
\begin{RecipeTitle}
Flannel Cakes\label{flannel_cakes}
\end{RecipeTitle}
\ingredient  1 tablespoonful of butter.
\ingredient  1 tablespoonful of sugar.
\ingredient  2 eggs.
\ingredient  2 cupfuls of flour.
\ingredient  1 teaspoonful of baking-powder.
\ingredient  Milk enough to make a smooth, rather thin batter.
\instruction
  Rub the butter and sugar to a cream, add the eggs, beaten
together lightly, then the flour, in which you have mixed the
baking-powder, and then the milk.  It is easy to know when you
have the batter just right, for you can put a tiny bit on the
griddle and make a little cake; if it rises high and is thick,
put more milk in the batter; if it is too thin, it will run
about on the griddle, and you must add more flour; but it is
better not to thin it too much, but to add more milk if the
batter is too thick.
\begin{RecipeTitle}
Sweet Corn Griddle-cakes\label{sweet_corn_griddle_cakes}
\end{RecipeTitle}
\instruction
  These ought to be made of fresh sweet corn, but you can
make them in winter out of canned grated corn, or canned corn
rubbed through a colander.
\ingredient  1 quart grated corn.
\ingredient  1 cup of flour.
\ingredient  1 cup of milk.
\ingredient  1 tablespoonful melted butter.
\ingredient  4 eggs.
\ingredient  \OneHalf teaspoonful of salt.
\instruction
  Beat the eggs separately, and put the yolks into the corn;
then add the milk, then the flour, then the salt, and beat well.
Last of all, fold in the whites and bake on a hot griddle.
\begin{RecipeTitle}
Waffles\label{waffles}
\end{RecipeTitle}
\ingredient  2 cups of flour.
\ingredient  1 teaspoonful baking powder.
\ingredient  1\OneHalf cups of milk.
\ingredient  1 tablespoonful butter.
\ingredient  \OneHalf teaspoonful of salt.
\ingredient  3 eggs, beaten separately.
\instruction
  Mix the flour, baking-powder, and salt; put the beaten egg yolks
in the milk, and add the melted butter, the flour and last the
beaten whites of the eggs.  Make the waffle-iron very hot, and
grease it very thoroughly on both sides by tying a little rag
to a clean stick and dipping in melted butter.  Put in some
batter on one side, filling the iron about half-full, and close
the iron, putting this side down over the fire; when it has cooked
for about two minutes, turn the iron over without opening it,
and cook the other side.  When you think it is done, open it a
little and look to see if it is brown; if not, keep it over the
coals till it is.  Take out the waffle, cut in four pieces, and
pile on a plate in the oven, while you again grease the iron
and cook another.  Serve very hot and crisp, with maple syrup
or powdered sugar and thick cream.
\instruction
  Some people like honey on their waffles.  You might try all
these things in turn.
\bigskip
\begin{center}
\hstroke
\end{center}
\medskip
\indpar
  Last of all the things Margaret learned to make for breakfast
came coffee, and this she could make in two ways; sometimes she
made it this\, first way,\, and sometimes\, the other,\, which is called
French coffee.
\begin{RecipeTitle}
Coffee\label{coffee}
\end{RecipeTitle}
\instruction
  First be sure your coffee-pot is shining clean; look in the
spout and in all the cracks, and wipe them out carefully, for
you cannot make good coffee except in a perfectly clean pot.
Then\, get\, three\, heaping\, tablespoonfuls\, of ground coffee, and
one tablespoonful of cold water, and one tablespoonful of
white of egg.  Mix the egg with the coffee and water thoroughly,
and put in the pot.  Pour in one quart of boiling water, and
let it boil up once.  Then stir down the grounds which come
to the top, put in two tablespoonfuls of cold water, and let
it stand for a minute on the back of the stove, and then strain
it into the silver pot for the table.  This pot must be made
very hot, by filling it with boiling water and letting it
stand on the kitchen table while the coffee is boiling.  If
this rule makes coffee stronger than the family like it, take
less coffee, and if it is not strong enough, take more coffee.
\begin{RecipeTitle}
French Coffee\label{french_coffee}
\end{RecipeTitle}
\instruction
  Get one of the pots which are made so the coffee will drip
through; put three tablespoonfuls of very finely powdered
coffee in this, and pour in a quart of boiling water.  When
it is all dripped through, it is ready to put in the hot
silver pot.
\newpage
\thispagestyle{empty}
\ 
\newpage
\thispagestyle{empty}
\vspace*{30ex}
\begin{center}
{\large \bf PART II.}\\
\ \\
THE THINGS MARGARET MADE FOR LUNCHEON OR SUPPER\label{PART_II}
\end{center}
\newpage
\thispagestyle{empty}
\ 
\newpage
\thispagestyle{plain}
\indpar
   So many things in this part of Margaret's book call for white
sauce, or cream sauce, that the rule for that came first of all.
\begin{RecipeTitle}
White or Cream Sauce\label{white_or_cream_sauce}
\end{RecipeTitle}
\ingredient  1 tablespoonful of butter.
\ingredient  1 tablespoonful of flour.
\ingredient  1 cup hot milk or cream, one-third teaspoonful of salt.
\instruction
  Melt the butter, and when it bubbles put in the flour, shaking
the saucepan as you do so, and rub till smooth.  Put in the hot
milk, a little at a time, and stir and cook without boiling till
all is smooth and free from lumps.  Add the salt, and, if you
choose, a little pepper.
\instruction
  Cream sauce is made exactly as is white sauce, but cream is
used in place of milk.  What is called thick white sauce is made
by taking two tablespoonfuls of butter and two of flour, and
only one cup of milk.
\begin{RecipeTitle}
Creamed Oysters\label{creamed_oysters}
\end{RecipeTitle}
\ingredient  1 pint oysters.
\ingredient  1 large cup of cream sauce.
\instruction
  Make the sauce of cream if you have it, and if not use a very
heaping tablespoonful of butter in the white sauce.  Keep this hot.
\instruction
  Drain off the oyster-juice and wash the oysters by holding them
under the cold-water faucet.  Strain the juice and put the oysters
back in it, and put them on the fire and let them just simmer
till the edges of the oysters curl; then drain them from the
juice again and drop them in the sauce, and add a little more
salt (celery-salt is nice if you have it), and just a tiny bit
of cayenne pepper.  You can serve the oysters on squares of
buttered toast, or put them in a large dish, with sifted
bread-crumbs over the top and tiny bits of butter, and brown
in the oven.  Or you can put them in small dishes as they are,
and put a sprig of parsley in each dish.
\begin{RecipeTitle}
Panned Oysters\label{panned_oysters}
\end{RecipeTitle}
\instruction
  Take the oysters from their juice, strain it, wash the oysters,
and put them back in.  Put them in a saucepan with a little
salt,---about half a teaspoonful to a pint of oysters,---and
a little pepper, and a piece of butter as large as the end
of your thumb.  Let them simmer till the edges curl, just as
before, and put them on squares of hot buttered toast.
\begin{RecipeTitle}
Scalloped Oysters\label{scalloped_oysters}
\end{RecipeTitle}
\ingredient  1 pint of oysters.
\ingredient  12 large crackers, or 1 cup of bread-crumbs.
\ingredient  \OneHalf cup of milk.
\ingredient  The strained oyster-juice.
\instruction
  Butter a deep baking-dish.  Roll the crackers, or make the
bread-crumbs of even size; some people like one better than
the other, and you can try both ways.  Put a layer of crumbs
in the dish, then a layer of oysters, washed, then a sprinkling
of salt and pepper and a few bits of butter.  Then another
layer of crumbs, oysters, and seasoning, till the dish is full,
with crumbs on the top.  Mix the milk and oyster-juice and
pour slowly over.  Then cover the top with bits of butter,
and bake in the oven till brown---about half an hour.
\instruction
  You can put these oysters into small dishes, just as you
did the creamed oysters, or into large scallop-shells, and
bake them only ten or fifteen minutes.  In serving, put a
small sprig of parsley into each.\pagebreak[4]
\begin{RecipeTitle}
Pigs in Blankets\label{oyster_pigs_in_blankets}
\end{RecipeTitle}
\instruction
  These were great fun to make, and Margaret often begged
to get them ready for company.
\ingredient  15 large oysters.
\ingredient  15 very thin slices of bacon.
\instruction
  Sprinkle each oyster with a very little salt and pepper.
Trim the rind from the bacon and wrap each oyster in one
slice, pinning this ``blanket'' tightly on the back with a
tiny Japanese wooden toothpick.  Have ready a hot frying-pan,
and lay in five oysters, and cook till the bacon is brown
and the edges of the oysters curl, turning each over once.
Put these on a hot plate in the oven with the door open,
and cook five more, and so on.  Put them on a long, narrow
platter, with slices of lemon and sprigs of parsley around.
Or you can put each one on a strip of toast which you have
dipped in the gravy in the pan; this is the better way.
This dish must be eaten very hot, or it will not be good.
\begin{RecipeTitle}
Creamed Fish\label{creamed_fish}
\end{RecipeTitle}
\ingredient  2 cups of cold fish.
\ingredient  1 cup of white sauce.
\instruction
  Pick any cold fish left from dinner into even bits, taking
out all the bones and skin, and mix with the hot white sauce.
$\!$Stir until smooth, and add a small half-teaspoonful of
chopped parsley.
\instruction
  You can put this in a buttered baking-dish and cover the
top with crumbs and bits of butter, and brown in the oven,
or you can put it in small dishes and brown also, or you can
serve it just as is, in little dishes.
\begin{RecipeTitle}
Creamed Lobster\label{creamed_lobster}
\end{RecipeTitle}
\ingredient  1 lobster, or the meat from 1 can.
\ingredient  1 large cup of white or cream sauce.
\instruction
  Take the lobster out of the shell and clean it;  Bridget
will have to show you how the first time.  Or, if you are
using canned lobster, pour away all the juice and pick out
the bits of shell, and find the black string which is apt
to be there, and throw it away.  Cut the meat in pieces
as large as the end of your finger, and heat it in the
sauce till it steams.  Put in a small half-teaspoonful of
salt, a pinch of cayenne, and a squeeze of lemon.  Do not
put this in a large dish, but in small ones, buttered well,
and serve at once.  Stand a little claw up in each dish.
\begin{RecipeTitle}
Creamed Salmon\label{creamed_salmon}
\end{RecipeTitle}
\ingredient  1 can salmon.
\ingredient  1 cup of white sauce.
\instruction
  Prepare this dish exactly as you did the plain creamed
white fish.  Take it out of the can, remove all the juice,
bones, and fat, and put in the white sauce, and cook a
moment till smooth.  Add a small half-teaspoonful of salt,
a little pepper, and a squeeze of lemon, and put in a
baking-dish and brown, or serve as it is, in small dishes.
\begin{RecipeTitle}
Scalloped Lobster or Salmon\label{scalloped_lobster_or_salmon}
\end{RecipeTitle}
\ingredient  1 can of fish, or 1 pint.
\ingredient  1 large cup of cracker or bread crumbs.
\ingredient  1 large cup of white sauce.
\instruction
  Prepare this dish almost as you did the scalloped
oysters.  Take out all the bones and skin and juice
from the fish; butter a baking-dish, put in a layer of
fish, then salt and pepper, then a layer of crumbs and
butter, and a layer of white sauce, then fish, seasoning,
crumbs and butter again, and have the crumbs on top.
Dot over with butter and brown in the oven, or serve in
small dishes.
\begin{RecipeTitle}
Crab Meat in Shells\label{crab_meat_in_shells}
\end{RecipeTitle}
\instruction
  You can buy very nice, fresh crab meat in tins, and the
shells also.  A very delicious dish is made by mixing a cup
of rich cream sauce with the crab meat, seasoning it well
with salt and pepper and putting in the crab-shells;
cover with crumbs, dot with butter, and brown in the oven.
This is a nice thing to have for a company luncheon.
\begin{RecipeTitle}
Creamed Chicken or Turkey\label{creamed_chicken_or_turkey}
\end{RecipeTitle}
\ingredient  2 cups of cold chicken.
\ingredient  1 large cup of white or creamed sauce.
\ingredient  \OneHalf teaspoonful of chopped parsley.
\ingredient  Salt and pepper.
\instruction
  Pick the chicken or turkey off the bones and cut into
small bits before you measure it.  Heat it in the sauce
till very hot, but do not let it boil, and add the
seasoning,---about half a teaspoonful of salt, and a
tiny bit of cayenne, or as much celery-salt in the place
of the common kind.  Put in a large buttered dish and
serve, or in small dishes, either with crumbs on top or not.
\instruction
  A nice addition to this dish is half a green pepper, the
seeds taken out, chopped very fine indeed, and mixed with
the white meat; the contrast of colors is pretty and the
taste improved.
\begin{RecipeTitle}
Scalloped Eggs\label{scalloped_eggs}
\end{RecipeTitle}
\ingredient  6 hard-boiled eggs.
\ingredient  1 cup cream or white sauce.
\ingredient  1 cup fine bread-crumbs.
\ingredient  Salt and pepper.
\instruction
  Cook the eggs twenty minutes, and while they are
cooking make the white sauce, and butter one large or
six small dishes.  Peel the eggs and cut them into bits
as large as the end of your finger.  Put a layer of
bread-crumbs on the bottom of the dish, then a layer of
egg, then a sprinkling of salt, pepper, and bits of
butter, then a layer of white sauce.  Then more crumbs,
egg, and seasoning, till the dish is full, with crumbs
on top.  Put bits of butter over all and brown in the oven.
\begin{RecipeTitle}
Eggs in Double Cream\label{double_cream_with_eggs}
\end{RecipeTitle}
\instruction
  This is a rule Margaret's Pretty Aunt got in Paris,
and it is a very nice one.  Have half a pint of very
thick cream---the kind you use to whip; the French
call this double cream.  Cook six eggs hard and cut them
into bits.  Butter a baking-dish, or small dishes, and
put in a layer of egg, then a layer of cream, then a
sprinkling of salt, and one of paprika, which is sweet
red pepper.  Put one thin layer of fine, sifted crumbs
on top with butter, and brown in the oven.  Or you can
put the eggs and cream together and heat them, and
serve on thin pieces of buttered toast, with one extra
egg put through the ricer over the whole.
\begin{RecipeTitle}
Creamed Eggs on Toast\label{creamed_eggs_on_toast}
\end{RecipeTitle}
\instruction
  Make small pieces of nice toast and dip each one in
white sauce.  Boil hard four eggs, and cut in even
slices and cover the toast, and then spread the rest
of the white sauce over all in a thin layer.
\begin{RecipeTitle}
Devilled Eggs\label{devilled_eggs}
\end{RecipeTitle}
\ingredient  6 eggs.
\ingredient  2 saltspoonfuls of dry mustard.
\ingredient  \OneHalf teaspoonful of salt.
\ingredient  1 saltspoonful of cayenne pepper.
\ingredient  1 teaspoonful of olive-oil or cream.
\ingredient  1 large tablespoonful of chopped ham.
\ingredient  \OneHalf teaspoonful of vinegar.
\instruction
  Boil the eggs hard for twenty minutes, and put them in
cold water at once to get perfectly cold so they will not
turn dark.  Then peel, cut in halves and take out the yolks.
Put these in a bowl, and rub in the seasoning, but you
can leave out the ham if you like.  With a small teaspoon,
put the mixture back into the eggs and smooth them over
with a knife.
\instruction
  If you do not serve these eggs with cold meat it is best
to lay them on lettuce when you send them to the table.
\begin{RecipeTitle}
Eggs in Beds\label{eggs_in_bed}
\end{RecipeTitle}
\instruction
  Chop a cup of nice cold meat, and season with a little
salt, pepper and chopped parsley.  Add enough stock or hot
water just to wet it, and cook till rather dry.  Put this in
buttered baking-dishes, filling each half-full, and on top
of each gently slip from a cup one egg.  Sprinkle over with
salt and pepper, and put in the oven till firm.\pagebreak[4]
\begin{RecipeTitle}
Shepherd's Pie\label{shepherds_pie}
\end{RecipeTitle}
\instruction
  This was a dish Margaret used to make on wash-day and
house-cleaning day, and such times when everybody was
busy and no one wanted to stop and go to market to buy
anything for luncheon.
\ingredient  1 cup of chopped meat.
\ingredient  1 cup of boiling water.
\ingredient  1 teaspoonful of chopped parsley.
\ingredient  \OneHalf teaspoonful of salt.
\ingredient  1 teaspoonful of lemon juice, or \OneHalf teaspoon\-ful
Worcestershire sauce.
\ingredient  Butter the size of a hickory-nut.
\ingredient  2 cups hot mashed potato.
\instruction
  If the potato is cold, put half a cup of hot milk in it,
beat it up well, and stand it on the back of the stove.
Then mix all the other things with the meat, and put it
in the frying-pan and let it cook till it seems rather
dry.  Butter a baking-dish, and cover the sides and
bottom with a layer of potato an inch thick.  Put the
meat in the centre and cover it over with potato and
smooth it.  Put bits of butter all over the top, and
brown it in the oven.  Serve with this a dish of chow-chow,
or one of small cucumber pickles.
\begin{RecipeTitle}
Chicken Hash\label{chicken_hash}
\end{RecipeTitle}
\ingredient  1 cup of cold chicken, cut in small, even pieces.
\ingredient  \OneHalf cup chicken stock, or hot water.
\ingredient  1 teaspoonful chopped parsley.
\ingredient  \OneHalf teaspoonful salt.
\ingredient  A pinch of pepper.
\ingredient  Butter the size of a hickory-nut.
\instruction
  Put the chicken stock,---which is the water the
chicken was cooked in, or chicken broth,---or, if
there is none, the hot water, into the frying-pan, and
mix in the chicken and seasoning, and cook and stir till
it is rather dry.  Serve as it is, or on squares of
buttered toast.  You can make any cold meat into hash
this way, having it different every time.  Sometimes you
can put in the chopped green pepper, as before, or a
slice of chopped onion, or a cup of hot, seasoned peas;
or, leave out half the soup or water, and put in a cup
of stewed tomato.
\begin{RecipeTitle}
Broiled Sardines\label{broiled_sardines}
\end{RecipeTitle}
\instruction
  These little fish are really not broiled at all, but
that is the name of the nice and easy dish.  Take a box
of large sardines and drain off all the oil, and lay them
on heavy brown paper while you make four slices of toast.
Trim off the edges and cut them into strips, laying them
in a row on a hot platter.  Put the sardines into the
oven and make them very hot, and lay one on each strip
of toast and sprinkle them with lemon juice, and put
sliced lemon and sprigs of parsley all around.
\begin{RecipeTitle}
Cheese Fondu\label{cheese_fondu}
\end{RecipeTitle}
\instruction
  This was a recipe the Pretty Aunt put in Margaret's
book out of the one she had made at cooking school.
\ingredient  1 cup fresh bread-crumbs.
\ingredient  2 cups grated cheese.
\ingredient  1 cup of milk.
\ingredient  1 bit of soda as large as a pea.
\ingredient  \OneHalf teaspoonful of salt.
\ingredient  1 pinch of red pepper.
\ingredient  1 teaspoonful of butter.
\ingredient  2 eggs.
\instruction
  Put the butter in a saucepan to heat while you beat
the eggs light without separating them; let these stand
while you stir everything else into the pan, beginning
with the milk; cook this five minutes, stirring all the
time, and then put in the eggs and cook three minutes
more.  Put six large crackers on a hot platter and pour
the whole over them, and send at once to the table to
be eaten very hot.  Sometimes Margaret made three or
four slices of toast before she began the fondu, and
used those in place of the crackers, and the dish was
just as nice.
\begin{RecipeTitle}
Easy Welsh Rarebit\label{easy_welsh_rarebit}
\end{RecipeTitle}
\ingredient  2 cups of rich cheese, grated.
\ingredient  Yolks of two eggs.
\ingredient  \OneHalf cup of milk.
\ingredient  \OneHalf teaspoonful of salt.
\ingredient  Saltspoonful of cayenne.
\instruction
  Make three nice slices of toast, cut off the crusts,
and cut each piece in two.  Butter these, and very quickly
dip each one in boiling water, being careful not to soak
them.  Put these on a hot platter in the oven.  Put the
milk in a saucepan over the fire, being careful not to
have one that is too hot, only moderate, and when it
boils up put in the cheese and stir without stopping,
until the cheese all melts and it looks smooth.  Then put
in the beaten yolks of the eggs and the seasoning, and
pour at once over the toast and serve very hot.  Many
people like a saltspoonful of dry mustard mixed in with
the pepper.  You can also serve this rarebit on toasted
and buttered crackers.
\begin{RecipeTitle}
Scalloped Cheese\label{scalloped_cheese}
\end{RecipeTitle}
\ingredient  6 slices of bread.
\ingredient  \ThreeQuarters of a pound of cheese.
\ingredient  2 eggs.
\ingredient  1 tablespoonful of butter.
\ingredient  1 cup of cream.
\ingredient  \OneHalf teaspoonful of salt.
\ingredient  \OneHalf teaspoonful of dry mustard.
\ingredient  \OneQuarter teaspoonful of paprika.
\instruction
  Butter the bread and cut it into strips, and line the
bottom and sides of a baking-dish with it.  Then beat the
eggs very light without separating them, and mix everything
with them; put in the dish and bake half an hour, and
serve at once.
\begin{RecipeTitle}
Veal Loaf\label{veal_loaf}
\end{RecipeTitle}
\ingredient  1\OneHalf pounds of veal and
\ingredient  2 strips of salt pork, chopped together.
\ingredient  \OneHalf cup of bread-crumbs.
\ingredient  1 beaten egg.
\ingredient  \OneHalf teaspoonful of grated nutmeg.
\ingredient  \OneHalf teaspoonful of black pepper.
\ingredient  1\OneHalf teaspoonfuls of salt.
\instruction
  Bake three hours.
\instruction
  Have the butcher chop the meat all together for you;
then put everything together in a dish and stir in the egg,
beaten without separating, and mix very well.  Press it
into a bread-pan and put in the oven for three hours by
the clock.
\instruction
  Every half-hour pour over it a tablespoonful hot water
and butter mixed; you can put a tablespoonful of butter
into a cup of water, and keep it on the back of the stove
ready all the time; after the meat has baked two hours,
put in a piece of heavy brown paper over the top, and
keep it there till it is done, or it may get too brown.
This is to slice cold; it is very nice for a picnic.
\begin{RecipeTitle}
Pressed Chicken\label{pressed_chicken}
\end{RecipeTitle}
\instruction
  This was one of the things Margaret liked to make for
Sunday night supper.  Have a good-sized chicken cut up,
and wipe each piece with a clean, damp cloth.  Put them
in a kettle or deep saucepan and cover with cold water,
and cook very slowly and gently, covered, till the meat
falls off the bones.  When it begins to grow tender,
put in a half teaspoonful of salt.  Take it out, and cut
it up in nice, even pieces, and put all the bones back
into the kettle, and let them cook till there is only
about a pint and a half of broth.  Add a little more
salt, and a sprinkling of pepper, and strain this through
a jelly bag.  Mix it with the chicken, and put them both
into a bread tin, and when cold put on ice over night.
After it has stood for an hour, put a weight on it, to
make it firm.  Slice with a very sharp knife, and put on
a platter with parsley all around.  This is a nice
luncheon dish for a summer day, as well as a supper dish.
\medskip
\instruction
  When you have bits of cold meat which you cannot slice,
and yet which you wish to serve in some nice way, make
this rule, which sounds difficult, but is really very easy:
\begin{RecipeTitle}
Meat Souffl\'{e}\label{meat_souffle}
\end{RecipeTitle}
\ingredient  1 cup of white sauce.
\ingredient  1 cup of chopped meat.
\ingredient  2 eggs.
\ingredient  Teaspoonful of chopped parsley.
\ingredient  Half a teaspoonful minced onion.
\instruction
  Put the parsley and onion in the meat, and mix with
the white sauce.  Beat the yolks of the eggs and stir in,
and cook one minute, and then cool.  Beat the whites of the
eggs and fold in, and bake half an hour, or a little more,
in a deep, buttered baking-dish.  You must serve this
immediately, or it will fall.
\begin{RecipeTitle}
Cold Meats\label{cold_meats}
\end{RecipeTitle}
\instruction
  Of course,\, like\, other\, people,\, Margaret's mother often had
cold meat for luncheon or supper, and one of the things her
cook-book told her was how to make it look nice when it
came on the table.
\instruction
  Always trim off all bits of skin and ragged pieces from
the meat, and remove the cold fat, except on ham, and then
you must trim it to a rather narrow edge.  If you have a
rather small dish for a large family, put slices of hard
boiled eggs around the edge, or make devilled eggs, and put
those around in halves.  Sometimes you can cut lettuce in
very narrow ribbons by holding several leaves in your hand
at once, folding them lengthwise, and using a pair of
scissors.  Sometimes a dozen pimolas may be sliced across
and put about the meat, especially if it is cold chicken
or turkey.  Always use parsley with meat, cold or hot.
Saratoga potatoes make a good border for lamb or roast beef,
and cold peas mixed with mayonnaise are always delicious
with either chicken or lamb.  If only the dish looks pretty,
it is almost certain to taste well.
\begin{RecipeTitle}
Sliced Meat with Gravy\label{sliced_meat_with_gravy}
\end{RecipeTitle}
  When there are a few slices left from a roast, put them
in a frying-pan with some of the gravy left also, and heat;
serve with parsley around.
\instruction
  If there is not gravy, take a little boiling water, add
a little salt, pepper, a half-teaspoon\-ful of minced
onion, and as much chopped parsley.  Lay in the meat in the
frying-pan, cover, and let it simmer, turning occasionally.
A few drops of Kitchen Bouquet will improve this; it is a
brown sauce which comes in small bottles.
\medskip
\indpar
  Some of the things Margaret made for breakfast she made
for lunch or supper, too, such as frizzled beef, and
scalloped eggs and omelettes.  She had some vegetables
besides, such as---
\begin{RecipeTitle}
Baked Tomatoes\label{baked_tomatoes}
\end{RecipeTitle}
\ingredient  6 large tomatoes.
\ingredient  1 cup bread-crumbs.
\ingredient  \OneHalf teaspoonful of salt.
\ingredient  1 tablespoonful of butter.
\ingredient  1 slice of onion.
\instruction
  Put the butter in the frying-pan, and when it bubbles
put in the bread-crumbs, the salt and onion, with a dusting
of pepper, and stir till the crumbs are a little brown and
the onion is all cooked; then take out the onion and throw
it away.  Wipe the tomatoes with a clean wet cloth, and cut
out the stem and a round hole or little well in the middle;
fill this with the crumbs, piling them up well on top; put
them in a baking-dish and stand them in a hot oven; mix a
cup of hot water with a tablespoonful of butter, and every
little while take out the baking-dish and wet the tomatoes
on top.  Cook them about half an hour, or till the skins
get wrinkled all over.  Serve them in the dish they are
cooked in, if you like, or put each one on a small plate,
pour some of the juice in the baking-dish over it, and
stick a sprig of parsley in the top.
\begin{RecipeTitle}
Stuffed Potatoes\label{stuffed_potatoes}
\end{RecipeTitle}
\instruction
  Wash six large potatoes and scrub them with a little brush,
till they are a nice clean light brown, and bake them for
half an hour in a hot oven; or, if they are quite large,
bake them till they are soft and puffy.  Cut off one end
from each and take out the inside with a teaspoon, holding
the potato in a towel as you do so, for it will be very hot.
Mix well this potato with two tablespoonfuls of rich milk
or cream, a half-teaspoonful of salt and just as much
butter, and put this back into the shells.  Stand the
potatoes side by side in a pan close together, the open
ends up, till they are browned.
\bigskip
\begin{center}
\hstroke
\end{center}
\begin{FoodTypeTitle}
SALADS\label{SALADS}
\end{FoodTypeTitle}
\indpar
  The Other Aunt said Margaret could never, never make
salads, but her mother said they were the easiest thing
of all to learn, so she did put them in just the same;
she bought a tin of olive oil from the Italian grocery,
because it was better and cheaper than bottled oil, and
she gave Margaret one important direction, ``When you
make salads, always have everything very cold,'' and
after that the rules were easy to follow, and the salads
were as nice as could be.
\begin{RecipeTitle}
French Dressing\label{french_dressing}
\end{RecipeTitle}
\ingredient  3 tablespoonfuls of oil.
\ingredient  \OneHalf teaspoonful lemon juice or vinegar.
\ingredient  \OneHalf teaspoonful of salt.
\ingredient  3 shakes of pepper.
\instruction
  Stir together till all is well mixed.
\instruction
  Many people prefer this dressing without pepper and
with a saltspoonful of sugar in its place; you can
try it both ways.
\begin{RecipeTitle}
Tomato and Lettuce Salad\label{tomato_and_lettuce_salad}
\end{RecipeTitle}
\instruction
  Peel four tomatoes; you can do this most easily by
pouring boiling water over them and skinning them when
they wrinkle, but you must drain off all the water
afterward, and let them get firm in the ice-box; wash
the lettuce and gently pat it dry with a clean cloth;
slice the tomatoes thin, pour off the juice, and arrange
four slices on each plate of lettuce, or mix them
together in the large bowl, and pour the dressing over.
\begin{RecipeTitle}
Egg Salad\label{egg_salad}
\end{RecipeTitle}
\instruction
  Cut up six hard-boiled eggs into quarters, lay them on
lettuce, and pour the dressing over.  Or pass a dish of
them with cold meat.
\begin{RecipeTitle}
Fish Salad\label{fish_salad}
\end{RecipeTitle}
\instruction
  Pick up cold fish and pour the dressing over it, and
put two sliced hard-boiled eggs around it; a few tips of
celery, nice white ones, are pretty around the whole.
\begin{RecipeTitle}
Cauliflower Salad\label{cauliflower_salad}
\end{RecipeTitle}
\instruction
  Take cold boiled cauliflower and pick it up into nice
pieces; pour the dressing over, and put on the ice till
you need it.\pagebreak[4]
\begin{RecipeTitle}
String Bean Salad\label{string_bean_salad}
\end{RecipeTitle}
\instruction
  Take cold string beans, either the green ones or the
yellow, pour the dressing over, put on ice, and serve on
lettuce.  Any cold vegetables can be used besides these,
especially asparagus, while lettuce alone is best of all.
\begin{RecipeTitle}
Pineapple Salad\label{pineapple_salad}
\end{RecipeTitle}
\instruction
  Put large bits of picked-up pineapple on white
lettuce, and pour the dressing over.
\begin{RecipeTitle}
Orange or Grapefruit Salad\label{orange_or_grapefruit_salad}
\end{RecipeTitle}
\instruction
  Peel three oranges or one grapefruit, and scrape off
all the white lining of the skin.  Divide it into sections,
or ``quarters,'' and with the scissors cut off the thin edge;
turn down the transparent sides and cut these off, too,
scraping the pulp carefully, so as not to waste it.
Take out all the seeds; lay the pieces on lettuce, and
pour the dressing over.  White grapes, cut in halves,
with the seeds taken out, are nice mixed with this, and
pineapple, grapes, and oranges, with a little banana,
are delicious.
\begin{RecipeTitle}
Mayonnaise\label{mayonnaise}
\end{RecipeTitle}
\ingredient  Yolk of 1 egg.
\ingredient  \OneHalf cup of olive-oil.
\ingredient  1 tablespoonful of lemon juice or vinegar.
\ingredient  \OneHalf teaspoonful of salt.
\ingredient  Pinch of red pepper.
\instruction
  Put the yolk of the egg into a very cold bowl; it is
better to put the bowl, the egg, the oil, and the beater
all on the ice a half-hour before you need them, for then
the mayonnaise comes quicker.  With a Dover egg-beater beat
till the yolk is very light indeed; then have some one else
begin to put in the oil, one drop at a time, till the
mayonnaise becomes so thick it is difficult to turn the
beater; then put in a drop or two of lemon or vinegar,
and this will thin it so you can use the oil again; keep
on doing this till you have nearly a cup of the dressing;
if you need more oil than the rule calls for, use it, and
toward the last add it two or three drops at a time.  When
you have enough, and it is stiff enough, put in the pepper
and salt and it is done.  Never use mustard except with
lobster, as this will spoil the taste.  Some salads,
especially fruit and vegetable, need very thick mayonnaise,
and then it is better to make it with lemon juice, while
a fish salad, or one to use with meats, may be thinner,
and then the vinegar will do; the lemon juice makes it
thick.  Always taste it before using it, to see if it is
just right, and, if not, put in more salt, or whatever
it needs.  You will soon learn.  Most people think
mayonnaise is very difficult to make, but, really, it
is as easy as baking potatoes, after you have once
learned how.  Every salad given before is just as nice
with mayonnaise as with French dressing, and you can try
each one both ways; then there are these, which are better
with mayonnaise.
\begin{RecipeTitle}
Chicken Salad\label{chicken_salad}
\end{RecipeTitle}
\ingredient  1 cup of chicken cut in large bits.
\ingredient  \OneHalf cup of celery, cut up and then dried.
\ingredient  2 hard-boiled eggs, cut into good-sized pieces.
\ingredient  6 olives, stoned and cut up.
\ingredient  \OneHalf cup mayonnaise.
\instruction
  Mix all very lightly together, as stirring will make the
salad mussy; put on lettuce.
\begin{RecipeTitle}
Lobster Salad\label{lobster_salad}
\end{RecipeTitle}
\ingredient  1 cup of lobster, cut in large bits.
\ingredient  2 hard-boiled eggs, cut in pieces.
\ingredient  \OneHalf teaspoonful of dry mustard, stirred in.
\ingredient  \OneHalf cup of mayonnaise.
\instruction
  Mix and put on lettuce.
\begin{RecipeTitle}
Celery Salad\label{celery_salad}
\end{RecipeTitle}
\ingredient  2 heads of celery.
\ingredient  3 hard-boiled eggs (or else
  1 cup of English walnuts).
\ingredient  \OneHalf cup very stiff mayonnaise.
\instruction
  Wash, wipe, and cut the celery into pieces as large as the
first joint of your little finger, and then rub it in a clean
towel till it is as dry as can be.  Cut up the eggs, sprinkle
all with salt, and add the mayonnaise and lay on lettuce.
Or mix the celery and the walnuts and mayonnaise; either
salad is nice.
\begin{RecipeTitle}
Celery and Apple Salad\label{celery_and_apple_salad}
\end{RecipeTitle}
\ingredient  2 sweet apples.
\ingredient  1 head of celery.
\ingredient  \OneHalf cup of English walnuts, broken up.
\ingredient  \OneHalf cup mayonnaise.
\instruction
  Peel the apples and cut into very small bits; chop the celery
and press in a towel; chop or break up the walnuts, but save
two halves for each person besides the half-cupful you put in
the salad.  Mix all together, lay on white hearts of lettuce
on plates, and then put the walnuts on top, two on each plate.
\begin{RecipeTitle}
Cabbage Salad\label{cabbage_salad}
\end{RecipeTitle}
\ingredient  \OneHalf a small cabbage.
\ingredient  1 cup very stiff mayonnaise.
\ingredient  1 teaspoonful celery-seed.
\instruction
  Cut the cabbage in four pieces and cut out the hard core;
slice the rest very fine on the cutter\, you use for\, Saratoga
potatoes;\, mix with the mayonnaise and put in the salad-dish;
sprinkle over with celery-seed, when you wish it to be very
nice, but it will do without this last touch.
\begin{RecipeTitle}
Cabbage Salad in Green Peppers\label{cabbage_in_green_peppers_salad}
\end{RecipeTitle}
\instruction
  Wipe green peppers and cut off the small end of each.
Take out the seed and the stem; fill each pepper with the
cabbage salad, letting it stand out at the top; put each
one on a plate on a leaf of lettuce.\pagebreak[4]
\begin{RecipeTitle}
Stuffed Tomato Salad\label{stuffed_tomato_salad}
\end{RecipeTitle}
\ingredient  1 cup of cut-up celery.
\ingredient  \OneHalf cup of English walnuts.
\ingredient  6 small, round tomatoes.
\ingredient  \OneHalf cup of mayonnaise.
\instruction
  Peel the tomatoes and scoop out as much of the inside as
you can, after cutting a round hole in the stem end; make a
salad with the celery, the cut-up walnuts, and the mayonnaise,
and fill the tomatoes, letting it stand up well on top.
Serve on plates, each one on a leaf of lettuce.
\begin{RecipeTitle}
Potato Salad\label{potato_salad}
\end{RecipeTitle}
\ingredient  3 cold boiled potatoes.
\ingredient  3 hard-boiled eggs.
\ingredient  \OneHalf cup English walnuts.
\ingredient  12 olives.
\instruction
  Break up the walnuts, saving a dozen halves unbroken.
Cut the potatoes and eggs into bits of even size, as large
as the tip of your finger; stone the olives and cut them up,
too; mix them together in a bowl, but do not stir them much,
or you will break the potatoes; sprinkle well with French
dressing, and put on the ice; when it is lunch or supper
time, mix quickly, only once, with stiff mayonnaise, and put
on lettuce; this is a delicious salad to have with cold meats.
\smallskip
\begin{center}
\hstroke
\end{center}
\medskip
\indpar
  Margaret's mother liked to have gingerbread or cookies for
lunch often, so those things came next in the cook-book.
\begin{RecipeTitle}
Gingerbread\label{gingerbread}
\end{RecipeTitle}
\ingredient  1 cup molasses.
\ingredient  1 egg.
\ingredient  1 teaspoonful of soda.
\ingredient  1 teaspoonful of ginger.
\ingredient  1 tablespoonful melted butter.
\ingredient  \OneHalf cup of milk.
\ingredient  2 cups of flour.
\instruction
  Beat the eggs without separating, but very light; put the
soda into the molasses, put them in the milk, with the ginger
and butter, then one cup of flour, measure in a medium-sized
cup and only level, then the egg, and last the rest of the
flour.  Bake in a buttered biscuit-tin.  For a change,
sometimes add a teaspoonful of cloves and cinnamon, mixed,
to this, and a cup of chopped almonds.  Or, when the
gingerbread is ready for the oven drop over halves of almonds.
\begin{RecipeTitle}
Soft Gingerbread, to Be Eaten Hot\label{soft_gingerbread}
\end{RecipeTitle}
\ingredient  1 cup of molasses.
\ingredient  \OneHalf cup boiling water.
\ingredient  \OneQuarter cup melted butter.
\ingredient  1\OneHalf cups flour.
\ingredient  \ThreeQuarters teaspoonful soda.
\ingredient  1 teaspoonful ginger.
\ingredient  \OneHalf teaspoonful salt.
\instruction
  Put the soda in the molasses and beat it well in a
good-sized bowl, then put in the melted butter, ginger,
salt, and flour, and beat again, and add last the water,
very hot indeed.  Have a buttered tin ready, and put it at
once in the oven; when half-baked, it is well to put a
piece of paper over it, as all gingerbread burns easily.
\instruction
  You can add cloves and cinnamon to this rule, and
sometimes you can make it and serve it hot as a pudding,
with a sauce of sugar and water, thickened and flavored.\pagebreak[4]
\begin{RecipeTitle}
Ginger Cookies\label{ginger_cookies}
\end{RecipeTitle}
\ingredient  \OneHalf cup butter.
\ingredient  1 cup molasses.
\ingredient  \OneHalf cup brown sugar.
\ingredient  1 teaspoonful ginger.
\ingredient  1 tablespoonful mixed cinnamon and cloves.
\ingredient  1 teaspoonful soda, dissolved in a tablespoonful of water.
\ingredient  Flour enough to make it so stiff you cannot stir it with
a spoon.
\instruction
  Melt the molasses and butter together on the stove, and
then take the saucepan off and add the rest of the things
in the recipe, and turn the dough out on a floured board
and roll it very thin, and cut in circles with a biscuit-cutter.
Put a little flour on the bottom of four shallow pans, lift
the cookies with the cake-turner and lay them in, and put
them in the oven.  They will bake very quickly, so you
must watch them.  When you want these to be extra nice,
put a teaspoonful of mixed cinnamon and cloves in them
and sprinkle the tops with sugar.\pagebreak[4]
\begin{RecipeTitle}
Grandmother's Sugar Cookies\label{grandmothers_sugar_cookies}
\end{RecipeTitle}
\ingredient  1 cup of butter.
\ingredient  2 cups of sugar.
\ingredient  2 eggs.
\ingredient  1 cup of milk.
\ingredient  2 teaspoonfuls of baking-powder.
\ingredient  \OneHalf teaspoonful of vanilla.
\ingredient  Flour enough to roll out easily.
\instruction
  Rub the butter and sugar to a cream; put in the milk,
then the eggs beaten together lightly, then two cups of
flour, into which you have sifted the baking-powder;$\!$
then the vanilla.  Take a bit of this and put it on the
floured board and see if it ``rolls out easily,'' and,
if it does not, but is soft and sticky, put in a handful
more of flour.  These cookies must not be any stiffer
than you can help, or they will not be good, so try not
to use any more flour than you must.
\bigskip
\indpar
  They usually had tea for luncheon or supper at
Margaret's house, but sometimes they had chocolate
instead, so these things came next in the cook-book.\pagebreak[4]
\begin{RecipeTitle}
Tea\label{tea}
\end{RecipeTitle}
\ingredient  \OneHalf teaspoonful of black tea for each person.
\ingredient  \OneHalf teaspoonful for the pot.
\ingredient  Boiling water.
\instruction
  Fill the kettle half-full of fresh, cold water,
because you cannot make good tea with water which
has been once heated.  When it is very hot, fill the
china teapot and put it where it will keep warm.
When the water boils very hard, empty out the teapot,
put in the tea, and put on the boiling water; do not
stand it on the stove, as too many people do, but send
it right to the table; it will be ready as soon as it is
time to pour it---about three minutes.  If you are
making tea for only one person, you will need a 
teaspoonful of tea, as you will see by the rule,
and two small cups of water will be enough.  If for more,
put in a half-teaspoonful for each person, and one cup of
water more.
\begin{RecipeTitle}
Iced Tea\label{iced_tea}
\end{RecipeTitle}
\instruction
  Put in a deep pitcher one teaspoonful of dry tea
for each person and two over.  Pour on a cup of boiling
water for each person, and cover the pitcher and let
it stand five minutes.  Then stir well, strain and pour
while still hot on large pieces of ice.  Put in a glass
pitcher and serve a bowl of cracked ice, a lemon,
sliced thin, and a bowl of powdered sugar with it.
Pour it into glasses instead of cups.
\begin{RecipeTitle}
Lemonade\label{lemonade}
\end{RecipeTitle}
\instruction
  Sometimes in the afternoon Margaret's aunts had tea
and cakes or wafers, and in summer they often had iced
tea or lemonade.  This is the way Margaret made lemonade:
\instruction
  Squeeze four lemons, and add ten teaspoonfuls of powdered
sugar; stir till it dissolves.  Add six glasses of water,
and strain.  Pour in a glass pitcher, and serve with
glasses filled half-full of cracked ice.  If you want
this very nice, put a little shredded pineapple with
the lemons.  Sometimes the juice of red raspberries
is liked, also.
\begin{RecipeTitle}
Lemonade with Grape-juice\label{lemonade_with_grape_juice}
\end{RecipeTitle}
\instruction
  Make the lemonade as before, and add half as much
bottled grape-juice, but do not put in any other fruit.
Serve with plenty of ice, in small glasses.\pagebreak[4]
\begin{RecipeTitle}
Chocolate\label{chocolate}
\end{RecipeTitle}
\ingredient  2 cups boiling water.
\ingredient  2 cups of boiling milk.
\ingredient  4 teaspoonfuls grated chocolate.
\ingredient  4 teaspoonfuls of sugar.
\instruction
  Scrape the chocolate off the bar, mix it with the
boiling water, and stir till it dissolves; mix the milk
and sugar in them and boil for one minute.  If you wish
to have it nicer, put a small teaspoonful of vanilla in
the chocolate-pot, and pour the hot chocolate in on it
when it is done, and have a little bowl of whipped cream
to send to the table with it, so that one spoonful may be
put on top of each cup.
\begin{RecipeTitle}
Cocoa\label{cocoa}
\end{RecipeTitle}
\ingredient  6 teaspoonfuls of cocoa.
\ingredient  1\OneHalf cups of boiling water.
\ingredient  1\OneHalf cups of boiling milk.
\ingredient  1 tablespoonful powdered sugar.
\instruction
  Put the cocoa into the boiling water and stir till it
dissolves, then put in the boiling milk and boil hard
two minutes, stirring it all the time; take from the fire
and put in the sugar and stir again.  If you like it quite
sweet, you may have to use more sugar.
\cleardoublepage
\thispagestyle{empty}
\vspace*{30ex}
\begin{center}
{\large \bf PART III.}\\
\ \\
THE THINGS MARGARET MADE\\
FOR DINNER\label{PART_III}
\end{center}
\newpage
\thispagestyle{empty}
\ 
\newpage
\thispagestyle{plain}
\vspace*{10ex}
\indpar
  At first, of course,  Margaret could not get dinner all alone;
indeed, it took her almost a year to learn how to cook everything
needed,---soup, vegetables, meat, salad, and dessert; but at
first she helped Bridget, and each day she cooked something.
Then she began to arrange very easy dinners when Bridget was out,
such as cream soup, beefsteak or veal cutlet, with potatoes and
one vegetable, and a plain lettuce salad, with a cold dessert
made in the morning.  The first time she really did every single
thing alone, Margaret's father gave her a dollar; he said it was
a ``tip'' for the best dinner he ever ate.
\medskip
\begin{FoodTypeTitle}
SOUPS\label{SOUPS}
\end{FoodTypeTitle}
\indpar
  The soups in the little cook-book began with those made of
milk and vegetables, because they were so easy to make, and,
when one was learned, all were made in the same way.  First
there was---\pagebreak[4]
\begin{RecipeTitle}
The General Rule\label{general_rule_cream_soup}
\end{RecipeTitle}
\ingredient  1 pint of fresh vegetable, cut up in small pieces, or one can.
\ingredient  1 pint of boiling water.
\ingredient  1 pint of hot milk.
\ingredient  1 tablespoonful of flour.
\ingredient  1 tablespoonful of butter.
\ingredient  \OneHalf teaspoonful of salt.
\ingredient  3 shakes of pepper.
\instruction
  After the vegetable is washed and cut in very small pieces,
put it in the pint of water and cook it for twenty minutes.
Or, if you use a canned vegetable, cook it ten minutes.
While it is cooking, make the rule for white sauce as before:
Melt one tablespoonful of butter, and when it bubbles put in
one tablespoonful of flour, with the salt and pepper; shake
well, and rub till smooth and thick with the hot milk.  Take
the vegetable from the fire and press it through the wire
sieve, letting the water go through, too; mix with the sauce
and strain again, and it is done.
\instruction
  Almost all soups are better for one very thin slice of
onion cooked with the vegetable.  When you want a cream soup
very nice indeed, whip a cup of cream and put in the hot
soup-tureen, and pour the soup in on it, beating it a little,
till it is all foamy.
\begin{RecipeTitle}
Cream of Corn\label{cream_of_corn}
\end{RecipeTitle}
\ingredient  1 pint of fresh grated corn, or one can.
\ingredient  1 pint of water.
\ingredient  1 pint of hot milk.
\ingredient  1 tablespoonful of flour.
\ingredient  1 tablespoonful of butter.
\ingredient  \OneHalf teaspoonful of salt.
\ingredient  3 shakes of pepper.
\ingredient  1 thin slice of onion.
\instruction
  Cook the corn with the water; make the white sauce with the
milk; strain the corn and water through the sieve, pressing
well, and add the milk and strain again.
\begin{RecipeTitle}
Cream of Green Peas\label{cream_of_green_peas}
\end{RecipeTitle}
\ingredient  1 pint of peas, or one can.
\ingredient  Milk, water, and seasoning, as before; mix by the general
rule.
\instruction
  In winter-time you can make a nice soup by taking dried
peas, soaking them overnight, and using them as you would
fresh.
\instruction
  All pea soup should have dropped in it just before serving
what are called croutons; that is, small, even cubes of bread
toasted to a nice brown in the oven, or put in a frying-pan
with a tiny bit of butter, and browned.
\begin{RecipeTitle}
Cream of Lima Beans\label{cream_of_lima_beans}
\end{RecipeTitle}
\ingredient  1 pint of fresh or canned beans, or those which have
been soaked.
\instruction
  Use milk, water, thickening, and seasoning as before.
Add a slice of onion, as these beans have little taste,
and beat the yolk of an egg and stir in quickly, after
you have taken the soup from the fire, just before you
strain it for the second time.
\begin{RecipeTitle}
Cream of Potato\label{cream_of_potato}
\end{RecipeTitle}
\instruction
  This is one of the best and most delicate soups.
\ingredient  5 freshly boiled potatoes.
\ingredient  1 slice of onion.
\ingredient  1 quart of hot milk.
\ingredient  1 small teaspoonful of salt.
\ingredient  1 teaspoonful chopped parsley.
\instruction
  This soup has no water in it, because that which has had
potatoes boiled in it is always spoiled for anything else
and must always be thrown away.  This is why you must take
a quart of milk instead of a pint.  There is no thickening
in the soup, because the potatoes will thicken it themselves.
Put the parsley in at the very last, after the soup is
in the tureen.
\instruction
  The yolk of an egg beaten and put in before the second
straining is nice sometimes in this soup, but not necessary.
\begin{RecipeTitle}
Cream of Almonds\label{cream_of_almonds}
\end{RecipeTitle}
\instruction
  This was what Margaret called a Dinner-party Soup, because
it seemed almost too good for every day, but, as her mother
explained, almonds cost no more than canned tomatoes or peas,
and the family can have the soup as well as guests, provided
one has plenty of cream.
\ingredient  1 cup of chopped almonds.
\ingredient  1 quart of thin cream.
\ingredient  Small half-teaspoonful of salt.
\instruction
  Get ten cents' worth of Jordan almonds, and put them in
boiling water for one minute; then pour off the water and
put on cold, till they are well chilled.  Turn this off,
and push the almonds out of their skins, one by one. If
they stick, it is because they were not in the hot water
long enough, and you must put them back into it, and then
into the cold.  Chop them while the cream heats in the
double boiler, and then put them in with the salt, and
simmer ten minutes and then strain.
\instruction
  This soup is especially delicious if whipped cream is
either mixed with it at the end, or served on top.
\instruction
  You can also make good almond soup by using the regular
rule; cooking the chopped nuts in a pint of water, adding
the thickened pint of milk and seasoning, and straining
twice.  Then, after it is in the tureen, you must put in
the egg-beater and whip well, to make it light.
\begin{RecipeTitle}
Cream of Spinach\label{cream_of_spinach}
\end{RecipeTitle}
\ingredient  1 pint cold cooked spinach.
\ingredient  1 quart of milk.
\instruction
  Heat the spinach, using a little of the quart of milk
with it, and press through the sieve; thicken the rest of
the milk, and the seasoning, and strain again.  It is better
to use cayenne pepper instead of black with spinach.\pagebreak[4]
\begin{RecipeTitle}
Cream of Tomato Soup, Called Tomato Bisque\label{cream_of_tomato_soup}
\end{RecipeTitle}
\ingredient  4 large tomatoes, cut up, or \OneHalf can, with \OneHalf cup of water.
\ingredient  2 slices of onion.
\ingredient  2 sprigs of parsley.
\ingredient  1 teaspoonful of sugar.
\ingredient  \OneHalf teaspoonful salt.
\ingredient  \OneQuarter teaspoonful soda.
\ingredient  1 quart of milk.
\ingredient  1 tablespoonful butter.
\ingredient  1 tablespoonful flour.
\instruction
  Cook the tomatoes with the onion, parsley, sugar, and
salt for twenty minutes.  Mix in the soda and stir well;
the soda prevents the milk from curdling.  Make the milk
and flour and butter into white sauce as usual; strain the
tomato, mix the two, and strain again.
\instruction
  Sometimes add a stalk of celery to the other seasoning
as it cooks.
\begin{RecipeTitle}
Cream of Clams\label{cream_of_clams}
\end{RecipeTitle}
\ingredient  1 dozen hard clams, or one bunch of soft ones.
\ingredient  1 quart of rich milk.
\ingredient  1 tablespoonful butter.
\ingredient  1 tablespoonful flour.
\ingredient  3 shakes of pepper.
\instruction
  Chop the clams and drain off the juice and add as much
water; cook till the scum rises, and skim this off.  Drop
in the clams and cook three minutes.  Heat the milk and
thicken as usual; put in the clams and juice, cook for
one minute, and strain.
\instruction
  Notice that there is no salt in this soup.  A cup of
cream, whipped, either put on top or stirred in, is
very nice.
\begin{RecipeTitle}
Oyster Soup\label{oyster_soup}
\end{RecipeTitle}
\ingredient  1 pint oysters.
\ingredient  \OneHalf pint water.
\ingredient  1 quart rich milk.
\ingredient  \OneHalf teaspoonful salt.
\instruction
  Drain off the oyster juice, add the water, boil it for
one minute, and skim it well.  Heat the milk and mix it
with this; drop in the oysters and cook one minute, or till
the edges begin to curl, and it is done.  This soup is
not thickened at all; but if you like you may add two
tablespoonfuls of finely powdered and sifted cracker-crumbs.
\begin{RecipeTitle}
Meat Soup or Bouillon Made from Extract\label{meat_soup_made_from_extract}
\end{RecipeTitle}
\indpar
  This Margaret made from beef extract, before she learned
to use the fresh beef.
\ingredient  2 teaspoonfuls of extract, or 2 capsules.
\ingredient  1 quart of boiling water.
\ingredient  \OneHalf an onion, sliced.
\ingredient  1 stalk of celery.
\ingredient  \OneHalf teaspoonful salt.
\ingredient  2 shakes of pepper.
\ingredient  2 sprays of parsley.
\instruction
  Simmer this for twenty minutes, strain, and pour over
six thin slices of lemon, one for each plate.  Serve with
hot crackers.
\begin{RecipeTitle}
Cream Bouillon\label{cream_bouillon}
\end{RecipeTitle}
  Make this same soup, and pour it over a half-pint of
thick cream, well whipped.  Do not put any lemon in it.
Serve with hot crackers.
\begin{RecipeTitle}
Meat Soups\label{meat_soups}
\end{RecipeTitle}
\instruction  You can make meat soup, or stock, out of almost any
kind of meat, cooked or raw, with bones or without.
Many cooks never buy fresh meat for it, and others think
they must always have it.  It is best to learn both ways.
\begin{RecipeTitle}
Plain Meat Soup\label{plain_meat_soup}
\end{RecipeTitle}
\ingredient  1 shin of beef.
\ingredient  5 quarts of water.
\ingredient  1 small tablespoonful of salt.
\ingredient  1 head celery, cut up.
\ingredient  1 onion.
\ingredient  1 carrot.
\ingredient  1 turnip.
\ingredient  1 sprig of parsley.
\ingredient  2 bay-leaves.
\ingredient  6 whole cloves.
\instruction  Wipe the meat and cut off all the bone.  Put the bone in a
clean kettle first, and then the meat on top, and pour in the
water; cover, and let this stand on the back of the stove an hour,
then draw it forward and let it cook.  This will bring scum on
the water in half an hour, and you must carefully pour in a
cup of cold water and skim off everything which rises to the top.
Cover the kettle tightly, and cook very slowly indeed for four
hours; then put in the cut up vegetables and cook one hour more,
always just simmering, not boiling hard.  Then it is done, and
you can put in the salt, and strain the soup first through a
heavy wire sieve, and then through a flannel bag, and set it
away to get cold, and you will have a strong, clear, delicious
stock, which you can put many things in to have variety.
\begin{RecipeTitle}
Clear Vegetable Soup\label{clear_vegetable_soup}
\end{RecipeTitle}
\instruction  Slice one carrot, turnip, and one potato, and cut them either
into small, even strips, or into tiny cubes, or take a vegetable
cutter and cut out fancy shapes.  Simmer them about twenty
minutes.  Meanwhile, take a pint of soup stock and a cup of
water and heat them.  Sprinkle a little salt over the vegetables
and drain them; put them in the soup-tureen and pour the hot
soup over.
\begin{RecipeTitle}
Split Pea Soup\label{split_pea_soup}
\end{RecipeTitle}
\ingredient  1 pint split peas.
\ingredient  1\OneHalf quarts of boiling water.
\ingredient  1 quart of soup stock.
\ingredient  1 small teaspoonful of salt.
\ingredient  3 shakes of pepper.
\instruction  Wash the peas in cold water and throw away those which float,
as they are bad.  Soak them overnight, and in the morning
pour away the water on them and cover them with a quart of
the boiling water in the rule, and cook an hour and a half.
Put in the rest of the water and the stock, and press the
whole through a sieve, and, after washing and wiping the
kettle, put the soup back to heat, adding the salt and pepper.
\begin{RecipeTitle}
Tomato Soup\label{tomato_soup}
\end{RecipeTitle}
\ingredient  1 can tomatoes, or 1 quart of fresh stewed ones.
\ingredient  1 pint of stock.  (You can use water instead in this soup,
if necessary.)
\ingredient  \OneQuarter teaspoonful soda.
\ingredient  1 tablespoonful of butter.
\ingredient  2 tablespoonfuls of flour.
\ingredient  1 teaspoonful of sugar.
\ingredient  1 small onion, cut up.
\ingredient  1 sprig of parsley.
\ingredient  1 bay-leaf.
\ingredient  1 small teaspoonful of salt.
\ingredient  3 shakes of pepper.
\instruction  Put the tomatoes into a saucepan with the parsley, onion,
bay-leaf, and stock, or water, and cook\, fifteen\, minutes,\,
and then\, strain through a sieve.  Wash the saucepan and
put the tomatoes back in it, and put on to boil again;
melt the butter, rub smooth with the flour, and put into
the soup while it boils, and stir till it is perfectly
smooth.  Then add the sugar, salt, and pepper and soda,
and strain into the hot tureen.  Serve croutons with
this soup.
\begin{RecipeTitle}
Soup Made with Cooked Meats\label{soup_made_with_cooked_meats}
\end{RecipeTitle}
\instruction  Put all the bones, bits of meat, and vegetables which
are in the refrigerator into one large kettle on the back
of the fire, and simmer all day in enough boiling water
to cover it all, adding more water as this cooks away.
Skim carefully from time to time.  If there are not many
vegetables to go in, put parsley and onion in their place.
At night strain through the sieve, then through the flannel,
and cool.
\instruction  This stock is never clear as is that made from fresh
meat, but it is almost as good for thick soups, such as
pea, or tomato.
\begin{RecipeTitle}
Chicken or Turkey Soup\label{chicken_or_turkey_soup}
\end{RecipeTitle}
\instruction  Break up the bones and cover with cold water; add a slice
of onion, a bay-leaf, and a sprig of parsley, and cook all
day, adding water when necessary, and skimming.  Cool,
take off the grease, heat again, and strain.  Serve with
small, even squares of chicken meat in it, or a little
cooked rice and salt.  Many people like a small pinch of
cinnamon in turkey soup.
\smallskip
\begin{center}
\hstroke
\end{center}
\smallskip
\begin{FoodTypeTitle}
VEGETABLES\label{VEGETABLES}
\end{FoodTypeTitle}
\begin{RecipeTitle}
Mashed Potatoes\label{mashed_potatoes}
\end{RecipeTitle}
\ingredient  6 large potatoes.
\ingredient  \OneHalf cup hot milk.
\ingredient  Butter the size of a hickory-nut.
\ingredient  3 teaspoonfuls salt.
\ingredient  3 shakes of pepper.
\instruction  Peel and boil the potatoes till tender; then turn off
the water and stand them on the back of the stove with
a cover half over them, where they will keep hot while
they get dry and floury, but do not let them burn; shake
the saucepan every little while.  Heat the milk with the
butter, salt, and pepper in it; mash the potatoes well,
either with the wooden potato-masher or with a wire one,
and put in the milk little by little.  When they are all
free from lumps, put them through the potato-ricer, or
pile them lightly in the tureen as they are.  Do not
smooth them over the top.
\begin{RecipeTitle}
Sweet Potatoes\label{sweet_potatoes}
\end{RecipeTitle}
\instruction  If they are large, scrub them well and bake in a hot
oven for about forty minutes.  If they are small, make
them into---
\begin{RecipeTitle}
Creamed Sweet Potatoes\label{creamed_sweet_potatoes}
\end{RecipeTitle}
\instruction  Boil the potatoes, skin them, and cut them up in small
slices.  Make a cup of cream sauce, mix with them, and
put them in the oven for half an hour.
\begin{RecipeTitle}
Scalloped Sweet Potatoes\label{scalloped_sweet_potatoes}
\end{RecipeTitle}
\instruction  Boil six potatoes in well-salted water till they are tender;
skin them, slice them thin, and put a layer of them in a
buttered baking-dish;  sprinkle with brown sugar, and put
on more potatoes and more sugar till the dish is full.
Bake for three-quarters of an hour.
\begin{RecipeTitle}
Beets\label{beets}
\end{RecipeTitle}
\instruction  Wash the beets but do not peel them.  Boil them gently
for three-quarters of an hour, or till they can be pierced
easily with a straw.  Then skin them and slice in a hot
dish, dusting each layer with a little salt, pepper, and
melted butter.  Those which are left over may have a little
vinegar poured over them, to make them into pickles for
luncheon.
\smallskip
\indpar
  Once Margaret made something very nice by a recipe her
Pretty Aunt put in her book.  It was called---
\begin{RecipeTitle}
                   Stuffed Beets\label{stuffed_beets}
\end{RecipeTitle}
\ingredient  1 can French peas.
\ingredient  6 medium-sized beets.
\instruction  Boil the beets as before and skin them, but leave them
whole.  Heat the peas after the juice has been turned off,
and season them with salt and pepper.  Cut off the stem
end of each beet so it will stand steadily, and scoop
a round place in the other end; sprinkle each beet with
salt and pepper, and put a tiny bit of butter down in
this little well, and then fill it high with the peas
it will hold.
\begin{RecipeTitle}
                  Creamed Cabbage\label{creamed_cabbage}
\end{RecipeTitle}
\ingredient  1 small cabbage.
\ingredient  1 cup cream sauce.
\instruction  Take off the outside leaves of the cabbage; cut it up in
four pieces, and cut out the hard core and lay it in cold,
salted water for half an hour.  Then wipe it dry and slice
it, not too fine, and put it in a saucepan; cover it with
boiling water with a teaspoonful of salt in it, and boil
hard for fifteen minutes without any cover.  While it is
cooking, make a cup of cream sauce.  Take up the cabbage,
press it in the colander with a plate till all the water
is out; put it in a hot covered dish, sprinkle well with
salt, and pour the cream sauce over.  This will not have
any unpleasant odor in cooking, and it will be so tender
and easy to digest that even a little girl may have two
helpings.
\instruction  If you like it to look green, put a tiny bit of soda
in the water when you cook it.
\begin{RecipeTitle}
Lima Beans\label{lima_beans}
\end{RecipeTitle}
  Shell them and cook like peas; pour over them a half-cup
of cream sauce, if you like this better than having them dry.
\begin{RecipeTitle}
Peas\label{peas}
\end{RecipeTitle}
\instruction  Shell them and drop them into a saucepan of boiling water,
into which you have put a teaspoonful of salt and one of
sugar.  Boil them till they are tender, from fifteen
minutes, if they are fresh from the garden, to half an
hour or more, if they have stood in the grocer's for a
day or two.  When they are done they will have little
dents in their sides, and you can easily mash two or three
with a fork on a plate.  Then drain off the water, put in
three shakes of pepper, more salt if they do not taste just
right, and a piece of butter the size of a hickory-nut,
and shake them till the butter melts; serve in a hot
covered dish.
\begin{RecipeTitle}
String Beans\label{string_beans}
\end{RecipeTitle}
\instruction  Pull off the strings and cut off the ends; hold three or
four beans in your hand and cut them into long, very narrow
strips, not into square pieces.  Then cook them exactly as
you did the peas.\pagebreak[4]
\begin{RecipeTitle}
Stewed Tomatoes\label{stewed_tomatoes}
\end{RecipeTitle}
\ingredient  6 large tomatoes.
\ingredient  1 teaspoonful of salt.
\ingredient  1 teaspoonful of sugar.
\ingredient  1 pinch soda.
\ingredient  3 shakes of pepper.
\ingredient  Butter as large as an English walnut.
\instruction  Peel and cut the tomatoes up small, saving the juice;
put together in a saucepan with the seasoning, the soda
mixed in a teaspoonful of water before it is put in.  Simmer
twenty minutes, stirring till it is smooth, and last put in
half a cup of bread or cracker crumbs, or a cup of toast,
cut into small bits.  Serve in a hot, covered dish.
\begin{RecipeTitle}
Asparagus\label{asparagus}
\end{RecipeTitle}
\instruction  Untie the bunch, scrape the stalks clean, and put it in
cold water for half an hour.  Tie the bunch again, and cut
enough off the white ends to make all the pieces the same
length.  Stand them in boiling water in a porcelain kettle,
and cook gently for about twenty minutes.  Lay on a platter
on squares of buttered toast, and pour over the toast and
the tips of the asparagus a cup of cream sauce.  Or do not
put it on toast, but pour melted butter over the tips after
it is on the platter.  To make it delicious, mix the juice
of a lemon with the butter.
\instruction  Sometimes put a little grated cheese on the ends last
of all.
\begin{RecipeTitle}
Onions\label{onions}
\end{RecipeTitle}
\instruction  Peel off the outside skin and cook them in boiling, salted
water till they are tender; drain them, put them in a
baking-dish, and pour over them a tablespoonful of melted
butter, three shakes of pepper, and a sprinkling of salt,
and put in the oven and brown a very little.  Or, cover
them with a cup of white sauce instead of the melted butter,
and sprinkle with salt and pepper, but do not put in the oven.
\begin{RecipeTitle}
Corn\label{corn}
\end{RecipeTitle}
\instruction  Strip off the husks and silk, and put in a kettle of
boiling water and boil hard for fifteen minutes; do not
salt the water, as salt makes corn tough.  Put a napkin
on a platter with one end hanging over the end; lay the corn
on and fold the end of the napkin over to keep it warm.
\begin{RecipeTitle}
Canned Corn\label{canned_corn}
\end{RecipeTitle}
\instruction  Turn the corn into the colander and pour water through it a
moment.  Heat a cup of milk with a tablespoonful of butter, a
teaspoonful of salt, and three shakes of pepper, and mix with
the corn and cook for two minutes.  Or, put in a buttered
baking-dish and brown in the oven.  Many people never wash
corn; it is better to do so.
\medskip
\indpar
  Sometimes Margaret had boiled rice for dinner in place of
potatoes, and then she looked back at the recipe she used when
she cooked it for breakfast, and made it in just the same way.
Very often in winter she had---
\begin{RecipeTitle}
                      Macaroni\label{macaroni}
\end{RecipeTitle}
\ingredient  6 long pieces of macaroni.
\ingredient  1 cup white sauce.
\ingredient  \OneHalf pound of cheese.
\ingredient  Paprika and salt.
\instruction  Break up the macaroni into small pieces, and boil fifteen
minutes in salted water, shaking the dish often.  Pour off the
water and hold the dish under the cold-water faucet until all
the paste is washed off the outside of the macaroni, which
will take only a minute if you turn it over once or twice.
Butter a baking-dish, put in a layer of macaroni, a good
sprinkle of salt, then a very little white sauce, and a layer
of grated cheese, sprinkled over with a tiny dusting of
paprika, or sweet red pepper, if you have it; only use a tiny
bit.  Then cover with a thin layer of white sauce, and so on
till the dish is full, with the last layer of white sauce
covered with an extra thick one of cheese.  Bake till brown.
\instruction  Margaret's mother got this rule in Paris, and she though it
a very nice one.
\smallskip
\indpar
  After the soup, meat, and vegetables at dinner came the
salad; for this Margaret almost always had lettuce, with
French dressing, as mayonnaise seemed too heavy for dinner.
Sometimes she had nice watercress; once in a long time she had
celery with mayonnaise.
\bigskip
\begin{center}
\hstroke
\end{center}\pagebreak[4]
\begin{FoodTypeTitle}
DESSERTS\label{DESSERTS}
\end{FoodTypeTitle}
\begin{RecipeTitle}
Corn-starch Pudding\label{plain_cornstarch_pudding}
\end{RecipeTitle}
\ingredient  1 pint of milk.
\ingredient  2 heaping tablespoonfuls of corn-starch.
\ingredient  3 tablespoonfuls of sugar.
\ingredient  Whites of three eggs.
\ingredient  \OneHalf teaspoonful vanilla.
\instruction  Beat the whites of the eggs very stiff.
Mix the corn-starch
with half a cup of the milk, and stir till it melts.  Mix the
rest of the milk and the sugar, and put them on the fire in
the double boiler.  When it bubbles, stir up the corn-starch
and milk well, and stir them in and cook and stir till it gets
as thick as oatmeal mush; then turn in the eggs and stir them
lightly, and cook for a minute more.  Take it off the stove,
mix in the vanilla, and put in a mould to cool.  When dinner
is ready, turn it out on a platter and put small bits of red
jelly around it, or pieces of preserved ginger, or a pretty
circle of preserved peaches, or preserved pineapple.  Have a
pitcher of cream to pass with it, or have a nice bowl of
whipped cream.  If you have a ring-mould, let it harden in
that, and have the whipped cream piled in the centre after it
is on the platter, and put the jelly or preserves around last.
\begin{RecipeTitle}
Chocolate Corn-starch Pudding\label{chocolate_cornstarch_pudding}
\end{RecipeTitle}
\instruction  Use the same rule as before, but put in one more
tablespoonful of sugar.  Then shave thin two squares of
Baker's chocolate, and stir in over the teakettle till it
melts, and stir it in very thoroughly before you put in the
eggs.  Instead of pouring this into one large mould, put it in
egg-cups to harden; turn these out carefully, each on a
separate plate, and put a spoonful of whipped cream by each one.
\begin{RecipeTitle}
Cocoanut Corn-starch Pudding\label{cocoanut_cornstarch_pudding}
\end{RecipeTitle}
\instruction  Make the first rule; before you put in the eggs, stir in a
cup of grated cocoanut, with an extra spoonful of sugar, or a
cup of that which comes in packages without more sugar, as it
is already sweetened.  Serve in a large mould, or in small
ones, with cream.\pagebreak[4]
\begin{RecipeTitle}
Baked Custard\label{baked_custard}
\end{RecipeTitle}
\ingredient  2 cups milk.
\ingredient  Yolks of two eggs.
\ingredient  2 tablespoonfuls of sugar.
\ingredient  A little nutmeg.
\instruction  Beat the eggs till they are light; mix the milk and sugar
till the sugar melts; put the two together, and put it into a
nice baking-dish, or into small cups, and dust the nutmeg over
the tops.  Bake till the top is brown, and till when you put a
knife-blade into the custard it comes out clean.
\begin{RecipeTitle}
Cocoanut Custard\label{cocoanut_custard}
\end{RecipeTitle}
\instruction  Add a cup of cocoanut to this rule and bake it in one dish,
stirring it up two or three times from the bottom, but, after
it begins to brown, leaving it alone to finish.  Do not put
any nutmeg on it.
\begin{RecipeTitle}
Tapioca Pudding\label{tapioca_pudding}
\end{RecipeTitle}
\ingredient  2 tablespoonfuls tapioca.
\ingredient  Yolks of two eggs.
\ingredient  \OneHalf cup of sugar.
\ingredient  1 quart of milk.
\instruction  Put the tapioca into a small half-cup of water and let it
stand one hour.  Then drain it and put it in the milk in the
double boiler, and cook and stir it till the tapioca looks
clear, like glass.  Beat the eggs and mix the sugar with them,
and beat again till both are light, and put them with the milk
and tapioca and cook three minutes, stirring all the time.
Then take it off the fire and add a saltspoonful of salt and a
half-teaspoonful of vanilla, and let it get perfectly cold.
\begin{RecipeTitle}
Floating Island\label{floating_island}
\end{RecipeTitle}
\ingredient  1 pint milk.
\ingredient  3 eggs.
\ingredient  One-third cup of sugar.
\instruction  Put the milk on the stove to heat in a good-sized pan.  Beat
the whites of the eggs very stiff, and as soon as the milk
scalds,---that is, gets a little wrinkled on top,---drop
spoonfuls of the egg on to it in little islands; let them
stand there to cook just one minute, and then with the skimmer
take them off and lay them on a plate.  Put the milk where it
will keep hot but not boil while you beat the yolks of the
eggs stiff, mixing in the sugar and beating that, too.  Pour
the milk into the bowl of egg, a little at a time, beating all
the while, and then put it in the double boiler and cook till
it is as thick as cream.  Take it off the fire, stir in a
saltspoonful of salt and half a teaspoonful of vanilla, and
set it away to cool.  When it is dinner-time, strain the
custard into a pretty dish and slip the whites off the top,
one by one.  If you like, you can dot them over with very tiny
specks of red jelly.
\begin{RecipeTitle}
Cake and Custard\label{cake_and_custard}
\end{RecipeTitle}
\instruction  Make a plain boiled custard, just as before, with---
\ingredient  1 pint of milk.
\ingredient  Yolks of three eggs.
\ingredient  One-third cup of sugar.
\ingredient  1 saltspoonful of salt.
\ingredient  \OneHalf teaspoonful of vanilla.
\instruction  Beat the eggs and sugar, add the hot milk, and cook till
creamy, put in the salt and vanilla, and cool.  Then cut stale
cake into strips, or split lady-fingers into halves, and
spread with jam.  Put them on the sides and bottom of a flat
glass dish, and gently pour the custard over.
\begin{RecipeTitle}
Brown Betty\label{brown_betty}
\end{RecipeTitle}
\instruction  Peel,\, core,\, and\, slice six apples.  Butter a baking-dish and
sprinkle the inside all over with fine bread-crumbs.  Then
take six very thin slices of buttered bread and line the sides
and bottom of the dish.  Put a layer of apples an inch thick,
a thin layer of brown sugar, six bits of butter, and a dusting
of cinnamon, another layer of crumbs, another of apples and
sugar,\, and so on\, till the dish\, is full,\, with crumbs and
butter on top, and three tablespoonfuls of molasses poured
over.  Bake this one hour, and have hard sauce to eat with it.
\begin{RecipeTitle}
Lemon Pudding\label{lemon_pudding}
\end{RecipeTitle}
\ingredient  1 cup of sugar.
\ingredient  4 eggs.
\ingredient  2 lemons.
\ingredient  1 pint of milk.
\ingredient  1 tablespoonful of sugar.
\ingredient  2 tablespoonfuls of corn-starch.
\ingredient  1 pinch of salt.
\instruction  Wet the corn-starch with half a cup of the milk, and heat
what is left.  Stir up the corn-starch well, and when the milk
is hot put it in and stir; then boil five minutes, stirring
all the time.  Melt the butter, and put that in with a pinch
of salt, and cool it.  Beat the yolks of the eggs, and add the
sugar, the juice of both lemons, and the grated rind of one,
pour into the milk, and stir well; put in a buttered
baking-dish and bake till slightly brown.  Take it out of the
oven; beat the whites of two eggs with a tablespoonful of
granulated sugar, and pile lightly on top, and put in the oven
again till it is just brown.  This is a very nice rule.
\begin{RecipeTitle}
Rice Pudding with Raisins\label{rice_pudding_with_raisins}
\end{RecipeTitle}
\ingredient  1 quart of milk.
\ingredient  2 tablespoonfuls of rice.
\ingredient  One-third cup of sugar.
\ingredient  \OneHalf cup seeded raisins.
\instruction  Wash the rice and the raisins and stir everything together
till the sugar dissolves.  Then put it in a baking-dish in the
oven.  Every little while open the door and see if a light
brown crust is forming on top, and, if it is, stir the pudding
all up from the bottom and push down the crust.  Keep on doing
this till the rice swells and makes the milk all thick and
creamy, which it will after about an hour.  Then let the
pudding cook, and when it is a nice deep brown take it out and
let it get very cold.
\begin{RecipeTitle}
 Bread Pudding\label{bread_pudding}
\end{RecipeTitle}
\ingredient  2 cups of milk.
\ingredient  1 cup soft bread-crumbs.
\ingredient  1 tablespoonful of sugar.
\ingredient  2 egg yolks.
\ingredient  1 egg white.
\ingredient  \OneHalf teaspoonful vanilla.
\ingredient  1 saltspoonful of salt.
\instruction  Crumb the bread evenly and soak in the milk till soft.  Beat
it till smooth, and put in the beaten yolks of the eggs, the
sugar, vanilla, and salt, and last the beaten white of the
egg.  Put it in a buttered pudding-dish, and stand this in a
pan of hot water in the oven for fifteen minutes.  Take it out
and spread its top with jam, and cover with the beaten white
of the other egg, with one tablespoonful of granulated sugar
put in it, and brown in the oven.  You can eat this as it is,
or with cream, and you may serve it either hot or cold.
  Sometimes you can put a cup of washed raisins into the
bread-crumbs and milk, and mix in the other things; sometimes
you can put in a cup of chopped almonds, or a little preserved
ginger.  Orange marmalade is especially nice on bread pudding.
\begin{RecipeTitle}
Orange Pudding\label{orange_pudding}
\end{RecipeTitle}
\instruction  Make just like Lemon Pudding, but use three oranges instead
of two lemons.
\begin{RecipeTitle}
Cabinet Pudding\label{cabinet_pudding}
\end{RecipeTitle}
\ingredient  1 pint of milk.
\ingredient  Yolks of three eggs.
\ingredient  3 tablespoonfuls of sugar.
\ingredient  1 saltspoonful of salt.
\instruction  Beat the eggs, add the sugar, and stir them into the milk,
which must be very hot but not boiling; stir till it thickens,
and then take it from the fire.  Put a layer of washed raisins
in the bottom of a mould, then a layer of slices of stale cake
or lady-fingers, then more raisins around the edge of the
mould, and more cake, till the mould is full.  Pour the
custard over very slowly, so the cake will soak well, and bake
in a pan of water in the oven for an hour.  This pudding is to
be eaten hot, with any sauce you like, such as Foamy Sauce.
\instruction  Cut-up figs are nice to use with the raisins, and chopped
nuts are a delicious addition, dropped between the layers of
the cake.
\begin{RecipeTitle}
Cottage Pudding\label{cottage_pudding}
\end{RecipeTitle}
\ingredient  1 egg.
\ingredient  \OneHalf cup of sugar.
\ingredient  \OneHalf cup of milk.
\ingredient  1\OneHalf teaspoonfuls of baking-powder.
\ingredient  1 cup of flour.
\ingredient  1 tablespoonful of butter.
\instruction  Beat the yolk of the egg light, add the sugar and butter
mixed, then put in the milk, the flour, the whites of the eggs
beaten stiff, and last of all the baking-powder, and stir it
up well.  Put in a greased pan and bake nearly half an hour.
If you want this very nice, put in half a cup of chopped figs,
mixed with part of the flour.
\instruction  Serve with Foamy Sauce.
\begin{RecipeTitle}
Prune Whips\label{prune_whips}
\end{RecipeTitle}
\instruction  This was a cooking-school rule which the Pretty Aunt put in,
because she said it was the best sort of pudding for little
girls to make.
\ingredient  1 tablespoonful of powdered sugar.
\ingredient  2 tablespoonfuls stewed prunes.
\ingredient  White of one egg.
\instruction  Cook the prunes till soft, take out the stones, and mash the
prunes fine.  Beat the white of the egg very stiff, mix in the
sugar and prunes, and bake in small buttered dishes.  Serve
hot or cold, with cream.
\begin{RecipeTitle}
Junket\label{junket}
\end{RecipeTitle}
\ingredient  1 junket tablet.
\ingredient  1 quart milk.
\ingredient  \OneHalf cup sugar.
\ingredient  1 teaspoonful vanilla.
\instruction  Break up the junket tablet into small pieces, and put them
into a tablespoonful of water to dissolve.  Put the sugar into
the milk with the vanilla, and stir till it is dissolved.
Warm the milk a little, but only till it is as warm as your
finger, so that if you try it by touching it with the tip, you
do not feel it at all as colder or warmer.  Then quickly turn
in the water with the tablet melted in it, stirring it only
once, and pour immediately into small cups on the table.
These must stand for half and hour without being moved, and
then the junket will be stiff, and the cups can be put in the
ice-box.  In winter you must warm the cups till they are like
the milk.  This is very nice with a spoonful of whipped cream
on each cup, and bits of preserved ginger or of jelly on it.
\begin{RecipeTitle}
Strawberry Shortcake\label{strawberry_shortcake}
\end{RecipeTitle}
\instruction  Margaret's mother called this the Thousand Mile Shortcake,
because she sent so far for the recipe to the place where she
had once eaten it, when she thought it the best she had ever
tasted.
\ingredient  1 pint flour.
\ingredient  \OneHalf cup butter.
\ingredient  1 egg.
\ingredient  1 teaspoonful baking-powder.
\ingredient  \OneHalf cup milk.
\ingredient  1 saltspoonful of salt.
\instruction  Mix the baking-powder and salt with the flour and sift all
together.  The butter should stand on the kitchen table till
it is warm and ready to melt, when it may be mixed in with a
spoon, and then the egg, well beaten, and the milk.
\instruction  Divide the dough into halves; put one in a round biscuit-tin,
butter it, and lay the other half on top, evenly. Bake
a light brown; when you take it out of the oven, let it
cool, and then lift the layer apart.  Mash the berries,
keeping out some of the biggest ones for the top of the cake,
and put on the bottom layer; put a small half-cup of powdered
sugar on them, and put the top layer on.  Dust this over with
sugar till it is white, and set the large berries about on it,
or cover the top with whipped cream and put the berries on
this.
\begin{RecipeTitle}
Cake Shortcake\label{cake_shortcake}
\end{RecipeTitle}
\ingredient  1 small cup sugar.
\ingredient  \OneHalf cup butter.
\ingredient  1 cup cold water.
\ingredient  1 egg.
\ingredient  2 cups flour.
\ingredient  3 teaspoonfuls baking-powder.
\instruction  Rub the butter and sugar to a cream; sift the flour and
baking-powder together; beat the egg stiff without separating;
put the egg with the sugar and butter, add the water and
flour in turn, a little at a time, stirring steadily; bake in
two layer-tins.  Put crushed berries between, and whole
berries on top.
\instruction  Tiny field strawberries make the most delicious shortcake of
all.
\begin{RecipeTitle}
Peach Shortcake\label{peach_shortcake}
\end{RecipeTitle}
\instruction  Make either of the rules above,\, and\, put\, mashed and sweetened
peaches between the layers.  Slice evenly about four more, and
arrange these on top, making a ring of them overlapping all
around the edge, and laying them inside in the same way.
Sugar well, and serve with whipped cream or a pitcher of plain
cream.
\begin{RecipeTitle}
Lemon Jelly\label{lemon_jelly}
\end{RecipeTitle}
\ingredient  \OneHalf box gelatine.
\ingredient  \OneHalf cup cold water.
\ingredient  2 cups boiling water.
\ingredient  1 cup sugar.
\ingredient  Juice of three lemons, and three scrapings of the yellow
rind.
\instruction  Put the gelatine into the cold water and soak one hour.  Put
the boiling water, the sugar, and the scrapings of the peel on
the fire, and still till the sugar dissolves.  Take it off the
fire and stir in the gelatine, and mix till this is dissolved;
when it is partly cool, turn in the lemon juice and strain
through a flannel bag dipped in water and wrung dry.  Put in a
pretty mould.
\begin{RecipeTitle}
Orange Jelly\label{orange_jelly}
\end{RecipeTitle}
\instruction  Make this exactly as you did the lemon jelly, only instead
of taking the juice of three lemons, take the juice of two
oranges and one lemon, and scrape the orange peel instead of
the lemon peel.
\instruction  Whipped cream is nicer with either of these jellies.
\begin{RecipeTitle}
Prune Jelly\label{prune_jelly}
\end{RecipeTitle}
\instruction  Wash well a cup of prunes, and cover them with cold water
and soak overnight.  In the morning put them on the fire in
the same water, and simmer till so tender that the stones will
slip out.  Cut each prune in two and sprinkle with sugar as
you lay them in the mould; pour over them lemon jelly made by
the recipe above, and put on ice.  Turn out on a pretty dish,
and put whipped cream around.
\bigskip
\indpar
  Sometimes Margaret colored lemon jelly with red raspberry
juice, and piled sugared raspberries around the mould.  Lemon
jelly is one of the best things to put things with; peaches
may be used instead of prunes, in that rule, or strawberries,
with plenty of sugar, or bits of pineapple.
\begin{RecipeTitle}
Fruit Jelly\label{fruit_jelly}
\end{RecipeTitle}
\instruction  Make a plain lemon jelly, as before.  Cut up very thin two
oranges, one banana, six figs, and a handful of white grapes,
which you have seeded, and sweeten them.  Put in a mould and
pour in the jelly; as it begins to grow firm you can gently
lift the fruit from the bottom once or twice.
\instruction  You can also fill the mould quite full of fruit, and make
only half the jelly and pour over.  Whipped cream is nice to
eat with this.
\begin{RecipeTitle}
Coffee Jelly\label{coffee_jelly}
\end{RecipeTitle}
\ingredient  \OneHalf box of gelatine.
\ingredient  \OneHalf cup of cold water.
\ingredient  1 pint strong hot coffee.
\ingredient  \ThreeQuarters cup sugar.
\ingredient  \OneHalf pint boiling water.
\instruction  Put the gelatine in the cold water and soak two minutes, and
pour over it the coffee, boiling hot.  When it is dissolved,
put in the sugar and boiling water and strain;  put in little 
individual moulds, and turn out with whipped cream under each
one.  Or, set in a large mould, and have whipped cream around
it.
\begin{RecipeTitle}
Snow Pudding\label{snow_pudding}
\end{RecipeTitle}
\ingredient  \OneHalf box of gelatine.
\ingredient  1 pint of cold water.
\ingredient  3 eggs.
\ingredient  Juice of three lemons.
\ingredient  \OneHalf cup of powdered sugar.
\instruction  Pour the water over the gelatine and let it stand ten
minutes; then put the bowl over the fire and stir till it is
dissolved, and take it off at once.  As soon as it seems
nearly cold, beat to a froth with the egg-beater.  Beat the
whites of the eggs stiffly, and add to the gelatine, with the
lemon juice and sugar, and mix well.  Put in a mould and set
on ice.  Make a soft custard by the rule, and pour around the
pudding when you serve it.
\begin{RecipeTitle}
Velvet Cream\label{velvet_cream}
\end{RecipeTitle}
\ingredient  \OneQuarter box of gelatine.
\ingredient  1 pint milk.
\ingredient  2 eggs.
\ingredient  3 tablespoonfuls of sugar.
\ingredient  Small teaspoonful of vanilla.
\instruction  Put the gelatine in the milk and soak fifteen minutes; put
on the stove and heat till it steams, but do not let it boil;
stir carefully often, as there is danger of its burning.  Beat
the yolks of the eggs with the sugar, and put these in the
custard, and cook till it all thickens and is smooth, but do
not boil it.  Strain, cool, and add the vanilla, and last fold
in the beaten whites of the eggs, and put in a mould on the
ice.
\instruction  Preserved peaches laid around this are very nice, or rich
pineapple, or apricot jam; or a ring of whipped cream, with
bits of red jelly, make a pretty border.
\begin{RecipeTitle}
Easy Charlotte Russe\label{easy_charlotte_russe}
\end{RecipeTitle}
\ingredient  \OneQuarter box gelatine.
\ingredient  \OneHalf pint of milk.
\ingredient  1 pint thick cream.
\ingredient  \OneHalf cup powdered sugar.
\ingredient  1 small teaspoonful vanilla.
\instruction  Put the gelatine in the milk and stand on the stove till the
gelatine is dissolved, stirring often.  Then take it off, and
beat with the egg-beater till cold.  Beat the cream with the
egg-beater till perfectly stiff, put in the sugar and
vanilla, and mix with the milk, and set on ice in a mould.
When you wish to use it, turn out and put lady-fingers split
in halves all around it.
\medskip
\begin{center}
\hstroke
\end{center}
\smallskip
\begin{FoodTypeTitle}
PUDDING SAUCES\label{PUDDING_SAUCES}
\end{FoodTypeTitle}
\begin{RecipeTitle}
Orange Sauce\label{orange_pudding_sauce}
\end{RecipeTitle}
\ingredient  3 egg-whites.
\ingredient  \OneHalf cup powdered sugar.
\ingredient  Juice of 2 oranges.
\ingredient  Grated rind.
\instruction  Beat the egg-whites very stiff, add the sugar, then the
grated orange-peel, then the juice; beat up lightly and serve
at once.
\begin{RecipeTitle}
Delicious Maple Sauce\label{delicious_maple_pudding_sauce}
\end{RecipeTitle}
\ingredient  2 egg-yolks.
\ingredient  \OneQuarter cup maple syrup.
\ingredient  \OneHalf cup whipped cream.
\instruction  Beat the yolks very light, putting in a pinch of salt; put
in the syrup and cook till the spoon coats over when you dip
it in; then cool and beat in the whipped cream, and serve very
cold.
\begin{RecipeTitle}
Hard Sauce\label{hard_pudding_sauce}
\end{RecipeTitle}
\instruction  Beat together a half-cup of powdered sugar and a half-cup of
butter with a fork till both are light and creamy.  Flavor
with a teaspoonful of vanilla and put on the ice to harden.
\begin{RecipeTitle}
Foamy Sauce\label{foamy_pudding_sauce}
\end{RecipeTitle}
\ingredient  \OneHalf cup butter.
\ingredient  \OneHalf cup boiling water.
\ingredient  1 cup powdered sugar.
\ingredient  1 teaspoonful vanilla.
\ingredient  White of one egg.
\instruction  Rub the butter and sugar to a cream; add vanilla and beat
well.  When it is time to serve, beat the egg stiff, stir the
boiling water into the sugar and butter, and then put in the
egg and beat till foamy, standing it on the stove as you do
so, to keep it hot.  Serve in the sauce-boat.\pagebreak[4]
\begin{RecipeTitle}
Grandmother's Sauce\label{grandmothers_pudding_sauce}
\end{RecipeTitle}
\ingredient  1 cup sugar.
\ingredient  \OneHalf cup butter.
\ingredient  Yolks of two eggs.
\ingredient  \OneQuarter cup boiling water.
\ingredient  A dusting of nutmeg.
\instruction  Cream the butter and sugar, stir in the beaten yolk, and
last the boiling water.  Beat till foamy, and then dust with
nutmeg.
\begin{RecipeTitle}
Lemon Sauce\label{lemon_pudding_sauce}
\end{RecipeTitle}
\ingredient  White of one egg.
\ingredient  \OneHalf cup powdered sugar.
\ingredient  Juice of half a lemon.
\instruction  Beat the egg, add the sugar and lemon, and beat again.
\begin{RecipeTitle}
White Sauce\label{white_pudding_sauce}
\end{RecipeTitle}
\ingredient  1 tablespoonful of corn-starch.
\ingredient  \OneHalf cup cold water.
\ingredient  1 cup boiling water.
\ingredient  \OneHalf cup powdered sugar.
\ingredient  Pinch of salt.
\ingredient  2 whites of eggs.
\ingredient  1 teaspoonful alons extract.
\instruction  Dissolve the corn-starch in the cold water, and then add the
boiling water and sugar and salt, and cook for fifteen
minutes, stirring all the time.  Take from the fire and fold
in the stiffly beaten egg-whites with the flavoring, and beat
till perfectly cold.  Any flavoring will do for this sauce;
pistache is very nice.
\begin{RecipeTitle}
Quick Pudding Sauce\label{quick_pudding_sauce}
\end{RecipeTitle}
\ingredient  1 egg.
\ingredient  \OneHalf cup powdered sugar.
\ingredient  1 teaspoonful vanilla.
\instruction  Put the egg in a bowl without separating it and beat till
very light; then pour in the sugar very slowly, beating all
the time; add the vanilla and serve at once.
\instruction  This is a very nice sauce, and so simple to make that
Margaret learned it among the first of her rules.
\begin{center}
\hstroke
\end{center}
\begin{FoodTypeTitle}
Ice-creams and Ices\label{packing_the_freezer}
\end{FoodTypeTitle}
\instruction  Margaret had a little ice-cream freezer which was all her
own, and held only enough for two little girls to eat at a
tea-party, and this she could pack alone.  When she made
ice-cream for all the family she had to use the larger
freezer, of course, and this Bridget helped her pack.  But the
same rule was used for either the large one or the small.
First break up the ice in a thick bag with a hammer until the
pieces are as large as eggs, and all about the same size.
Then put two big bowls or dippers of this into a tub or pail,
and add one bowl or dipper of coarse salt, and so on, till you
have enough, mixing it well with a long-handled spoon.  Put
the freezer in its pail and put the cover on; then fill the
space between with the ice and salt till it is full, pressing
it down as you work.  Let it stand now in a cool place, till
you know the inside is very cold, and then wipe off the top
carefully and pour in the cream, which must be very cold, too.
Put on the top and turn smoothly and slowly till it is stiff,
which should be fifteen minutes.  Then draw off the water from
the pail, wipe the top of the cover again, so no salt can get
in, and take out the dasher, pushing the cream down with a
spoon from the sides and packing it firmly.  Put a cork in the
hole in the cover, and put it on tightly.  Mix more ice with a
little salt; only a cupful to two bowls this time, and pack
the freezer again up to the top.  Wring out a heavy cloth in
the salty water you drew off the pail, and cover it over
tightly with this, and then stand in a cool, dark place till
you need it; all ice-creams are better for standing two hours.
\begin{RecipeTitle}
Plain Ice-cream\label{plain_ice_cream}
\end{RecipeTitle}
\ingredient  3 cups of cream.
\ingredient  1 cup of milk.
\ingredient  1 small cup of sugar.
\ingredient  2 teaspoonfuls vanilla.
\instruction  Put the cream, milk, and sugar on the fire, and stir till
the sugar dissolves and cream just wrinkles on top; do not let
it boil.  Take it off, beat it till it is cold, add the
vanilla, and freeze.
\begin{RecipeTitle}
French Ice-cream\label{french_ice_cream}
\end{RecipeTitle}
\ingredient  1 pint of milk.
\ingredient  1 cup of cream.
\ingredient  1 cup of sugar.
\ingredient  4 eggs.
\ingredient  1 tablespoonful vanilla.
\ingredient  1 saltspoonful of salt.
\instruction  Put the milk on the fire and let it just scald or wrinkle.
Beat the yolks of the eggs, put in the sugar, and beat again;
then pour the hot milk into these slowly, and the salt, and
put it on the fire in the double boiler and let it cook to a
nice thick cream.  (This is a plain boiled custard, such as
you made for floating island.)  Take it off and let it cool
while you beat the whites of the eggs stiff, and then the cup
of cream.  Put the eggs in first lightly when the custard is
entirely cold, and then the whipped cream last, and the
vanilla, and freeze.
\begin{RecipeTitle}
Coffee Ice-cream\label{coffee_ice_cream}
\end{RecipeTitle}
\instruction  Make either of these creams, and flavor with half a cup of
strong coffee in place of vanilla.
\begin{RecipeTitle}
Chocolate Ice-cream\label{chocolate_ice_cream}
\end{RecipeTitle}
\instruction Make plain ice-cream; melt two squares of chocolate in a
little saucer over the teakettle.  Mix a little of the milk or
cream with this, and stir it smooth, and then put it in with
the rest.  You will need to use a large cup of sugar instead
of a small one in making this, as the chocolate is not
sweetened.
\begin{RecipeTitle}
Peach Ice-cream\label{peach_ice_cream}
\end{RecipeTitle}
\instruction  Peel, cut up, and mash a cup of peaches.  Make plain
ice-cream, with a large cup of sugar, and when it is cold stir
in the peaches and freeze.
\begin{RecipeTitle}
Strawberry Ice-cream\label{strawberry_ice_cream}
\end{RecipeTitle}
\instruction  Mix a large cup\, of berries,\, mashed\, and strained carefully so
that there are no seeds, with the ice-cream, and freeze.
\begin{RecipeTitle}
The Easiest Ice-cream of All---\\
Vanilla Parfait\label{vanilla_parfait}
\end{RecipeTitle}
\ingredient  1 cup of sugar.
\ingredient  1 cup of water.
\ingredient  Whites of three eggs.
\ingredient  1 pint of cream.
\ingredient  1 teaspoonful vanilla.
\instruction  Put the sugar and water in a nice enamelled saucepan and
cook it without stirring.  You must shake the pan often to
prevent its burning, but if you stir it, it will make it
sugary.  After about five minutes hold your spoon up in the
air and drop one drop back into the saucepan; if a little
thread is made which blows off to one side, it is done, but
if not you must cook till it does.  If your fire is very hot
it may make the thread in less time, so try it every few
moments.  Have the whites of your eggs beaten very stiff, and
slowly pour the syrup into them, beating hard with a fork all
the time.  You must keep on beating till this is cold.  Have
ready a pint of thick cream, whipped very stiff, either with a
Dover egg-beater, or in a little tin cream-churn, and when the
egg is cold, mix the two lightly and put in the vanilla.  If
you have a mould with a tight cover, put it in this, but if
not, take a lard-pail; cover tightly, and stand in a pail on a
layer of ice and salt, mixed just as for freezing ice-cream,
and pile more ice and salt all over it, the more the better.
Let this stand five hours, or four will do, if necessary, and
turn the cream on a pretty dish.  After you have made this
once it will seem no trouble at all to make it.
\instruction  If your mother would like a change from this recipe
sometimes, try putting in the yolks of the eggs, well beaten,
with the cream, and use some other flavoring.\pagebreak[4]
\begin{RecipeTitle}
Lemon Ice\label{lemon_ice}
\end{RecipeTitle}
\ingredient  1 quart of water.
\ingredient  4 lemons.
\ingredient  2\OneHalf cups sugar.
\ingredient  1 orange.
\instruction  Boil the sugar and water for ten minutes; strain it and add
the juice of the lemons and orange; cool and freeze.
\begin{RecipeTitle}
Orange Ice\label{orange_ice}
\end{RecipeTitle}
\ingredient  1 quart of water.
\ingredient  6 oranges.
\ingredient  1 lemon.
\ingredient  2\OneHalf cups sugar.
\instruction  Prepare exactly as you did lemon ice.
\begin{RecipeTitle}
Strawberry Ice\label{strawberry_ice}
\end{RecipeTitle}
\ingredient  1 quart of water.
\ingredient  2\OneHalf cups sugar.
\ingredient  1\OneHalf cups strawberry juice, strained.  Prepare like lemon
ice.\pagebreak[4]
\begin{RecipeTitle}
Raspberry Ice\label{raspberry_ice}
\end{RecipeTitle}
\ingredient  1 quart of water.
\ingredient  2\OneHalf cups sugar.
\ingredient  1\OneHalf cups raspberry-juice, strained.  Prepare like lemon
ice.
\begin{RecipeTitle}
Peach Surprise\label{peach_surprise}
\end{RecipeTitle}
\ingredient  1 quart of peaches cut up in small bits.
\ingredient  2 cups of sugar.
\ingredient  Whites of five eggs.
\instruction  Do not beat the eggs at all; just mix everything together and
put in the freezer and stir till stiff; this is very
delicious, and the easiest thing to make there is.
\smallskip
\indpar
  When Margaret wanted to make her own freezer full of
ice-cream, she just took a cup of cream and heated it with the
sugar, and when it was cold put in three drops of vanilla and
froze it.
\begin{center}
\hstroke
\end{center}\pagebreak[4]
\begin{FoodTypeTitle}
CAKE\label{CAKE}
\end{FoodTypeTitle}
\indpar 
  Next after the ices in her book, Margaret found the cake to
eat with them, and first of all there was a rule for some
little cakes which the smallest girl in the neighborhood used
to make all alone.
\begin{RecipeTitle}
Eleanor's Cakes\label{eleanors_cakes}
\end{RecipeTitle}
\ingredient  \OneQuarter cup of butter.
\ingredient  \OneHalf cup of sugar.
\ingredient  \OneQuarter cup of milk.
\ingredient  1 egg.
\ingredient  1 cup flour.
\ingredient  1 teaspoonful baking-powder.
\ingredient  \OneHalf teaspoonful of vanilla.
\instruction  Rub the butter and sugar to a cream, beat the egg light
without separating, and put it in next; then the milk, a
little at a time; mix the baking-powder with the flour and
stir in, and last the vanilla.  Bake in small scalloped tins,
and fill each one only half-full.
\begin{RecipeTitle}
Grandmother's Little Feather Cake\label{grandmothers_little_feather_cake}
\end{RecipeTitle}
\ingredient  1 cup of sugar.
\ingredient  2 tablespoonfuls soft butter.
\ingredient  1 egg.
\ingredient  \OneHalf cup milk and water mixed.
\ingredient  1\OneHalf cups sifted flour.
\ingredient  1 teaspoonful baking-powder.
\instruction  Rub the butter and sugar to a cream.  Beat the yolk of the
egg stiff and put that in; then add part of the milk and water,
and part of the flour and baking-powder, which has been sifted
together; next the vanilla, and last the stiff whites of the
eggs, not stirred in, but just lightly folded in.  If you put
them in heavily and roughly, cake will always be heavy.  Bake
this in a buttered biscuit-tin, and cut in squares when cold.
It is nice covered with caramel or chocolate frosting.
\begin{RecipeTitle}
Domino Cake\label{domino_cake}
\end{RecipeTitle}
\instruction  Make this feather cake and pour it into two pans, so that
the bottom shall be just covered, and bake it quickly.  When
it is done, take it out of the pans and frost it, and while
the frosting is still a little soft, mark it off into
dominoes.  When it is entirely cold, cut these out, and with a
clean paint-brush paint little round spots on them with a
little melted chocolate, to exactly represent the real
dominoes.  It is fun to play a game with these at a tea-party
and eat them up afterwards.
\begin{RecipeTitle}
Margaret's Own Cake\label{margarets_own_cake}
\end{RecipeTitle}
\instruction  Margaret's mother named this cake for her, because she liked
it so much to make it and to eat it.  It is a very nice cake
for little girls.
\ingredient  5 eggs.
\ingredient  1 cup granulated sugar.
\ingredient  1 cup of flour.
\ingredient  1 pinch of salt.
\ingredient  \OneHalf teaspoonful of lemon-juice, or vanilla.
\instruction  Separate the eggs, and beat the yolks very light and foamy;
then put in the sugar which you have sifted, a little at a
time, and the flour in the same way, but put them in in turn,
first sugar, then flour, and so on.  Then put in the
flavoring, and last fold in the whites of the eggs, beaten
very stiff.  Bake in a buttered pan.
\begin{RecipeTitle}
Sponge Cake\label{sponge_cake}
\end{RecipeTitle}
\ingredient  4 eggs.
\ingredient  1 cup powdered sugar.
\ingredient  1 cup sifted flour.
\ingredient  1 level teaspoonful baking-powder.
\ingredient  Juice of half a lemon.
\instruction  Separate the yolks and whites of the eggs and beat them both
very light.  Mix the sugar in the yolks and beat again till
they are very foamy; then put in the stiff whites, and last
the flour, sifted with baking-powder; then the lemon-juice.
Bake in a buttered biscuit-tin.  You can frost and put
walnut-halves on top.
\begin{RecipeTitle}
Velvet Cake\label{velvet_cake}
\end{RecipeTitle}
\instruction  This is a large cake, baked in a roasting-pan; it is very
light and delicious, and none too large for two luncheons, or
for a picnic.
\ingredient  6 eggs.
\ingredient  2 cups of sugar.
\ingredient  1 cup of boiling water.
\ingredient  2\OneHalf cups of flour.
\ingredient  3 teaspoonfuls of baking-powder.
\instruction  Put the yolks of the eggs in a deep bowl and beat two
minutes; then put in the sugar, and beat ten minutes, or
fifteen, if you want it perfect.  Put in the water, a little
at a time, and next the stiffly beaten whites of the eggs.
Mix the baking-powder and flour, put these in next, and add
the flavoring last.  This is a queer way to mix the cake, but
it is right.
\begin{RecipeTitle}
Easy Fruit-cake\label{easy_fruit_cake}
\end{RecipeTitle}
\indpar  Margaret's Other Aunt begged to have this in the book,
because she said it was so simple any little girl could make
it, and all the family could help eat it, as they were
especially fond of fruit-cake.
\ingredient  1 cup butter.
\ingredient  1 cup sugar.
\ingredient  1 cup molasses.
\ingredient  1 cup milk.
\ingredient  1 cup currants.
\ingredient  1 cup raisins.
\ingredient  1 egg.
\ingredient  1 teaspoonful soda.
\ingredient  2 teaspoonfuls mixed spices.
\ingredient  3 cups flour.
\instruction  Wash and dry the currants.  Buy the seeded raisins and wash
these, too, and then chop them.  Cream the butter and sugar,
add the egg beaten well without separating, then the molasses
with the soda stirred in it, then the milk, then the cinnamon
and cloves.  Measure the flour, and then take out a half-cup
of it, and stir in the raisins and currants, to keep them from
going to the bottom of the cake when it is baked.  Stir these
in, add the rest of the flour, and beat well.  Bake in two
buttered bread-pans.
\begin{RecipeTitle}
Layer Cake\label{layer_cake}
\end{RecipeTitle}
\ingredient  1 cup sugar.
\ingredient  \OneHalf cup water.
\ingredient  2 eggs.
\ingredient  2 teaspoonfuls baking-powder.
\ingredient  \OneHalf cup butter.
\ingredient  2\OneHalf cups flour.
\ingredient  Teaspoonful vanilla.
\instruction  Rub the butter to a cream in a deep bowl, and put in the
sugar a little at a time, and rub this till it, too, creams.
Then put in the beaten yolks of the eggs, and then the water.
Beat the egg-whites well, and fold in half, then add the
flour, in which you have mixed and sifted the baking-powder,
and then put in the vanilla and the rest of the eggs.
\instruction  Divide in two layers, or in three if the tins are small, and
bake till a light brown.
\begin{center}
\hstroke
\end{center}\pagebreak[4]
\begin{FoodTypeTitle}
FILLING FOR LAYER CAKES\label{CAKE_FILLING}
\end{FoodTypeTitle}
\begin{RecipeTitle}
Nut and Raisin Filling\label{nut_and_raisin_filling}
\end{RecipeTitle}
\instruction  Make the rule for plain icing, and add a half-cup of chopped
raisins mixed with half a cup of chopped almonds or English
walnuts.
\begin{RecipeTitle}
Fig Filling\label{fig_filling}
\end{RecipeTitle}
\instruction  Mix a cup of chopped figs with the same icing.
\begin{RecipeTitle}
Marshmallow Filling\label{marshmallow_filling}
\end{RecipeTitle}
\instruction  Chop a quarter of a pound of marshmallows; put them over the
teakettle to get soft; make a plain icing and beat them in.
\begin{RecipeTitle}
Maple Filling\label{maple_filling}
\end{RecipeTitle}
\ingredient  2 cups maple syrup.
\ingredient  Whites of 2 eggs.
\instruction  Boil the syrup slowly till it makes a thread when you hold
it up; then add it slowly to your beaten egg-whites, beating
till cold.
\begin{RecipeTitle}
Orange Filling\label{orange_filling}
\end{RecipeTitle}
\ingredient  1 cup powdered sugar.
\ingredient  1 tablespoonful boiling water.
\ingredient  Grated rind of 1 orange.
\ingredient  1 tablespoonful orange-juice.
\instruction  Put the sugar in a bowl, add the rind, then the water and
juice, and spread at once on the cake.  This icing must be
very thick when made, and if is seems thin put in more sugar.
\begin{RecipeTitle}
Caramel Filling\label{caramel_filling}
\end{RecipeTitle}
\ingredient  2 cups brown sugar.
\ingredient  \OneHalf cup cream or milk.
\ingredient  Butter the size of an egg.
\ingredient  \OneHalf teaspoonful vanilla.
\instruction  Mix all together and cook till it is smooth and thick.
\begin{RecipeTitle}
Plain Icing\label{plain_icing}
\end{RecipeTitle}
\instruction  Put the white of one egg into a bowl with a half-teaspoonful
of water, and beat till light.  Then stir in a cup of sifted
powdered sugar, and put on the cake while that is still warm,
and smooth it over with a wet knife.
\begin{RecipeTitle}
Chocolate Icing\label{chocolate_icing}
\end{RecipeTitle}
\instruction  Melt one square of Baker's chocolate in a saucer over the
teakettle, and put in two tablespoonfuls of milk and stir till
smooth.  Add two tablespoonfuls of sugar and a small
half-teaspoonful of butter, and stir again.  Take it off the
stove and put it on the cake while both are warm.
\begin{RecipeTitle}
Caramel Icing\label{caramel_icing}
\end{RecipeTitle}
\ingredient  \OneHalf cup of milk.
\ingredient  2 cups brown sugar.
\ingredient  Butter the size of an egg.
\ingredient  1 teaspoonful of vanilla.
\instruction  Mix the butter, sugar, and milk, and cook till it is smooth
and thick, stirring all the time and watching it carefully to
see that it does not burn; take it off and put in the vanilla,
and spread while warm on a warm cake.
\begin{RecipeTitle}
Doughnuts\label{doughnuts}
\end{RecipeTitle}
\indpar
  Margaret's mother did not approve of put\-ting this rule in
her cook-book, because she did not want Margaret ever to eat
rich things; but her grandmother said it really must go in,
for once in awhile very nice doughnuts would not hurt anybody.
\ingredient  1\OneHalf cups of sugar.
\ingredient  \OneHalf cup of butter.
\ingredient  3 eggs.
\ingredient  1\OneHalf cups of milk.
\ingredient  2 teaspoonfuls baking-powder.
\ingredient  Pinch of salt.
\instruction  Put in flour enough to make a very soft dough, just as soft
as you can handle it.  Mix, and put on a slightly floured
board and make into round balls, or roll out and cut with a
cooky cutter with a hole in the centre.  Heat two cups of lard
with one cup of beef suet which you have melted and strained,
and heat till it browns a bit of bread instantly.  Then drop
in three doughnuts,---not more, or you will chill the fat,
--- and when you take them out dry on brown paper.  It is much
better to use part suet than all lard, yet that will do if you
have no suet in the house.
\begin{RecipeTitle}
Oatmeal Macaroons\label{oatmeal_macaroons}
\end{RecipeTitle}
\instruction  These little cakes are so like real macaroons that no one
who had not seen the recipe would guess how they were made.
\ingredient  2\OneHalf cups rolled oats.
\ingredient  2\OneHalf teaspoonfuls baking-powder.
\ingredient  \OneHalf teaspoonful salt.
\ingredient  3 even tablespoonfuls butter.
\ingredient  1 cup sugar.
\ingredient  3 eggs, beaten separately.
\ingredient  1 teaspoonful vanilla.
\instruction  Cream the butter, add the sugar and well beaten egg-yolks,
then the oatmeal, salt, and baking-powder, then the vanilla,
and last the whites of the eggs.  Drop in small bits, no
larger than the end of your finger, on a shallow pan, three
inches apart.  Bake in a very slow oven till brown, and take
from the pan while hot.
\begin{RecipeTitle}
Peanut Wafers\label{peanut_wafers}
\end{RecipeTitle}
\ingredient  1 cup of sugar.
\ingredient  \OneHalf cup of butter.
\ingredient  \OneHalf cup of milk.
\ingredient  \OneHalf teaspoonful soda.
\ingredient  2 cups of flour.
\ingredient  1 cup chopped peanuts.
\instruction  Cream the butter and sugar, put the soda in the milk and
stir well, and put this in next; add the flour and beat well.
Butter a baking-pan and spread this evenly over the bottom,
and then spread the peanuts over all.  Bake till a light
brown.
\begin{RecipeTitle}
Tea-party Cakes\label{tea_party_cakes}
\end{RecipeTitle}
\ingredient  2 squares of Baker's chocolate.
\ingredient  1 teaspoonful of sugar.
\ingredient  Bit of butter the size of a pea.
\instruction  Melt the chocolate over the teakettle and stir in the sugar
and butter and a couple of drops of vanilla, if you like.
Take little round crackers, and with a fork roll them quickly
in this till they are covered; dry on buttered paper.  You can
also take saltines, or any long, thin  cracker, and spread one
side with the chocolate.
\begin{RecipeTitle}
Almond Strips\label{almond_strips}
\end{RecipeTitle}
\ingredient  White of 1 egg.
\ingredient  1 cup chopped almonds.
\ingredient  2 tablespoonfuls powdered sugar.
\instruction  Beat the egg just a little and put in the sugar and almonds;
spread on thin crackers, and brown in the oven with the door
open.
\bigskip
\begin{center}
\hstroke
\end{center}
\begin{FoodTypeTitle}
PIES\label{PIES}
\end{FoodTypeTitle}
\begin{RecipeTitle}
General Rule\label{pies_general_rule}
\end{RecipeTitle}
\indpar
  Margaret's mother did not like her to eat pie, but she let
her learn how to make it, and once in awhile she had a small
piece.  Here is her rule:
\ingredient  1 pint of flour.
\ingredient  \OneQuarter cup of butter.
\ingredient  \OneQuarter cup lard, 1 teaspoonful salt.
\ingredient  \OneHalf cup ice-water.
\instruction  Put the flour, butter, lard, and salt in the chopping-bowl
and chop till well mixed.  Then add the water, a little at a
time, turning the paste\, and\, chopping\, till smooth,\, but never
touching with the hand.  Put a very little flour on the
pastry-board and lift the crust on this, and with a floured
rolling-pin lightly roll it out once each way; fold it over
and roll again, and do this several times till the crust looks
even, with no lumps of butter showing anywhere.  Put it on a
plate and lay it in the ice-chest for at least an hour before
you use it.
\instruction  Pie-crust will never be light and nice if you handle it.  Do
not touch it with your fingers unless it is really necessary.
When you use it, get everything ready for the pie first, and
then bring out the crust, roll quickly, and spread over the
pie.
\instruction  In putting the pie in the pan, cut the bottom piece a little
larger than you want it, as it will shrink.  Sprinkle the tin
with flour, lay on the crust, and after it has been fitted
evenly, and is not too tight, cut off the edge.  Put a narrow
strip of paste all around the edge, and press it together; if
you wet it with a little water it will stick.  If you wish to
be sure the filling of the pie will not soak into the under
crust, brush that over with beaten white of egg.  After you
put in the filling, fold your top crust together and cut some
little shutters to let out the steam.  Put on the cover, wet
the edges so they will stick together, and pinch evenly.
\begin{RecipeTitle}
Deep Apple Pie, or Apple Tart\label{deep_apple_pie_or_tart}
\end{RecipeTitle}
\instruction  Fill a baking-dish with apples, peeled and cut in slices.
Sprinkle with flour, cinnamon, and plenty of sugar, about half
a cup.  Put in the oven and bake till the apples are soft, and
then cool, put on the crust, and bake till brown.  Serve
powdered sugar and rich cream with this.  All pies cooked in a
baking-dish, with no crust on the bottom or sides of the dish,
are called tarts by the English.  They are the best kind of
pie.
\begin{RecipeTitle}
Peach Pie\label{peach_pie}
\end{RecipeTitle}
\instruction  Line a pie-plate with crust, lay in the peaches, peeled and
sliced, sprinkle with flour, and then cover with sugar; put on
a top crust, cut some little slits in it to let out the steam,
and cook till brown.  Or, make a deep peach tart.
\begin{RecipeTitle}
French Peach Pie\label{french_peach_pie}
\end{RecipeTitle}
\instruction  Put the crust in the pie-pan as before; boil a cup of sugar
with two tablespoonfuls of water till it threads.  Lay
quarters of peaches in the paste, around and around, evenly,
no one on top of the other.  Break ten peach-stones and
arrange evenly on top; the pour the syrup over, and put a few
narrow strips of crust across the pie, four each way, and
bake.\pagebreak[4]
\begin{RecipeTitle}
Pumpkin Pie\label{pumpkin_pie}
\end{RecipeTitle}
\ingredient  1 small pumpkin.
\ingredient  2\OneHalf cups of pulp.
\ingredient  2 cups of milk.
\ingredient  1 tablespoonful molasses.
\ingredient  2 eggs.
\ingredient  1 teaspoonful each of salt, ginger, cinnamon, and butter.
\ingredient  2 heaping tablespoonfuls of sugar.
\instruction  Cut the pumpkin in small pieces and take out the seeds and
remove the peel.  Put the good part over the kettle and steam
it till it is tender, keeping it covered.  Then you take off
the cover, and stand the steamer you have cooked it in on the
back of the stove, till the heat makes the pumpkin nice and
dry.  Then mash it and put it through the colander.  While it
is warm, mix in everything in the rule except the eggs; let it
cool, and put these in last, beating them till light.  Line
the pie-tin with crust, and pour in the filling and bake.
This rule is a very nice one; it makes two pies.
\begin{RecipeTitle}
Cranberry Pie\label{cranberry_pie}
\end{RecipeTitle}
\instruction  Cook a quart of cranberries till tender, with a small cup of
water; when they have simmered till rather thick, put in a
heaping cup of sugar and cook five minutes more.  When as
thick as oatmeal mush, take them off the fire and put through
the colander; line a tin with crust, fill with berries, put
strips of crust across, and bake.  A nice plan is to take half
a cup of raisins and a cup of cranberries for a pie, chopping
together and cooking with water as before, adding a sprinkling
of flour and a little vanilla when done.
\begin{RecipeTitle}
Orange Pie\label{orange_pie}
\end{RecipeTitle}
\ingredient  1 orange.
\ingredient  1 cup of water.
\ingredient  1 small cup of sugar.
\ingredient  2 teaspoonfuls corn-starch.
\ingredient  Butter the size of a hickory-nut.
\ingredient  Yolk of one egg.
\instruction  Grate\, the\, rind of the orange,\, and\, then squeeze out the juice.
Beat the yolk of the egg, add the water, with the corn-starch
stirred in, orange juice and rind and butter, and cook till it
grows rather thick.  Bake your crust first; then bake the
orange filling in it;  then beat the white of your egg with
a tablespoonful of granulated sugar, and put over it and brown.
This is an especially nice rule.
\begin{RecipeTitle}
Lemon Pie\label{lemon_pie}
\end{RecipeTitle}
\instruction  Make exactly as you did the orange-pie, but put in a
good-sized cup of sugar instead of a small one, with a lemon
in place of the orange.
\begin{RecipeTitle}
Tarts\label{tarts}
\end{RecipeTitle}
\instruction  Whenever\, Margaret\, made pie\, she\, always saved all the bits of
the crust and rolled them out, and lined patty-pans with them
and baked them.  She often filled them with raw rice while
they baked, to keep them in shape, saving the rice when they
were done.  She filled the shells with jelly, and used the
tarts for lunch.
\medskip
\begin{center}
\hstroke
\end{center}
\begin{FoodTypeTitle}
CANDY\label{CANDY}
\end{FoodTypeTitle}
\smallskip
\indpar
Margaret did not wait till she reached the recipes for candy
at the back of her book before she began to make it.  She made
it all the way along, whenever another little girl came to
spend the afternoon, or it was such a rainy day that she could
not go out.  Nearly always she made molasses candy, because it
was such fun to pull it, and she used the same rule her mother
used when she was a little girl.
\begin{RecipeTitle}
Molasses Candy\label{molasses_candy}
\end{RecipeTitle}
\ingredient  2 cups New Orleans molasses.
\ingredient  1 cup white sugar.
\ingredient  1 tablespoonful butter.
\ingredient  1 tablespoonful vinegar.
\ingredient  1 small teaspoonful soda.
\instruction  Boil hard twenty minutes, stirring all the time, and cool in
shallow pans.  If you double the rule you must boil the candy
five minutes longer.
\instruction  The best thing about this candy is that it does not stick to
the fingers, if you let it get quite cool before touching it,
and pull it in small quantities.  Do not put any butter on
your fingers, but work fast.
\begin{RecipeTitle}
Maple Wax\label{maple_wax}
\end{RecipeTitle}
\instruction  Boil two cups of maple syrup till it hardens when dropped in
cold water.  Fill a large pan with fresh snow, pack well; keep
the kettle on the back of the stove, where the syrup will be
just warm, but will not cook, and fill a small pitcher with
it, and pour on the snow, a little at a time.  Take it off in
small pieces with a fork.  If there is no snow, use a cake of
ice.
\begin{RecipeTitle}
Peanut Brittle\label{peanut_brittle}
\end{RecipeTitle}
\instruction  Make the molasses candy given above, and stir in a large cup
of shelled peanuts just before taking it from the fire.  Put
in shallow, buttered pans.
\begin{RecipeTitle}
Peppermint Drops\label{peppermint_drops}
\end{RecipeTitle}
\ingredient  1 cup sugar.
\ingredient  2 tablespoonfuls of water.
\ingredient  3 teaspoonfuls of peppermint essence.
\instruction  Boil the sugar and water till when you drop a little in
water it will make a firm ball in your fingers.  Then take it
off the fire and stir in the peppermint, and carefully drop
four drops, one exactly on top of another, on a buttered
platter.  Do not put these too near together.
\begin{RecipeTitle}
Pop-corn Balls\label{popcorn_balls}
\end{RecipeTitle}
\instruction  Make half the rule for molasses candy.  Pop a milk-can full
of corn, and pour in a little candy while it is hot; take up
all that sticks together and roll in a ball; then pour in
more, and so on.
\begin{RecipeTitle}
Maple Fudge\label{maple_fudge}
\end{RecipeTitle}
\ingredient  3 cups brown sugar.
\ingredient  2 cups maple syrup.
\ingredient  1 cup of milk.
\ingredient  \OneHalf cup of water.
\ingredient  Butter the size of an egg.
\ingredient  1 cup English walnut meats, or hickory-nuts.
\instruction  Boil the sugar and maple syrup till you can make it into a
very soft ball when you drop it in water; only half as hard as
you boil molasses candy.  Then put in the milk, water, and
butter, and boil till when you try in water it makes quite a
firm ball in your fingers.  Put in the nuts and take off the
fire at once, and stir till it begins to sugar.  Spread it
quickly on buttered pans, and when partly cool mark in squares
with a knife.
\begin{RecipeTitle}
Chocolate Fudge\label{chocolate_fudge}
\end{RecipeTitle}
\ingredient  1 cup of milk.
\ingredient  1 cup of sugar.
\ingredient  1 pinch of soda.
\ingredient  3 squares Baker's chocolate.
\ingredient  Butter the size of an egg.
\instruction  Put the soda in the milk and scrape the chocolate.  Mix all
together until when you drop a little in water it will make a
ball in your fingers.  Take off the fire then, and beat until
it is a stiff paste, and then spread on a buttered platter.
Sometimes Margaret added a cup of chopped nuts to this rule,
putting them in just before she took the fudge off the fire.
\begin{RecipeTitle}
Cream Walnuts\label{cream_walnuts}
\end{RecipeTitle}
\ingredient  2 cups of light brown sugar.
\ingredient  Two-thirds cup of boiling water.
\ingredient  1 small saltspoonful of cream of tartar.
\ingredient  1 cup chopped walnuts.
\instruction  Boil till the syrup makes a thread, then cool till it begins
to thicken, and stir in the walnuts and drop on buttered
paper.
\begin{RecipeTitle}
Cream Made from Confectioners' Sugar\label{cream_confectioners_sugar}
\end{RecipeTitle}
\instruction  Take the white of one egg, and measure just as much cold
water; mix the two well, and stir stiff with confectioners'
sugar; add a little flavoring, vanilla, or almond, or
pistache, and, for some candies, color with a tiny speck of
fruit paste.  This is the beginning of all sorts of cream
candy.
\begin{RecipeTitle}
Candy Potatoes\label{candy_potatoes}
\end{RecipeTitle}
\instruction  Make the plain white candy just given, and to it add a
tablespoonful of cocoanut, and flavor with vanilla.  Make into
little balls, rather long then round, and with a fork put eyes
in them like potato eyes.  Roll in cinnamon.  These candies
are very quickly made, and are excellent for little girls'
parties.
\begin{RecipeTitle}
Chocolate Creams\label{chocolate_creams}
\end{RecipeTitle}
\instruction  Make the cream candy into balls, melt three squares of
Baker's chocolate; put a ball on a little skewer or a fork,
and dip into the chocolate and lay on buttered paper.
\begin{RecipeTitle}
Nut Candy\label{nut_candy}
\end{RecipeTitle}
\instruction  Chop a cup of almonds and mix with the cream candy; make
into bars, and when cold cut in slices.
\begin{RecipeTitle}
Walnut Creams\label{walnut_creams}
\end{RecipeTitle}
\instruction  Press two walnut halves on small balls of cream candy, one
on either side.
\begin{RecipeTitle}
Creamed Dates\label{creamed_dates}
\end{RecipeTitle}
\instruction  Wash, wipe, and open the dates; remove the stones and put a
small ball of cream candy into each one.
\begin{RecipeTitle}
Butter Scotch\label{butter_scotch}
\end{RecipeTitle}
\ingredient  3 tablespoonfuls sugar.
\ingredient  3 tablespoonfuls of molasses.
\ingredient  2 tablespoonfuls of water.
\ingredient  1 tablespoonful of butter.
\ingredient  1 saltspoonful of soda.
\instruction  Boil all together without stirring till it hardens in water;
then put in a small teaspoonful of vanilla and pour at once on
a buttered platter.  When hard break up into squares.
\begin{RecipeTitle}
Pinoche\label{pinoche}
\end{RecipeTitle}
\ingredient  1 cup light brown sugar.
\ingredient  1 cup cream.
\ingredient  1 cup walnuts, chopped fine.
\ingredient  Butter the size of a walnut.
\ingredient  1 teaspoonful vanilla.
\instruction  Cook the sugar and cream till it makes a ball in water; then
put in the butter, vanilla, and nuts, and beat till creamy and
spread on a platter.
\begin{RecipeTitle}
Betty's Orange Candy\label{bettys_orange_candy}
\end{RecipeTitle}
\instruction  Betty was Margaret's particular friend, so this was her
favorite rule:
\ingredient  2 cups sugar.
\ingredient  Juice of one orange.
\instruction  Boil till it hardens in water, and then pull it.
\begin{RecipeTitle}
Creamed Dates, Figs, and Cherries\label{creamed_dates_figs_cherries}
\end{RecipeTitle}
\instruction Make the plain cream candy, as before; wash the dates well,
open at one side, and take out the stones and press in a ball
of the candy; leave the side open.  You can sprinkle with
granulated sugar if you choose.
\instruction   Cut figs in small pieces, and roll each piece in the cream
candy till it is hidden.
\instruction   For the cherries, color the cream candy light pink and make
into little balls.  On top of each press a candied cherry.
\begin{RecipeTitle}
Dates with Nuts\label{dates_with_nuts}
\end{RecipeTitle}
\instruction  Wash and wipe the dates dry, and take out the stones.  Put
half an English walnut in each and press the edges together;
roll in granulated sugar.  Small figs may be prepared in the
same way.
\begin{center}
\hstroke
\end{center}
\medskip
\begin{FoodTypeTitle}
MARGARET'S SCHOOL\\
LUNCHEONS\label{SCHOOL_LUNCHEONS}
\end{FoodTypeTitle}
\medskip 
\indpar
  As Margaret had to take her luncheon to school with her
sometimes, she had to learn how to make a good many kinds of
sandwiches, because she soon grew tired of one or two sorts.
\medskip
\instruction  Cut the bread very thin and spread lightly with butter,
and after they are done trim off the crusts neatly, not taking
off all the crust, but making the two pieces even.  For plain
meat sandwiches, chop the meat very fine, sprinkle with salt,
and spread on the bread; if it is too dry, put in a very little
cream as you chop the meat.
\begin{RecipeTitle}
Egg Sandwiches\label{egg_sandwiches}
\end{RecipeTitle}
\instruction  Make a very little French dressing,---about a teaspoonful
of oil, a sprinkling of salt, and four drops of lemon juice,
or vinegar.  Chop a hard-boiled egg very fine, mix with the
dressing, and spread.
\begin{RecipeTitle}
Lettuce Sandwiches\label{lettuce_sandwiches}
\end{RecipeTitle}
\instruction  Spread the bread, lay on a lettuce-leaf and cover with French
dressing, or with mayonnaise.  These sandwiches are about the
best for school, as they do not get dry.
\begin{RecipeTitle}
Celery Sandwiches\label{celery_sandwiches}
\end{RecipeTitle}
\instruction  Chop the celery fine, mix with a French or mayonnaise
dressing, and spread.
\begin{RecipeTitle}
Olive Sandwiches\label{olive_sandwiches}
\end{RecipeTitle}
\instruction  Chop six olives fine, mix with a tiny bit of mayonnaise
and spread.\pagebreak[4]
\begin{RecipeTitle}
Chicken and Celery Sandwiches\label{chicken_and_celery_sandwiches}
\end{RecipeTitle}
\instruction  Mix chopped celery and chopped chicken, as much of one
as the other, wet with French or mayonnaise dressing and spread.
\begin{RecipeTitle}
Nut Sandwiches\label{nut_sandwiches}
\end{RecipeTitle}
\instruction  Chop the nuts fine and add just enough cream to moisten;
sprinkle with salt and spread.
\begin{RecipeTitle}
Sardine Sandwiches\label{sardine_sandwiches}
\end{RecipeTitle}
\instruction  Scrape off all the skin from the sardines, and take out the
bones and drain them by laying them on brown paper; mash them
with a fork, and sprinkle with lemon juice, and spread.
\begin{RecipeTitle}
Tomato and Cheese Sandwiches\label{tomato_and_cheese_sandwiches}
\end{RecipeTitle}
\instruction  Slice a small, firm tomato very thin indeed, and take out
all the seeds and soft pulp, leaving only the firm part;
put one slice on the bread, and one thin shaving of cheese
over it, and then put on bread.  A slice of tomato with a
spreading of mayonnaise makes a nice sandwich.
\begin{RecipeTitle}
Cream Cheese and Nut Sandwiches\label{cream_cheese_and_nut_sandwiches}
\end{RecipeTitle}
\instruction  Spread thin Boston brown bread with just a scraping of
butter, then spread with cream cheese and cover with nuts;
this is a delicious sandwich.
\begin{RecipeTitle}
Sweet Sandwiches\label{sweet_sandwiches}
\end{RecipeTitle}
\instruction  All jams and jellies make good sandwiches, and fresh dates,
chopped figs, and preserved ginger are also nice.
\begin{FoodTypeTitle}
Some of Margaret's School Luncheons\label{school_luncheons}
\end{FoodTypeTitle}
\indpar  1. Two Boston brown bread, cream cheese, and nut sandwiches,
and two white bread and jam; a little round cake; a pear.
\indpar  2. Two chopped ham sandwiches, two with whole wheat bread
and peanut-butter; a piece of gingerbread; a peach.
\indpar  3. Two whole wheat-bread and chopped egg sandwiches with
French dressing; two crackers spread with jam; three thin
slices of cold meat, salted; a cup custard; an apple.
\indpar  4. Two whole wheat sandwiches spread with chopped celery and
French dressing, two of white bread and sardines; three
gingersnaps; three figs.
\indpar  5. Three sandwiches of white bread filled with cooked oysters,
chopped fine, one of whole wheat with orange marmalade; a few
pieces of celery, salted, a spice cake; a handful of nuts.
\indpar  6. Four sandwiches, two of minced chicken moistened with
cream, two of whole wheat and chopped olives; a little jar
of apple-sauce; gingerbread.
\indpar  7. Two date sandwiches, two of chopped cold meat; sugar
cookies; three olives; an orange.
\indpar  8. Two fig sandwiches, two whole wheat with chopped celery
and French dressing; a devilled egg; a little scalloped cake;
an apple.
\indpar  9. Three lettuce sandwiches, one with brown sugar and butter;
three tiny sweet pickles; ginger cookies; fresh plums.
\bigskip
\begin{center}
{\small THE END.}
\end{center}
\newpage
\thispagestyle{empty}
\ 
\newpage
\markboth{INDEX}{INDEX}
\thispagestyle{plain}
\vspace*{5ex}
\begin{center}
{\Large INDEX}\\
\smallskip
\hstroke
\smallskip
\end{center}
\begin{FoodTypeTitle}
BEVERAGES
\end{FoodTypeTitle}
Chocolate\ixfill\pageref{chocolate}\linebreak
Cocoa\ixfill\pageref{cocoa}\linebreak
Coffee\ixfill\pageref{coffee}\linebreak
Coffee, French\ixfill\pageref{french_coffee}\linebreak
Lemonade\ixfill\pageref{lemonade}\linebreak
Lemonade with Grape-juice\ixfill\pageref{lemonade_with_grape_juice}\linebreak
Tea\ixfill\pageref{tea}\linebreak
Tea, Iced\ixfill\pageref{iced_tea}\linebreak
\begin{FoodTypeTitle}
BREAD
\end{FoodTypeTitle}
Baking-powder Biscuit\ixfill\pageref{baking_powder_biscuit}\linebreak
Barneys\ixfill\pageref{barneys}\linebreak
Corn Bread, Grandmother's\ixfill\pageref{grandmothers_cornbread}\linebreak
Corn Bread, Perfect\ixfill\pageref{perfect_cornbread}\linebreak
Flannel Cakes\ixfill\pageref{flannel_cakes}\linebreak
Griddle-cakes\ixfill\pageref{griddle_cakes}\linebreak
Griddle-cakes, Sweet Corn\ixfill\pageref{sweet_corn_griddle_cakes}\linebreak
Milk Toast\ixfill\pageref{milk_toast}\linebreak
Muffins, Cooking-school\ixfill\pageref{cooking_school_muffins}\linebreak
Popovers\ixfill\pageref{popovers}\linebreak
Toast\ixfill\pageref{toast}\linebreak
Waffles\ixfill\pageref{waffles}\linebreak
\begin{FoodTypeTitle}
                        CAKE
\end{FoodTypeTitle}
Almond Strips\ixfill\pageref{almond_strips}\linebreak
Domino\ixfill\pageref{domino_cake}\linebreak
Doughnuts\ixfill\pageref{doughnuts}\linebreak
Eleanor's\ixfill\pageref{eleanors_cakes}\linebreak
Filling for Layer Cake:\\
\indix  Caramel\ixfill\pageref{caramel_filling}\linebreak
\indix  Fig\ixfill\pageref{fig_filling}\linebreak
\indix  Maple\ixfill\pageref{maple_filling}\linebreak
\indix  Marshmallow\ixfill\pageref{marshmallow_filling}\linebreak
\indix  Nut and Raisin\ixfill\pageref{nut_and_raisin_filling}\linebreak
\indix  Orange\ixfill\pageref{orange_filling}\linebreak
Frosting:\linebreak
\indix  Caramel\ixfill\pageref{caramel_icing}\linebreak
\indix  Chocolate\ixfill\pageref{chocolate_icing}\linebreak
\indix  Plain\ixfill\pageref{plain_icing}\linebreak
Fruit, Easy\ixfill\pageref{easy_fruit_cake}\linebreak
Gingerbread\ixfill\pageref{gingerbread}\linebreak
Gingerbread, Soft\ixfill\pageref{soft_gingerbread}\linebreak
Ginger Cookies\ixfill\pageref{ginger_cookies}\linebreak
Grandmother's Little Feather Cake\ixfill\pageref{grandmothers_little_feather_cake}\linebreak
Grandmother's Sugar Cookies\ixfill\pageref{grandmothers_sugar_cookies}\linebreak
Layer\ixfill\pageref{layer_cake}\linebreak
Margaret's Own\ixfill\pageref{margarets_own_cake}\linebreak
Oatmeal Macaroons\ixfill\pageref{oatmeal_macaroons}\linebreak
Peanut Wafers\ixfill\pageref{peanut_wafers}\linebreak
Sponge\ixfill\pageref{sponge_cake}\linebreak
Tea-party\ixfill\pageref{tea_party_cakes}\linebreak
Velvet\ixfill\pageref{velvet_cake}\linebreak
\begin{FoodTypeTitle}
                     CANDY
\end{FoodTypeTitle}
Betty's Orange\ixfill\pageref{bettys_orange_candy}\linebreak
Butter Scotch\ixfill\pageref{butter_scotch}\linebreak
Candy Potatoes\ixfill\pageref{candy_potatoes}\linebreak
Chocolate Creams\ixfill\pageref{chocolate_creams}\linebreak
Chocolate Fudge\ixfill\pageref{chocolate_fudge}\linebreak
Creamed Dates\ixfill\pageref{creamed_dates}\linebreak
Creamed Dates, Figs and Cherries\ixfill\pageref{creamed_dates_figs_cherries}\linebreak
Cream Walnuts\ixfill\pageref{cream_walnuts}\linebreak
Cream Made from Confectioners' Sugar\ixfill\pageref{cream_confectioners_sugar}\linebreak
Dates with Nuts\ixfill\pageref{dates_with_nuts}\linebreak
Maple Fudge\ixfill\pageref{maple_fudge}\linebreak
Maple Wax\ixfill\pageref{maple_wax}\linebreak
Molasses\ixfill\pageref{molasses_candy}\linebreak
Nut\ixfill\pageref{nut_candy}\linebreak
Peanut Brittle\ixfill\pageref{peanut_brittle}\linebreak
Peppermint Drops\ixfill\pageref{peppermint_drops}\linebreak
Pinoche\ixfill\pageref{pinoche}\linebreak
Pop-corn Balls\ixfill\pageref{popcorn_balls}\linebreak
Walnut Creams\ixfill\pageref{walnut_creams}\linebreak
\begin{FoodTypeTitle}
CEREALS
\end{FoodTypeTitle}
Corn-meal Mush\ixfill\pageref{cornmeal_mush}\linebreak
Corn-meal Mush, Fried\ixfill\pageref{fried_cornmeal_mush}\linebreak
Farina Croquettes\ixfill\pageref{farina_croquettes}\linebreak
Hominy\ixfill\pageref{hominy}\linebreak
Rice, Boiled\ixfill\pageref{boiled_rice}\linebreak
Rice Croquettes\ixfill\pageref{rice_croquettes}\linebreak
Rice, Fried\ixfill\pageref{fried_rice}\linebreak
\begin{FoodTypeTitle}
CHEESE
\end{FoodTypeTitle}
Fondu\ixfill\pageref{cheese_fondu}\linebreak
Scalloped\ixfill\pageref{scalloped_cheese}\linebreak
Welsh Rarebit, Easy\ixfill\pageref{easy_welsh_rarebit}\linebreak
\begin{FoodTypeTitle}
DESSERTS
\end{FoodTypeTitle}
Bread Pudding\ixfill\pageref{bread_pudding}\linebreak
Brown Betty\ixfill\pageref{brown_betty}\linebreak
Cabinet Pudding\ixfill\pageref{cabinet_pudding}\linebreak
Charlotte Russe, Easy\ixfill\pageref{easy_charlotte_russe}\linebreak
Coffee Jelly\ixfill\pageref{coffee_jelly}\linebreak
Corn-starch Pudding, Plain\ixfill\pageref{plain_cornstarch_pudding}\linebreak
Corn-starch Pudding, Chocolate\ixfill\pageref{chocolate_cornstarch_pudding}\linebreak
Corn-starch Pudding, Cocoanut\ixfill\pageref{cocoanut_cornstarch_pudding}\linebreak
Cottage Pudding\ixfill\pageref{cottage_pudding}\linebreak
Custard, Baked\ixfill\pageref{baked_custard}\linebreak
Custard, Cake and\ixfill\pageref{cake_and_custard}\linebreak
Custard, Cocoanut\ixfill\pageref{cocoanut_custard}\linebreak
Floating Island\ixfill\pageref{floating_island}\linebreak
Fruit Jelly\ixfill\pageref{fruit_jelly}\linebreak
Ice-creams and Ices:\\
\indix  Packing the Freezer\ixfill\pageref{packing_the_freezer}\linebreak
\indix  Chocolate Ice-cream\ixfill\pageref{chocolate_ice_cream}\linebreak
\indix  Coffee Ice-cream\ixfill\pageref{coffee_ice_cream}\linebreak
\indix  French Ice-cream\ixfill\pageref{french_ice_cream}\linebreak
\indix  Peach Ice-cream\ixfill\pageref{peach_ice_cream}\linebreak
\indix  Plain Ice-cream\ixfill\pageref{plain_ice_cream}\linebreak
\indix  Strawberry Ice-cream\ixfill\pageref{strawberry_ice_cream}\linebreak
\indix  Lemon Ice\ixfill\pageref{lemon_ice}\linebreak
\indix  Orange Ice\ixfill\pageref{orange_ice}\linebreak
\indix  Peach Surprise\ixfill\pageref{peach_surprise}\linebreak
\indix  Raspberry Ice\ixfill\pageref{raspberry_ice}\linebreak
\indix  Strawberry Ice\ixfill\pageref{strawberry_ice}\linebreak
\indix  Vanilla Parfait, the Easiest of All\ixfill\pageref{vanilla_parfait}\linebreak
Lemon Jelly\ixfill\pageref{lemon_jelly}\linebreak
Lemon Pudding\ixfill\pageref{lemon_pudding}\linebreak
Orange Jelly\ixfill\pageref{orange_jelly}\linebreak
Orange Pudding\ixfill\pageref{orange_pudding}\linebreak
Peach Shortcake\ixfill\pageref{peach_shortcake}\linebreak
Prune Jelly\ixfill\pageref{prune_jelly}\linebreak
Prune Whips\ixfill\pageref{prune_whips}\linebreak
Rice Pudding with Raisins\ixfill\pageref{rice_pudding_with_raisins}\linebreak
Snow Pudding\ixfill\pageref{snow_pudding}\linebreak
Strawberry Shortcake\ixfill\pageref{strawberry_shortcake}\linebreak
Strawberry Shortcake Made with Cake\ixfill\pageref{cake_shortcake}\linebreak
Tapioca Pudding\ixfill\pageref{tapioca_pudding}\linebreak
Pudding Sauces:\\
\indix  Foamy\ixfill\pageref{foamy_pudding_sauce}\linebreak
\indix  Grandmother's\ixfill\pageref{grandmothers_pudding_sauce}\linebreak
\indix  Hard\ixfill\pageref{hard_pudding_sauce}\linebreak
\indix  Lemon\ixfill\pageref{lemon_pudding_sauce}\linebreak
\indix  Orange\ixfill\pageref{orange_pudding_sauce}\linebreak
\indix  Maple, Delicious\ixfill\pageref{delicious_maple_pudding_sauce}\linebreak
\indix  Quick\ixfill\pageref{quick_pudding_sauce}\linebreak
\indix  White\ixfill\pageref{white_pudding_sauce}\linebreak
Velvet Cream\ixfill\pageref{velvet_cream}\linebreak
\begin{FoodTypeTitle}
EGGS
\end{FoodTypeTitle}
Baked in Little Dishes\ixfill\pageref{eggs_baked_in_little_dishes}\linebreak
Beds, Eggs in\ixfill\pageref{eggs_in_bed}\linebreak
Birds' Nests\ixfill\pageref{birds_nests}\linebreak
Boiled Eggs, Soft\ixfill\pageref{soft_boiled_eggs}\linebreak
Bacon, Eggs with\ixfill\pageref{eggs_with_bacon}\linebreak
Cheese, Eggs with\ixfill\pageref{eggs_with_cheese}\linebreak
Creamed Eggs\ixfill\pageref{creamed_eggs}\linebreak
Creamed in Baking-Dishes\ixfill\pageref{creamed_eggs_in_baking_dishes}\linebreak
Creamed on Toast\ixfill\pageref{creamed_eggs_on_toast}\linebreak
Devilled\ixfill\pageref{devilled_eggs}\linebreak
Double Cream with Eggs\ixfill\pageref{double_cream_with_eggs}\linebreak
Ham and Eggs, Moulded\ixfill\pageref{moulded_ham_and_eggs}\linebreak
Omelette\ixfill\pageref{omelette}\linebreak
Omelette with Mushrooms\ixfill\pageref{omelette_with_mushrooms}\linebreak
Omelette with Mushrooms and Olives\ixfill\pageref{omelette_with_mushrooms_and_olives}\linebreak
Omelette, Spanish\ixfill\pageref{spanish_omelette}\linebreak
Poached Eggs\ixfill\pageref{poached_eggs}\linebreak
Poached Eggs with Potted Ham\ixfill\pageref{poached_eggs_with_potted_ham}\linebreak
Scalloped\ixfill\pageref{scalloped_eggs}\linebreak
Scrambled\ixfill\pageref{scrambled_eggs}\linebreak
Scrambled with Parsley\ixfill\pageref{scrambled_eggs_with_parsley}\linebreak
Scrambled with Chicken\ixfill\pageref{scrambled_eggs_with_chicken}\linebreak
Scrambled with Tomato\ixfill\pageref{scrambled_eggs_with_tomato}\linebreak
\begin{FoodTypeTitle}
FISH
\end{FoodTypeTitle}
Codfish Balls\ixfill\pageref{fish_balls}\linebreak
Crab Meat in Shells\ixfill\pageref{crab_meat_in_shells}\linebreak
Creamed Codfish\ixfill\pageref{creamed_codfish}\linebreak
Creamed Fish\ixfill\pageref{creamed_fish}\linebreak
Creamed Lobster\ixfill\pageref{creamed_lobster}\linebreak
Creamed Salmon\ixfill\pageref{creamed_salmon}\linebreak
Fish-balls\ixfill\pageref{fish_balls}\linebreak
Mackerel, Salt\ixfill\pageref{salt_mackerel}\linebreak
Oysters, Creamed\ixfill\pageref{creamed_oysters}\linebreak
Oysters, Panned\ixfill\pageref{panned_oysters}\linebreak
Oyster Pigs in Blankets\ixfill\pageref{oyster_pigs_in_blankets}\linebreak
Oysters, Scalloped\ixfill\pageref{scalloped_oysters}\linebreak
Sardines, Broiled\ixfill\pageref{broiled_sardines}\linebreak
Scalloped Lobster or Salmon\ixfill\pageref{scalloped_lobster_or_salmon}\linebreak
Smelts, Fried\ixfill\pageref{fried_smelts}\linebreak
\begin{FoodTypeTitle}
MEATS
\end{FoodTypeTitle}
Bacon, Broiled\ixfill\pageref{broiled_bacon}\linebreak
Chicken or Turkey, Creamed\ixfill\pageref{creamed_chicken_or_turkey}\linebreak
Chicken Hash\ixfill\pageref{chicken_hash}\linebreak
Chicken, Pressed\ixfill\pageref{pressed_chicken}\linebreak
Chops, Broiled\ixfill\pageref{broiled_chops}\linebreak
Chops, Panned\ixfill\pageref{panned_chops}\linebreak
Cold\ixfill\pageref{cold_meats}\linebreak
Corned Beef Hash\ixfill\pageref{corned_beef_hash}\linebreak
Dried Beef, Frizzled\ixfill\pageref{frizzled_dried_beef}\linebreak
Liver and Bacon\ixfill\pageref{liver_and_bacon}\linebreak
Liver and Bacon on Skewers\ixfill\pageref{liver_and_bacon_on_skewers}\linebreak
Shepherd's Pie\ixfill\pageref{shepherds_pie}\linebreak
Sliced with Gravy\ixfill\pageref{sliced_meat_with_gravy}\linebreak
Souffl\'{e}\ixfill\pageref{meat_souffle}\linebreak
Steak, Broiled\ixfill\pageref{broiled_steak}\linebreak
Steak with Bananas\ixfill\pageref{steak_with_bananas}\linebreak
Veal Cutlet\ixfill\pageref{veal_cutlet}\linebreak
Veal Loaf\ixfill\pageref{veal_loaf}\linebreak
\begin{FoodTypeTitle}
PIES
\end{FoodTypeTitle}
Apple Pie or Tart, Deep\ixfill\pageref{deep_apple_pie_or_tart}\linebreak
Cranberry\ixfill\pageref{cranberry_pie}\linebreak
General Rule\ixfill\pageref{pies_general_rule}\linebreak
Lemon\ixfill\pageref{lemon_pie}\linebreak
Orange\ixfill\pageref{orange_pie}\linebreak
Peach\ixfill\pageref{peach_pie}\linebreak
Peach Pie, French\ixfill\pageref{french_peach_pie}\linebreak
Pumpkin\ixfill\pageref{pumpkin_pie}\linebreak
Tarts\ixfill\pageref{tarts}\linebreak
\begin{FoodTypeTitle}
POTATOES
\end{FoodTypeTitle}
Cakes\ixfill\pageref{potato_cakes}\linebreak
Creamed\ixfill\pageref{creamed_potatoes}\linebreak
Hashed Browned\ixfill\pageref{hashed_browned_potatoes}\linebreak
Mashed\ixfill\pageref{mashed_potatoes}\linebreak
Saratoga\ixfill\pageref{saratoga_potatoes}\linebreak
Stuffed\ixfill\pageref{stuffed_potatoes}\linebreak
Sweet Potatoes\ixfill\pageref{sweet_potatoes}\linebreak
\indix  Creamed\ixfill\pageref{creamed_sweet_potatoes}\linebreak
\indix  Fried\ixfill\pageref{fried_sweet_potatoes}\linebreak
\indix  Scalloped\ixfill\pageref{scalloped_sweet_potatoes}\linebreak
\begin{FoodTypeTitle}
SALADS
\end{FoodTypeTitle}
Cabbage\ixfill\pageref{cabbage_salad}\linebreak
Cabbage in Green Peppers\ixfill\pageref{cabbage_in_green_peppers_salad}\linebreak
Cauliflower\ixfill\pageref{cauliflower_salad}\linebreak
Celery\ixfill\pageref{celery_salad}\linebreak
Celery and Apple\ixfill\pageref{celery_and_apple_salad}\linebreak
Chicken\ixfill\pageref{chicken_salad}\linebreak
Egg\ixfill\pageref{egg_salad}\linebreak
Fish\ixfill\pageref{fish_salad}\linebreak
Lobster\ixfill\pageref{lobster_salad}\linebreak
Orange or Grapefruit\ixfill\pageref{orange_or_grapefruit_salad}\linebreak
Pineapple\ixfill\pageref{pineapple_salad}\linebreak
Potato\ixfill\pageref{potato_salad}\linebreak
String Bean\ixfill\pageref{string_bean_salad}\linebreak
Tomato and Lettuce\ixfill\pageref{tomato_and_lettuce_salad}\linebreak
Tomato, Stuffed\ixfill\pageref{stuffed_tomato_salad}\linebreak
Salad Dressings:\\
\indix  French\ixfill\pageref{french_dressing}\linebreak
\indix  Mayonnaise\ixfill\pageref{mayonnaise}\linebreak
\begin{FoodTypeTitle}
SANDWICHES
\end{FoodTypeTitle}
Celery\ixfill\pageref{celery_sandwiches}\linebreak
Cream Cheese and Nut\ixfill\pageref{cream_cheese_and_nut_sandwiches}\linebreak
Chicken and Celery\ixfill\pageref{chicken_and_celery_sandwiches}\linebreak
Egg\ixfill\pageref{egg_sandwiches}\linebreak
Lettuce\ixfill\pageref{lettuce_sandwiches}\linebreak
Nut\ixfill\pageref{nut_sandwiches}\linebreak
Olive\ixfill\pageref{olive_sandwiches}\linebreak
Sardine\ixfill\pageref{sardine_sandwiches}\linebreak
Sweet\ixfill\pageref{sweet_sandwiches}\linebreak
Tomato and Cheese\ixfill\pageref{tomato_and_cheese_sandwiches}\linebreak
\ \linebreak
Sauce: White or Cream\ixfill\pageref{white_or_cream_sauce}\linebreak
School Luncheons\ixfill\pageref{school_luncheons}\linebreak
\begin{FoodTypeTitle}
SOUPS
\end{FoodTypeTitle}
Cream Soup, General Rule\ixfill\pageref{general_rule_cream_soup}\linebreak
Cream of Almonds\ixfill\pageref{cream_of_almonds}\linebreak
Cream of Clams\ixfill\pageref{cream_of_clams}\linebreak
Cream of Corn\ixfill\pageref{cream_of_corn}\linebreak
Cream of Green Peas\ixfill\pageref{cream_of_green_peas}\linebreak
Cream of Lima Beans\ixfill\pageref{cream_of_lima_beans}\linebreak
Cream of Oysters\ixfill\pageref{oyster_soup}\linebreak
Cream of Potato\ixfill\pageref{cream_of_potato}\linebreak
Cream of Spinach\ixfill\pageref{cream_of_spinach}\linebreak
Cream of Tomato (Tomato Bisque)\ixfill\pageref{cream_of_tomato_soup}\linebreak
Meat Soups\ixfill\pageref{meat_soups}\linebreak
\indix  Bouillon, Cream\ixfill\pageref{cream_bouillon}\linebreak
\indix  Extract, Made from\ixfill\pageref{meat_soup_made_from_extract}\linebreak
\indix  Chicken or Turkey\ixfill\pageref{chicken_or_turkey_soup}\linebreak
\indix  Made with Cooked Meats\ixfill\pageref{soup_made_with_cooked_meats}\linebreak
Oyster Soup\ixfill\pageref{oyster_soup}\linebreak
Pea, Split\ixfill\pageref{split_pea_soup}\linebreak
Plain Meat\ixfill\pageref{plain_meat_soup}\linebreak
Tomato\ixfill\pageref{tomato_soup}\linebreak
Vegetable, Clear\ixfill\pageref{clear_vegetable_soup}\linebreak
\begin{FoodTypeTitle}
VEGETABLES
\end{FoodTypeTitle}
Asparagus\ixfill\pageref{asparagus}\linebreak
Beans, Lima\ixfill\pageref{lima_beans}\linebreak
Beans, String\ixfill\pageref{string_beans}\linebreak
Beets\ixfill\pageref{beets}\linebreak
Beets, Stuffed\ixfill\pageref{stuffed_beets}\linebreak
Cabbage, Creamed\ixfill\pageref{creamed_cabbage}\linebreak
Corn\ixfill\pageref{corn}\linebreak
Corn, Canned\ixfill\pageref{canned_corn}\linebreak
Macaroni\ixfill\pageref{macaroni}\linebreak
Onions\ixfill\pageref{onions}\linebreak
Peas\ixfill\pageref{peas}\linebreak
Tomatoes, Baked\ixfill\pageref{baked_tomatoes}\linebreak
Tomatoes, Stewed\ixfill\pageref{stewed_tomatoes}\linebreak
\newpage
\setlength{\headheight}{0ex}\setlength{\headsep}{0ex}
\addtolength{\textheight}{0.5in}\pagestyle{empty}{\tiny
\noindent *** END OF THE PROJECT GUTENBERG EBOOK A LITTLE COOK BOOK FOR A LITTLE GIRL ***

\smallskip\noindent ******* This file should be named 16514-p.pdf or 16514-p.zip *******

\smallskip\noindent This and all associated files of various formats will be found in:\\
\hspace*{\fill}{\em http://www.gutenberg.org/dirs/1/6/5/1/16514}\hspace*{\fill}\linebreak[4]

\noindent Updated editions will replace the previous one--the old editions
will be renamed.

\smallskip\noindent Creating the works from public domain print editions means that no
one owns a United States copyright in these works, so the Foundation
(and you!) can copy and distribute it in the United States without
permission and without paying copyright royalties.  Special rules,
set forth in the General Terms of Use part of this license, apply to
copying and distributing \pgtm\ electronic works to
protect the \PGtm\ concept and trademark.  Project
Gutenberg is a registered trademark, and may not be used if you
charge for the eBooks, unless you receive specific permission.  If you
do not charge anything for copies of this eBook, complying with the
rules is very easy.  You may use this eBook for nearly any purpose
such as creation of derivative works, reports, performances and
research.  They may be modified and printed and given away--you may do
practically ANYTHING with public domain eBooks.  Redistribution is
subject to the trademark license, especially commercial
redistribution.



\bigskip\noindent *** START: FULL LICENSE ***

\smallskip\noindent THE FULL PROJECT GUTENBERG LICENSE\\
PLEASE READ THIS BEFORE YOU DISTRIBUTE OR USE THIS WORK

\smallskip\noindent To protect the \pgtm\ mission of promoting the free
distribution of electronic works, by using or distributing this work
(or any other work associated in any way with the phrase "Project
Gutenberg"), you agree to comply with all the terms of the Full
\pgtm\ License (available with this file or online at
{\em http://gutenberg.net/license}).


\medskip\noindent Section 1.  General Terms of Use \& Redistributing \pgtm\ electronic works

\smallskip\noindent 1.A.  By reading or using any part of this \pgtm\ 
electronic work, you indicate that you have read, understand, agree to
and accept all the terms of this license and intellectual property
(trademark/copyright) agreement.  If you do not agree to abide by all
the terms of this agreement, you must cease using and return or destroy
all copies of \pgtm\ electronic works in your possession.
If you paid a fee for obtaining a copy of or access to a
\pgtm\ electronic work and you do not agree to be bound by the
terms of this agreement, you may obtain a refund from the person or
entity to whom you paid the fee as set forth in paragraph 1.E.8.

\smallskip\noindent 1.B.  "Project Gutenberg" is a registered trademark.  It may only be
used on or associated in any way with an electronic work by people who
agree to be bound by the terms of this agreement.  There are a few
things that you can do with most \pgtm\ electronic works
even without complying with the full terms of this agreement.  See
paragraph 1.C below.  There are a lot of things you can do with
\pgtm\ electronic works if you follow the terms of this agreement
and help preserve free future access to \pgtm\ electronic
works.  See paragraph 1.E below.

\smallskip\noindent 1.C.  The Project Gutenberg Literary Archive Foundation ("the Foundation"
or PGLAF), owns a compilation copyright in the collection of
\pgtm\ electronic works.  Nearly all the individual works in the
collection are in the public domain in the United States.  If an
individual work is in the public domain in the United States and you are
located in the United States, we do not claim a right to prevent you from
copying, distributing, performing, displaying or creating derivative
works based on the work as long as all references to Project Gutenberg
are removed.  Of course, we hope that you will support the
\pgtm\ mission of promoting free access to electronic works by
freely sharing \pgtm\ works in compliance with the terms of
this agreement for keeping the \pgtm\ name associated with
the work.  You can easily comply with the terms of this agreement by
keeping this work in the same format with its attached full
\pgtm\ License when you share it without charge with others.

\smallskip\noindent 1.D.  The copyright laws of the place where you are located also govern
what you can do with this work.  Copyright laws in most countries are in
a constant state of change.  If you are outside the United States, check
the laws of your country in addition to the terms of this agreement
before downloading, copying, displaying, performing, distributing or
creating derivative works based on this work or any other
\pgtm\ work.  The Foundation makes no representations concerning
the copyright status of any work in any country outside the United
States.

\smallskip\noindent 1.E.  Unless you have removed all references to Project Gutenberg:

\smallskip\noindent 1.E.1.  The following sentence, with active links to, or other immediate
access to, the full \pgtm\ License must appear prominently
whenever any copy of a \pgtm\ work (any work on which the
phrase "Project Gutenberg" appears, or with which the phrase "Project
Gutenberg" is associated) is accessed, displayed, performed, viewed,
copied or distributed:

\smallskip\noindent This eBook is for the use of anyone anywhere at no cost and with
almost no restrictions whatsoever.  You may copy it, give it away or
re-use it under the terms of the Project Gutenberg License included
with this eBook or online at {\em www.gutenberg.net}

\smallskip\noindent 1.E.2.  If an individual \pgtm\ electronic work is derived
from the public domain (does not contain a notice indicating that it is
posted with permission of the copyright holder), the work can be copied
and distributed to anyone in the United States without paying any fees
or charges.  If you are redistributing or providing access to a work
with the phrase "Project Gutenberg" associated with or appearing on the
work, you must comply either with the requirements of paragraphs 1.E.1
through 1.E.7 or obtain permission for the use of the work and the
\pgtm\ trademark as set forth in paragraphs 1.E.8 or
1.E.9.

\smallskip\noindent 1.E.3.  If an individual \pgtm\ electronic work is posted
with the permission of the copyright holder, your use and distribution
must comply with both paragraphs 1.E.1 through 1.E.7 and any additional
terms imposed by the copyright holder.  Additional terms will be linked
to the \pgtm\ License for all works posted with the
permission of the copyright holder found at the beginning of this work.

\smallskip\noindent 1.E.4.  Do not unlink or detach or remove the full
\pgtm\ License terms from this work, or any files containing a part of this
work or any other work associated with \pgtm.

\smallskip\noindent 1.E.5.  Do not copy, display, perform, distribute or redistribute this
electronic work, or any part of this electronic work, without
prominently displaying the sentence set forth in paragraph 1.E.1 with
active links or immediate access to the full terms of the
\pgtm\ License.

\smallskip\noindent 1.E.6.  You may convert to and distribute this work in any binary,
compressed, marked up, nonproprietary or proprietary form, including any
word processing or hypertext form.  However, if you provide access to or
distribute copies of a \pgtm\ work in a format other than
"Plain Vanilla ASCII" or other format used in the official version
posted on the official \pgtm\ web site ({\em www.gutenberg.net}),
you must, at no additional cost, fee or expense to the user, provide a
copy, a means of exporting a copy, or a means of obtaining a copy upon
request, of the work in its original "Plain Vanilla ASCII" or other
form.  Any alternate format must include the full
\pgtm\ License as specified in paragraph 1.E.1.

\smallskip\noindent 1.E.7.  Do not charge a fee for access to, viewing, displaying,
performing, copying or distributing any \pgtm\ works
unless you comply with paragraph 1.E.8 or 1.E.9.

\smallskip\noindent 1.E.8.  You may charge a reasonable fee for copies of or providing
access to or distributing \pgtm\ electronic works provided
that\vspace*{-\smallskipamount}
\begin{list}{}{\setlength{\leftmargin}{2.5em}\setlength{\rightmargin}{0em}
\setlength{\itemindent}{-2.5em}\setlength{\labelwidth}{0em}
\setlength{\labelsep}{0em}\setlength{\parsep}{\smallskipamount}}
\item ---You pay a royalty fee of 20\% of the gross profits you derive from
     the use of \pgtm\ works calculated using the method
     you already use to calculate your applicable taxes.  The fee is
     owed to the owner of the \pgtm\ trademark, but he
     has agreed to donate royalties under this paragraph to the
     Project Gutenberg Literary Archive Foundation.  Royalty payments
     must be paid within 60 days following each date on which you
     prepare (or are legally required to prepare) your periodic tax
     returns.  Royalty payments should be clearly marked as such and
     sent to the Project Gutenberg Literary Archive Foundation at the
     address specified in Section 4, "Information about donations to
     the Project Gutenberg Literary Archive Foundation."
\item ---You provide a full refund of any money paid by a user who notifies
     you in writing (or by e-mail) within 30 days of receipt that s/he
     does not agree to the terms of the full
     \pgtm\ License.  You must require such a user to return or
     destroy all copies of the works possessed in a physical medium
     and discontinue all use of and all access to other copies of
     \pgtm\ works.
\item ---You provide, in accordance with paragraph 1.F.3, a full refund of any
     money paid for a work or a replacement copy, if a defect in the
     electronic work is discovered and reported to you within 90 days
     of receipt of the work.

\item ---You comply with all other terms of this agreement for free
     distribution of \pgtm\ works.
\end{list}
\noindent 1.E.9.  If you wish to charge a fee or distribute a
\pgtm\ electronic work or group of works on different terms than are set
forth in this agreement, you must obtain permission in writing from
both the Project Gutenberg Literary Archive Foundation and Michael
Hart, the owner of the \pgtm\ trademark.  Contact the
Foundation as set forth in Section 3 below.

\smallskip\noindent 1.F.

\smallskip\noindent 1.F.1.  Project Gutenberg volunteers and employees expend considerable
effort to identify, do copyright research on, transcribe and proofread
public domain works in creating the
\pgtm\ collection.  Despite these efforts, \pgtm\ electronic
works, and the medium on which they may be stored, may contain
"Defects," such as, but not limited to, incomplete, inaccurate or
corrupt data, transcription errors, a copyright or other intellectual
property infringement, a defective or damaged disk or other medium, a
computer virus, or computer codes that damage or cannot be read by
your equipment.

\smallskip\noindent 1.F.2.  LIMITED WARRANTY, DISCLAIMER OF DAMAGES - Except for the "Right
of Replacement or Refund" described in paragraph 1.F.3, the Project
Gutenberg Literary Archive Foundation, the owner of the
\pgtm\ trademark, and any other party distributing a
\pgtm\ electronic work under this agreement, disclaim all
liability to you for damages, costs and expenses, including legal
fees.  YOU AGREE THAT YOU HAVE NO REMEDIES FOR NEGLIGENCE, STRICT
LIABILITY, BREACH OF WARRANTY OR BREACH OF CONTRACT EXCEPT THOSE
PROVIDED IN PARAGRAPH F3.  YOU AGREE THAT THE FOUNDATION, THE
TRADEMARK OWNER, AND ANY DISTRIBUTOR UNDER THIS AGREEMENT WILL NOT BE
LIABLE TO YOU FOR ACTUAL, DIRECT, INDIRECT, CONSEQUENTIAL, PUNITIVE OR
INCIDENTAL DAMAGES EVEN IF YOU GIVE NOTICE OF THE POSSIBILITY OF SUCH
DAMAGE.

\smallskip\noindent 1.F.3.  LIMITED RIGHT OF REPLACEMENT OR REFUND - If you discover a
defect in this electronic work within 90 days of receiving it, you can
receive a refund of the money (if any) you paid for it by sending a
written explanation to the person you received the work from.  If you
received the work on a physical medium, you must return the medium with
your written explanation.  The person or entity that provided you with
the defective work may elect to provide a replacement copy in lieu of a
refund.  If you received the work electronically, the person or entity
providing it to you may choose to give you a second opportunity to
receive the work electronically in lieu of a refund.  If the second copy
is also defective, you may demand a refund in writing without further
opportunities to fix the problem.

\smallskip\noindent 1.F.4.  Except for the limited right of replacement or refund set forth
in paragraph 1.F.3, this work is provided to you 'AS-IS,' WITH NO OTHER
WARRANTIES OF ANY KIND, EXPRESS OR IMPLIED, INCLUDING BUT NOT LIMITED TO
WARRANTIES OF MERCHANTIBILITY OR FITNESS FOR ANY PURPOSE.

\smallskip\noindent 1.F.5.  Some states do not allow disclaimers of certain implied
warranties or the exclusion or limitation of certain types of damages.
If any disclaimer or limitation set forth in this agreement violates the
law of the state applicable to this agreement, the agreement shall be
interpreted to make the maximum disclaimer or limitation permitted by
the applicable state law.  The invalidity or unenforceability of any
provision of this agreement shall not void the remaining provisions.

\smallskip\noindent 1.F.6.  INDEMNITY - You agree to indemnify and hold the Foundation, the
trademark owner, any agent or employee of the Foundation, anyone
providing copies of \pgtm\ electronic works in accordance
with this agreement, and any volunteers associated with the production,
promotion and distribution of \pgtm\ electronic works,
harmless from all liability, costs and expenses, including legal fees,
that arise directly or indirectly from any of the following which you do
or cause to occur: (a) distribution of this or any \pgtm\ work,
(b) alteration, modification, or additions or deletions to any
\pgtm\ work, and (c) any Defect you cause.


\medskip\noindent Section  2.  Information about the Mission of \pgtm

\smallskip\noindent \pgtm\ is synonymous with the free distribution of
electronic works in formats readable by the widest variety of computers
including obsolete, old, middle-aged and new computers.  It exists
because of the efforts of hundreds of volunteers and donations from
people in all walks of life.

\smallskip\noindent Volunteers and financial support to provide volunteers with the
assistance they need, is critical to reaching \pgtm's
goals and ensuring that the \pgtm\ collection will
remain freely available for generations to come.  In 2001, the Project
Gutenberg Literary Archive Foundation was created to provide a secure
and permanent future for \pgtm\ and future generations.
To learn more about the Project Gutenberg Literary Archive Foundation
and how your efforts and donations can help, see Sections 3 and 4
and the Foundation web page at {\em http://www.pglaf.org}.


\medskip\noindent Section 3.  Information about the Project Gutenberg Literary Archive
Foundation

\smallskip\noindent The Project Gutenberg Literary Archive Foundation is a non profit
501(c)(3) educational corporation organized under the laws of the
state of Mississippi and granted tax exempt status by the Internal
Revenue Service.  The Foundation's EIN or federal tax identification
number is 64-6221541.  Its 501(c)(3) letter is posted at
{\em http://pglaf.org/fundraising}.  Contributions to the Project Gutenberg
Literary Archive Foundation are tax deductible to the full extent
permitted by U.S. federal laws and your state's laws.

\smallskip\noindent The Foundation's principal office is located at 4557 Melan Dr. S.
Fairbanks, AK, 99712, but its volunteers and employees are scattered
throughout numerous locations.  Its business office is located at
809 North 1500 West, Salt Lake City, UT 84116, (801) 596-1887, email
{\em business@pglaf.org}.  Email contact links and up to date contact
information can be found at the Foundation's web site and official
page at {\em http://pglaf.org}

\smallskip\noindent For additional contact information:\\
\hspace*{3em}  Dr. Gregory B. Newby\\
\hspace*{3em}  Chief Executive and Director\\
\hspace*{3em}  {\em gbnewby@pglaf.org}

\medskip\noindent Section 4.  Information about Donations to the Project Gutenberg
Literary Archive Foundation

\smallskip\noindent \pgtm\ depends upon and cannot survive without widespread
public support and donations to carry out its mission of
increasing the number of public domain and licensed works that can be
freely distributed in machine readable form accessible by the widest
array of equipment including outdated equipment.  Many small donations
(\$1 to \$5,000) are particularly important to maintaining tax exempt
status with the IRS.

\smallskip\noindent The Foundation is committed to complying with the laws regulating
charities and charitable donations in all 50 states of the United
States.  Compliance requirements are not uniform and it takes a
considerable effort, much paperwork and many fees to meet and keep up
with these requirements.  We do not solicit donations in locations
where we have not received written confirmation of compliance.  To
SEND DONATIONS or determine the status of compliance for any
particular state visit {\em http://www.gutenberg.net/fundraising/donate}

\smallskip\noindent While we cannot and do not solicit contributions from states where we
have not met the solicitation requirements, we know of no prohibition
against accepting unsolicited donations from donors in such states who
approach us with offers to donate.

\smallskip\noindent International donations are gratefully accepted, but we cannot make
any statements concerning tax treatment of donations received from
outside the United States.  U.S. laws alone swamp our small staff.

\smallskip\noindent Please check the Project Gutenberg Web pages for current donation
methods and addresses.  Donations are accepted in a number of other
ways including checks, online payments and credit card donations.  To
donate, please visit: {\em http://www.gutenberg.net/fundraising/donate}


\medskip\noindent Section 5.  General Information About \pgtm\ electronic
works.

\smallskip\noindent Professor Michael S. Hart is the originator of the \pgtm\ 
concept of a library of electronic works that could be freely shared
with anyone.  For thirty years, he produced and distributed
\pgtm\ eBooks with only a loose network of volunteer support.

\smallskip\noindent \pgtm\ eBooks are often created from several printed
editions, all of which are confirmed as Public Domain in the U.S.
unless a copyright notice is included.  Thus, we do not necessarily
keep eBooks in compliance with any particular paper edition.

\smallskip\noindent Most people start at our Web site which has the
main PG search facility:

\smallskip\noindent\hspace*{3em}\em{http://www.gutenberg.net}

\smallskip\noindent This Web site includes information about \pgtm,
including how to make donations to the Project Gutenberg Literary Archive
Foundation, how to help produce our new eBooks, and how to subscribe
to our email newsletter to hear about new eBooks.

\smallskip\noindent *** END: FULL LICENSE ***}
\end{document}
